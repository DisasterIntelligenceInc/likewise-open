The main functions deal with the nitty-gritty details: verifying
tickets, creating authenticators, and the like.

\subsubsection{The krb5_context}
The \datatype{krb5_context} is designed to represent the per process
state. When the library is made thread-safe, the context will represent
the per-thread state. Global parameters which are ``context'' specific
are stored in this structure, including default realm, default
encryption type, default configuration files and the like. Functions
exist to provide full access to the data structures stored in the
context and should not be accessed directly by developers.

\begin{funcdecl}{krb5_init_context}{krb5_error_code}{\funcout}
\funcarg{krb5_context *}{context}
\end{funcdecl}

Initializes the context \funcparam{*context} for the
application. Currently the context contains the encryption types, a
pointer to operating specific data and the default realm. In the future,
the context may be also contain thread specific data.  The data in the
context should be freed with \funcname{krb5_free_context}.

Returns system errors.

\begin{funcdecl}{krb5_free_context}{void}{\funcinout}
\funcarg{krb5_context}{context}
\end{funcdecl}

Frees the context returned by \funcname{krb5_init_context}. Internally
calls \funcname{krb5_os_free_context}.

\begin{funcdecl}{krb5_set_default_in_tkt_etypes}{krb5_error_code}{\funcinout}
\funcarg{krb5_context}{context}
\funcin
\funcarg{const krb5_enctype *}{etypes}
\end{funcdecl}

Sets the desired default encryption type \funcparam{etypes} for the context
if valid.

Returns {\sc enomem}, {\sc krb5_prog_etype_nosupp}.

\begin{funcdecl}{krb5_get_default_in_tkt_etypes}{krb5_error_code}{\funcinout}
\funcarg{krb5_context}{context}
\funcout
\funcarg{krb5_enctype **}{etypes}
\end{funcdecl}

Retrieves the default encryption types from the context and stores them
in \funcarg{etypes} which should be freed by the caller.

Returns {\sc enomem}.

\subsubsection{The krb5_auth_context}

While the \datatype{krb5_context} represents a per-process or per-thread
context, the \datatype{krb5_auth_context} represents a per-connection
context are are used by the various functions involved directly in
client/server authentication.  Some of the data stored in this context
include keyblocks, addresses, sequence numbers, authenticators, checksum
type, and replay cache pointer. 

\begin{funcdecl}{krb5_auth_con_init}{krb5_error_code}{\funcinout}
\funcarg{krb5_context}{context}
\funcout
\funcarg{krb5_auth_context *}{auth_context}
\end{funcdecl}

The auth_context may be described as a per connection context. This
context contains all data pertinent to the the various authentication
routines. This function initializes the auth_context. 

The default flags for the context are set to enable the use of the replay cache
(KRB5_AUTH_CONTEXT_DO_TIME) but no sequence numbers. The function
\funcname{krb5_auth_con_setflags} allows the flags to be changed.

The default checksum type is set to CKSUMTYPE_RSA_MD4_DES. This may be
changed with \funcname{krb5_auth_con_setcksumtype}.

The auth_context structure should be freed with
\funcname{krb5_auth_con_free}.

\begin{funcdecl}{krb5_auth_con_free}{krb5_error_code}{\funcinout}
\funcarg{krb5_context}{context}
\funcarg{krb5_auth_context}{auth_context}
\end{funcdecl}

Frees the auth_context \funcparam{auth_context} returned by
\funcname{krb5_auth_con_init}.

% perhaps some comment about which substructures are freed and which are not?

\begin{funcdecl}{krb5_auth_con_setflags}{krb5_error_code}{\funcinout}
\funcarg{krb5_context}{context}
\funcarg{krb5_auth_context}{auth_context}
\funcin
\funcarg{krb5_int32}{flags}
\end{funcdecl}

Sets the flags of \funcparam{auth_context} to funcparam{flags}. Valid
flags are:

\begin{tabular}{ll}
\multicolumn{1}{c}{Symbol} & Meaning \\
KRB5_AUTH_CONTEXT_DO_TIME & Use timestamps \\
KRB5_AUTH_CONTEXT_RET_TIME & Save timestamps\\ &\  to output structure\\
KRB5_AUTH_CONTEXT_DO_SEQUENCE   & Use sequence numbers \\
KRB5_AUTH_CONTEXT_RET_SEQUENCE  & Copy sequence numbers \\ &\ to output structure\\
\end{tabular}


\begin{funcdecl}{krb5_auth_con_getflags}{krb5_error_code}{\funcinout}
\funcarg{krb5_context}{context}
\funcin
\funcarg{krb5_auth_context}{auth_context}
\funcout
\funcarg{krb5_int32 *}{flags}
\end{funcdecl}

Retrievs the flags of \funcparam{auth_context}.

\begin{funcdecl}{krb5_auth_con_setaddrs}{krb5_error_code}{\funcinout}
\funcarg{krb5_context}{context}
\funcarg{krb5_auth_context}{auth_context}
\funcin
\funcarg{krb5_address *}{local_addr}
\funcarg{krb5_address *}{remote_addr}
\end{funcdecl}

Copies the \funcparam{local_addr} and \funcparam{remote_addr} into the
\funcparam{auth_context}. If either address is NULL, the previous
address remains in place.

\begin{funcdecl}{krb5_auth_con_getaddrs}{krb5_error_code}{\funcinout}
\funcarg{krb5_context}{context}
\funcarg{krb5_auth_context}{auth_context}
\funcout
\funcarg{krb5_address **}{local_addr}
\funcarg{krb5_address **}{remote_addr}
\end{funcdecl}

Retrieves  \funcparam{local_addr} and \funcparam{remote_addr} from the
\funcparam{auth_context}. If \funcparam{local_addr} or
\funcparam{remote_addr} is not NULL, the memory is first freed with
\funcname{krb5_free_address} and then newly allocated. It is the callers
responsibility to free the returned addresses in this way.


\begin{funcdecl}{krb5_auth_con_setports}{krb5_error_code}{\funcinout}
\funcarg{krb5_context}{context}
\funcarg{krb5_auth_context}{auth_context}
\funcin
\funcarg{krb5_address *}{local_port}
\funcarg{krb5_address *}{remote_port}
\end{funcdecl}

Copies the \funcparam{local_port} and \funcparam{remote_port} addresses
into the \funcparam{auth_context}. If either address is NULL, the previous
address remains in place. These addresses are set by
\funcname{krb5_auth_con_genaddrs}. 

\begin{funcdecl}{krb5_auth_con_setuserkey}{krb5_error_code}{\funcinout}
\funcarg{krb5_context}{context}
\funcarg{krb5_auth_context}{auth_context}
\funcin
\funcarg{krb5_keyblock *}{keyblock}
\end{funcdecl}

This function overloads the keyblock field. It is only useful prior to a
\funcname{krb5_rd_req_decode} call for user to user authentication where
the server has the key and needs to use it to decrypt the incoming
request.  Once decrypted this key is no longer necessary and is then
overwritten with the session key sent by the client. 

\begin{funcdecl}{krb5_auth_con_getkey}{krb5_error_code}{\funcinout}
\funcarg{krb5_context}{context}
\funcarg{krb5_auth_context}{auth_context}
\funcout
\funcarg{krb5_keyblock **}{keyblock}
\end{funcdecl}

Retrieves the keyblock stored in \funcparam{auth_context}. The memory
allocated in this function should be freed with a call to
\funcname{krb5_free_keyblock}. 

\begin{funcdecl}{krb5_auth_con_getrecvsubkey}{krb5_error_code}{\funcinout}
\funcarg{krb5_context}{context}
\funcarg{krb5_auth_context}{auth_context}
\funcout
\funcarg{krb5_keyblock **}{keyblock}
\end{funcdecl}

Retrieves the recv\_subkey keyblock stored in
\funcparam{auth_context}. The memory allocated in this function should
be freed with a call to \funcname{krb5_free_keyblock}.

\begin{funcdecl}{krb5_auth_con_getsendsubkey}{krb5_error_code}{\funcinout}
\funcarg{krb5_context}{context}
\funcarg{krb5_auth_context}{auth_context}
\funcout
\funcarg{krb5_keyblock **}{keyblock}
\end{funcdecl}

Retrieves the send\_subkey keyblock stored in
\funcparam{auth_context}. The memory allocated in this function should
be freed with a call to \funcname{krb5_free_keyblock}.

\begin{funcdecl}{krb5_auth_con_setrecvsubkey}{krb5_error_code}{\funcinout}
\funcarg{krb5_context}{context}
\funcarg{krb5_auth_context}{auth_context}
\funcout
\funcarg{krb5_keyblock *}{keyblock}
\end{funcdecl}

Sets the recv\_subkey keyblock stored in \funcparam{auth_context}.

\begin{funcdecl}{krb5_auth_con_setsendsubkey}{krb5_error_code}{\funcinout}
\funcarg{krb5_context}{context}
\funcarg{krb5_auth_context}{auth_context}
\funcout
\funcarg{krb5_keyblock *}{keyblock}
\end{funcdecl}

Sets the send\_subkey keyblock stored in \funcparam{auth_context}.

\begin{funcdecl}{krb5_auth_setcksumtype}{krb5_error_code}{\funcinout}
\funcarg{krb5_context}{context}
\funcarg{krb5_auth_context}{auth_context}
\funcin
\funcarg{krb5_cksumtype}{cksumtype}
\end{funcdecl}

Sets the checksum type used by the other functions in the library. 

\begin{funcdecl}{krb5_auth_getlocalseqnumber}{krb5_error_code}{\funcinout}
\funcarg{krb5_context}{context}
\funcarg{krb5_auth_context}{auth_context}
\funcin
\funcarg{krb5_int32 *}{seqnumber}
\end{funcdecl}

Retrieves the local sequence number that was used during authentication
and stores it in \funcparam{seqnumber}.

\begin{funcdecl}{krb5_auth_getremoteseqnumber}{krb5_error_code}{\funcinout}
\funcarg{krb5_context}{context}
\funcarg{krb5_auth_context}{auth_context}
\funcin
\funcarg{krb5_int32 *}{seqnumber}
\end{funcdecl}

Retrieves the remote sequence number that was used during authentication
and stores it in \funcparam{seqnumber}.

\begin{funcdecl}{krb5_auth_getauthenticator}{krb5_error_code}{\funcinout}
\funcarg{krb5_context}{context}
\funcarg{krb5_auth_context}{auth_context}
\funcout
\funcarg{krb5_authenticator **}{authenticator}
\end{funcdecl}

Retrieves the authenticator that was used during mutual
authentication. It is the callers responsibility to free the memory
allocated to \funcparam{authenticator} by calling
\funcname{krb5_free_authenticator}. 

\begin{funcdecl}{krb5_auth_con_initivector}{krb5_error_code}{\funcinout}
\funcarg{krb5_context}{context}
\funcarg{krb5_auth_context}{auth_context}
\end{funcdecl}

Allocates memory for and zeros the initial vector in the
\funcparam{auth_context} keyblock.

\begin{funcdecl}{krb5_auth_con_setivector}{krb5_error_code}{\funcinout}
\funcarg{krb5_context}{context}
\funcarg{krb5_auth_context *}{auth_context}
\funcin
\funcarg{krb5_pointer}{ivector}
\end{funcdecl}

Sets the i_vector portion of \funcparam{auth_context} to
\funcparam{ivector}. 

\begin{funcdecl}{krb5_auth_con_setrcache}{krb5_error_code}{\funcinout}
\funcarg{krb5_context}{context}
\funcarg{krb5_auth_context}{auth_context}
\funcin
\funcarg{krb5_rcache}{rcache}
\end{funcdecl}

Sets the replay cache that is used by the authentication routines to \funcparam{rcache}.

%%
%% The following prototypes exist, but there is no code at this time
%% krb5_auth_con_getcksumtype, krb5_auth_con_getivector,
%% krb5_auth_con_getrcache. -- Ezra

\subsubsection{Principal access functions}

Principals define a uniquely named client or server instance that
participates in a network communication. The following functions allow
one to create, modify and access portions of the
\datatype{krb5_principal}. 

Other functions found in orther portions of the manual include
\funcname{krb5_sname_to_principal}, \funcname{krb5_free_principal}, 

While it is possible to directly access the data structure in the
structure, it is recommended that the functions be used. 

\begin{funcdecl}{krb5_parse_name}{krb5_error_code}{\funcinout}
\funcarg{krb5_context}{context}
\funcin
\funcarg{const char *}{name}
\funcout
\funcarg{krb5_principal *}{principal}
\end{funcdecl}

Converts a single-string representation \funcparam{name} of the
principal name to the multi-part principal format used in the protocols.

A single-string representation of a Kerberos name consists of one or
more principal name components, separated by slashes, optionally
followed by the ``@'' character and a realm name.  If the realm name
is not specified, the local realm is used.

The slash and ``@'' characters may be quoted (i.e., included as part
of a component rather than as a component separator or realm prefix)
by preceding them with a backslash (``$\backslash$'') character.
Similarly, newline, tab, backspace, and null characters may be
included in a component by using $\backslash{}n$, $\backslash{}t$,
$\backslash{}b$ or $\backslash{}0$, respectively.

The realm in a Kerberos name may not contain the slash, colon or null
characters.

\funcparam{*principal} will point to allocated storage which should be freed by
the caller (using \funcname{krb5_free_principal}) after use.

\funcname{krb5_parse_name} returns KRB5_PARSE_MALFORMED if the string is
 badly formatted, or ENOMEM if space for the return value can't be
allocated.

\begin{funcdecl}{krb5_unparse_name}{krb5_error_code}{\funcinout}
\funcarg{krb5_context}{context}
\funcin
\funcarg{krb5_const_principal}{principal}
\funcout
\funcarg{char **}{name}
\end{funcdecl}

Converts the multi-part principal name \funcparam{principal} from the
format used in the protocols to a single-string representation of the
name.  The resulting single-string representation will use the format
and quoting conventions described for \funcname{krb5_parse_name}
above.

\funcparam{*name} points to allocated storage and should be freed by the caller
when finished.

\funcname{krb5_unparse_name} returns {\sc KRB_PARSE_MALFORMED} if the principal
does not contain at least 2 components, and system errors ({\sc ENOMEM} if
unable to allocate memory).

\begin{funcdecl}{krb5_unparse_name_ext}{krb5_error_code}{\funcinout}
\funcarg{krb5_context}{context}
\funcin
\funcarg{krb5_const_principal}{principal}
\funcinout
\funcarg{char **}{name}
\funcarg{unsigned int *}{size}
\end{funcdecl}

\funcname{krb5_unparse_name_ext} is designed for applications which
must unparse a large number of principals, and are concerned about the
speed impact of needing to do a lot of memory allocations and
deallocations.  It functions similarly to \funcname{krb5_unparse_name}
except if \funcparam{*name} is non-null, in which case, it is assumed
to contain an allocated buffer of size \funcparam{*size} and this
buffer will be resized with \funcname{realloc} to hold the unparsed
name.  Note that in this case,
\funcparam{size} must not be null.  

If \funcparam{size} is non-null (whether or not \funcparam{*name} is
null when the function is called), it will be filled in with the size
of the unparsed name upon successful return.

\begin{funcdecl}{krb5_princ_realm}{krb5_data *}{\funcinout}
\funcparam{krb5_context}{context}
\funcparam{krb5_principal}{principal}
\end{funcdecl}

A macro which returns the realm of \funcparam{principal}.

\begin{funcdecl}{krb5_princ_set_realm}{void}{\funcinout}
\funcparam{krb5_context}{context}
\funcparam{krb5_principal}{principal}
\funcparam{krb5_data *}{realm}
\end{funcdecl}

A macro which returns sets the realm of \funcparam{principal} to
\funcparam{realm}. 

\begin{funcdecl}{krb5_princ_set_realm_data}{void}{\funcinout}
\funcparam{krb5_context}{context}
\funcparam{krb5_principal}{principal}
\funcparam{char *}{data}
\end{funcdecl}

\internalfunc

A macro which returns sets the data portion of the realm of
\funcparam{principal} to \funcparam{data}. 

\begin{funcdecl}{krb5_princ_set_realm_length}{void}{\funcinout}
\funcparam{krb5_context}{context}
\funcparam{krb5_principal}{principal}
\funcparam{int}{length}
\end{funcdecl}

\internalfunc

A macro which returns sets the length \funcparam{principal} to
\funcparam{length}. 

\begin{funcdecl}{krb5_princ_size}{void}{\funcinout}
\funcparam{krb5_context}{context}
\funcparam{krb5_principal}{principal}
\end{funcdecl}

\internalfunc

A macro which gives the number of elements in the principal.
May also be used on the left size of an assignment.

\begin{funcdecl}{krb5_princ_type}{void}{\funcinout}
\funcparam{krb5_context}{context}
\funcparam{krb5_principal}{principal}
\end{funcdecl}

\internalfunc

A macro which gives the type of the principal.
May also be used on the left size of an assignment.

\begin{funcdecl}{krb5_princ_data}{}
\funcparam{krb5_context}{context}
\funcparam{krb5_principal}{principal}
\end{funcdecl}

\internalfunc

A macro which gives the pointer to data portion of the principal.
May also be used on the left size of an assignment.


\begin{funcdecl}{krb5_princ_component}{}
\funcparam{krb5_context}{context}
\funcparam{krb5_principal}{principal}
\funcparam{int}{i}
\end{funcdecl}

\internalfunc

A macro which gives the pointer to ith element of the principal.
May also be used on the left size of an assignment.

\begin{funcdecl}{krb5_build_principal}{krb5_error_code}{\funcinout}
\funcarg{krb5_context}{context}
\funcout
\funcarg{krb5_principal *}{princ}
\funcin
\funcarg{unsigned int}{rlen}
\funcarg{const char *}{realm}
\funcarg{char}{*s1, *s2, ..., 0}
\end{funcdecl}
\begin{funcdecl}{krb5_build_principal_va}{krb5_error_code}{\funcinout}
\funcarg{krb5_context}{context}
\funcout
\funcarg{krb5_principal *}{princ}
\funcin
\funcarg{unsigned int}{rlen}
\funcarg{const char *}{realm}
\funcarg{va_list}{ap}
\end{funcdecl}

\begin{sloppypar}
\funcname{krb5_build_principal} and
\funcname{krb5_build_principal_va} 
perform the same function; the former takes variadic arguments, while
the latter takes a pre-computed varargs pointer.
\end{sloppypar}

Both functions take a realm name \funcparam{realm}, realm name length
\funcparam{rlen}, and a list of null-terminated strings, and fill in a
pointer to a principal structure \funcparam{princ}, making it point to a
structure representing the named principal.
The last string must be followed in the argument list by a null pointer.


\begin{funcdecl}{krb5_build_principal_ext}{krb5_error_code}{\funcinout}
\funcarg{krb5_context}{context}
\funcout
\funcarg{krb5_principal *}{princ}
\funcin
\funcarg{unsigned int}{rlen}
\funcarg{const char *}{realm}
\funcarg{}{int len1, char *s1, int len2, char *s2, ..., 0}
\end{funcdecl}

\funcname{krb5_build_principal_ext} is similar to
\funcname{krb5_build_principal} but it takes its components as a list of
(length, contents) pairs rather than a list of null-terminated strings.
A length of zero indicates the end of the list.

\begin{funcdecl}{krb5_copy_principal}{krb5_error_code}{\funcinout}
\funcarg{krb5_context}{context}
\funcin
\funcarg{krb5_const_principal}{inprinc}
\funcout
\funcarg{krb5_principal *}{outprinc}
\end{funcdecl}

Copy a principal structure, filling in \funcparam{*outprinc} to point to
the newly allocated copy, which should be freed with
\funcname{krb5_free_principal}.

\begin{funcdecl}{krb5_principal_compare}{krb5_boolean}{\funcinout}
\funcarg{krb5_context}{context}
\funcin
\funcarg{krb5_const_principal}{princ1}
\funcarg{krb5_const_principal}{princ2}
\end{funcdecl}

If the two principals are the same, return TRUE, else return FALSE.

\begin{funcdecl}{krb5_realm_compare}{krb5_boolean}{\funcinout}
\funcarg{krb5_context}{context}
\funcin
\funcarg{krb5_const_principal}{princ1}
\funcarg{krb5_const_principal}{princ2}
\end{funcdecl}

If the realms of the two principals are the same, return TRUE, else
return FALSE. 


\begin{funcdecl}{krb5_425_conv_principal}{krb5_error_code}{\funcinout}
\funcarg{krb5_context}{context}
\funcin
\funcarg{const char *}{name}
\funcarg{const char *}{instance}
\funcarg{const char *}{realm}
\funcout
\funcarg{krb5_principal *}{princ}
\end{funcdecl}

Build a principal \funcparam{princ} from a V4 specification made up of 
\funcparam{name}.\funcparam{instance}@\funcparam{realm}. The routine is
site-customized to convert the V4 naming scheme to a V5 one. For
instance, the V4 ``rcmd'' is changed to ``host''. 

The returned principal should be freed with
\funcname{krb5_free_principal}. 

\subsubsection{The application functions}

\begin{funcdecl}{krb5_encode_kdc_rep}{krb5_error_code}{\funcin}
\funcarg{const krb5_msgtype}{type}
\funcarg{const krb5_enc_kdc_rep_part *}{encpart}
\funcarg{krb5_encrypt_block *}{eblock}
\funcarg{const krb5_keyblock *}{client_key}
\funcinout
\funcarg{krb5_kdc_rep *}{dec_rep}
\funcout
\funcarg{krb5_data **}{enc_rep}
\end{funcdecl}

\internalfunc

Takes KDC rep parts in \funcparam{*rep} and \funcparam{*encpart}, and
formats it into \funcparam{*enc_rep}, using message type
\funcparam{type} and encryption key \funcparam{client_key} and
encryption block \funcparam{eblock}.

\funcparam{enc_rep{\ptsto}data} will point to  allocated storage upon
non-error return; the caller should free it when finished.

Returns system errors.

\begin{funcdecl}{krb5_decode_kdc_rep}{krb5_error_code}{\funcinout}
\funcarg{krb5_context}{context}
\funcin
\funcarg{krb5_data *}{enc_rep}
\funcarg{const krb5_keyblock *}{key}
\funcarg{const krb5_enctype}{etype}
\funcout
\funcarg{krb5_kdc_rep **}{dec_rep}
\end{funcdecl}

\internalfunc

Takes a KDC_REP message and decrypts encrypted part using
\funcparam{etype} and \funcparam{*key}, putting result in \funcparam{*dec_rep}.
The pointers in \funcparam{dec_rep}
are all set to allocated storage which should be freed by the caller
when finished with the response (by using \funcname{krb5_free_kdc_rep}).


If the response isn't a KDC_REP (tgs or as), it returns an error from
the decoding routines.

Returns errors from encryption routines, system errors.

\begin{funcdecl}{krb5_kdc_rep_decrypt_proc}{krb5_error_code}{\funcinout}
\funcarg{krb5_context}{context}
\funcin
\funcarg{const krb5_keyblock *}{key}
\funcarg{krb5_const_pointer}{decryptarg}
\funcinout
\funcarg{krb5_kdc_rep *}{dec_rep}
\end{funcdecl}

Decrypt the encrypted portion of \funcparam{dec_rep}, using the
encryption key \funcparam{key}.  The parameter \funcparam{decryptarg} is
ignored.

The result is in allocated storage pointed to by
\funcparam{dec_rep{\ptsto}enc_part2}, unless some error occurs.

This function is suitable for use as the \funcparam{decrypt_proc}
argument to \funcname{krb5_get_in_tkt}.

\begin{funcdecl}{krb5_encrypt_tkt_part}{krb5_error_code}{\funcinout}
\funcarg{krb5_context}{context}
\funcin
\funcarg{const krb5_keyblock *}{srv_key}
\funcinout
\funcarg{krb5_ticket *}{dec_ticket}
\end{funcdecl}

\internalfunc

\begin{sloppypar}
Encrypts the unecrypted part of the ticket found in 
\funcparam{dec_ticket{\ptsto}enc_part2} using
\funcparam{srv_key}, and places result in 
\funcparam{dec_ticket{\ptsto}enc_part}.
The \funcparam{dec_ticket{\ptsto}enc_part} will be allocated by this
function.
\end{sloppypar}

Returns errors from encryption routines, system errors

\funcparam{enc_part{\ptsto}data} is allocated and filled in with
encrypted stuff.

\begin{funcdecl}{krb5_decrypt_tkt_part}{krb5_error_code}{\funcinout}
\funcarg{krb5_context}{context}
\funcin
\funcarg{const krb5_keyblock *}{srv_key}
\funcinout
\funcarg{krb5_ticket *}{dec_ticket}
\end{funcdecl}

\internalfunc

Takes encrypted \funcparam{dec_ticket{\ptsto}enc_part}, decrypts with
\funcparam{dec_ticket{\ptsto}etype}
using \funcparam{srv_key}, and places result in
\funcparam{dec_ticket{\ptsto}enc_part2}.  The storage of
\funcparam{dec_ticket{\ptsto}enc_part2} will be allocated before return.

Returns errors from encryption routines, system errors

\begin{funcdecl}{krb5_send_tgs}{krb5_error_code}{\funcinout}
\funcarg{krb5_context}{context}
\funcin
\funcarg{const krb5_flags}{kdcoptions}
\funcarg{const krb5_ticket_times *}{timestruct}
\funcarg{const krb5_enctype *}{etypes}
\funcarg{const krb5_cksumtype}{sumtype}
\funcarg{krb5_const_principal}{sname}
\funcarg{krb5_address * const *}{addrs}
\funcarg{krb5_authdata * const *}{authorization_data}
\funcarg{krb5_pa_data * const *}{padata}
\funcarg{const krb5_data *}{second_ticket}
\funcinout
\funcarg{krb5_creds *}{in_cred}
\funcout
\funcarg{krb5_response *}{rep}
\end{funcdecl}

\internalfunc

Sends a request to the TGS and waits for a response.
\funcparam{kdcoptions} is used for the options in the KRB_TGS_REQ.
\funcparam{timestruct} values are used for from, till, and rtime in the
KRB_TGS_REQ.
\funcparam{etypes} is a list of etypes used in the KRB_TGS_REQ.
\funcparam{sumtype} is used for the checksum in the AP_REQ in the KRB_TGS_REQ.
\funcparam{sname} is used for sname in the KRB_TGS_REQ.
\funcparam{addrs}, if non-NULL, is used for addresses in the KRB_TGS_REQ.
\funcparam{authorization_data}, if non-NULL, is used for
\funcparam{authorization_data} in the KRB_TGS_REQ.  
\funcparam{padata}, if non-NULL, is combined with any other supplied
pre-authentication data for the KRB_TGS_REQ.
\funcparam{second_ticket}, if required by options, is used for the 2nd
ticket in the KRB_TGS_REQ.
\funcparam{in_cred} is used for the ticket and session key in the KRB_AP_REQ header in the KRB_TGS_REQ.

The KDC realm is extracted from \funcparam{in_cred{\ptsto}server}'s realm.

The response is placed into \funcparam{*rep}.
\funcparam{rep{\ptsto}response.data} is set to point at allocated storage
which should be freed by the caller when finished.

Returns system errors.

\begin{funcdecl}{krb5_get_cred_from_kdc}{krb5_error_code}{\funcinout}
\funcarg{krb5_context}{context}
\funcin
\funcarg{krb5_ccache}{ccache}
\funcarg{krb5_creds *}{in_cred}
\funcout                        
\funcarg{krb5_cred **}{out_cred}
\funcarg{krb5_creds ***}{tgts}
\end{funcdecl}


Retrieve credentials for principal \funcparam{in_cred{\ptsto}client},
server \funcparam{creds{\ptsto}server}, possibly
\funcparam{creds{\ptsto}second_ticket} if needed by the ticket flags.

\funcparam{ccache} is used to fetch initial TGT's to start the authentication
path to the server.

Credentials are requested from the KDC for the server's realm.  Any
TGT credentials obtained in the process of contacting the KDC are
returned in an array of credentials; \funcparam{tgts} is filled in to
point to an array of pointers to credential structures (if no TGT's were
used, the pointer is zeroed).  TGT's may be returned even if no useful
end ticket was obtained.

The returned credentials are NOT cached.

If credentials are obtained, \funcparam{creds} is filled in with the results;
\funcparam{creds{\ptsto}ticket} and
\funcparam{creds{\ptsto}keyblock{\ptsto}key} are set to allocated storage,
which should be freed by the caller when finished.

Returns errors, system errors.

\begin{funcdecl}{krb5_get_cred_via_tkt}{krb5_error_code}{\funcinout}
\funcarg{krb5_context}{context}
\funcin
\funcarg{krb5_creds *}{tkt}
\funcarg{const krb5_flags}{kdcoptions}
\funcarg{krb5_address *const *}{address}
\funcarg{krb5_creds *}{in_cred}
\funcout
\funcarg{krb5_creds **}{out_cred}
\end{funcdecl}

Takes a ticket \funcparam{tkt} and a target credential
\funcparam{in_cred}, attempts to fetch a TGS from the KDC. Upon
success the resulting is stored in \funcparam{out_cred}. The memory
allocated in \funcparam{out_cred} should be freed by the called when
finished by using \funcname{krb5_free_creds}. 

\funcparam{kdcoptions} refers to the options as listed in Table
\ref{KDCOptions}. The optional \funcparam{address} is used for addressed
in the KRB_TGS_REQ (see \funcname{krb5_send_tgs}).

Returns errors, system errors.


\begin{funcdecl}{krb5_get_credentials}{krb5_error_code}{\funcinout}
\funcarg{krb5_context}{context}
\funcin
\funcarg{const krb5_flags}{options}
\funcarg{krb5_ccache}{ccache}
\funcarg{krb5_creds *}{in_creds}
\funcout
\funcarg{krb5_creds *}{out_creds}
\end{funcdecl}

This routine attempts to use the credentials cache \funcparam{ccache} or a TGS
exchange to get an additional ticket for the client identified by
\funcparam{in_creds{\ptsto}client}, with following information: 
\begin{itemize}
\item {\bf The server} identified by \funcparam{in_creds{\ptsto}server} 
\item {\bf The options} in \funcparam{options}.
Valid choices are KRB5_GC_USER_USER and KRB5_GC_GC_CACHED
\item {\bf The expiration date} specified in
\funcparam{in_creds{\ptsto}times.endtime}  
\item {\bf The session key type} specified in
\funcparam{in_creds{\ptsto}keyblock.keytype} if it is non-zero.
\end{itemize}

If \funcparam{options} specifies KRB5_GC_CACHED,
then \funcname{krb5_get_credentials} will only search the credentials cache
for a ticket.  

If \funcparam{options} specifies KRB5_GC_USER_USER, then
\funcname{krb5_get_credentials} will get credentials for a user to user
authentication.  In a user to user authentication, the secret key for
the server 
is the session key from the server's ticket-granting-ticket
(TGT).  The TGT is passed from the server to the client over the
network --- this is safe since the TGT is encrypted in a key
known only by the Kerberos server --- and the client must pass
this TGT to \funcname{krb5_get_credentials} in
\funcparam{in_creds{\ptsto}second_ticket}.  The Kerberos server will use
this TGT to construct a user to user ticket which can be verified by
the server by using the session key from its TGT.

The effective {\bf expiration date} is the minimum of the following:
\begin{itemize}
\item The expiration date as specified in
\funcparam{in_creds{\ptsto}times.endtime} 
\item The requested start time plus the maximum lifetime of the
server as specified by the server's entry in the
Kerberos database.
\item The requested start time plus the maximum lifetime of tickets
allowed in the local site, as specified by the KDC.
This is currently a compile-time option,
KRB5_KDB_MAX_LIFE in config.h, and is by default 1 day.
\end{itemize}

If any special authorization data needs to be included in the ticket,
--- for example, restrictions on how the ticket can be used --- 
they should be specified in \funcparam{in_creds{\ptsto}authdata}.   If there
is no special authorization data to be passed,
\funcparam{in_creds{\ptsto}authdata} should be NULL.

Any returned ticket and intermediate ticket-granting tickets are
stored in \funcparam{ccache}.

Returns errors from encryption routines, system errors.

\begin{funcdecl}{krb5_get_in_tkt}{krb5_error_code}{\funcinout}
\funcarg{krb5_context}{context}
\funcin
\funcarg{const krb5_flags}{options}
\funcarg{krb5_address * const *}{addrs}
\funcarg{const krb5_enctype *}{etypes}
\funcarg{const krb5_preauthtype *}{ptypes}
\funcfuncarg{krb5_error_code}{(*key_proc)}
        \funcarg{krb5_context}{context}
        \funcarg{const krb5_keytype}{type}
        \funcarg{krb5_data *}{salt}
        \funcarg{krb5_const_pointer}{keyseed}
        \funcarg{krb5_keyblock **}{key}
\funcendfuncarg
\funcarg{krb5_const_pointer}{keyseed}
\funcfuncarg{krb5_error_code}{(*decrypt_proc)}
        \funcarg{krb5_context}{context}
        \funcarg{const krb5_keyblock *}{key}
        \funcarg{krb5_const_pointer}{decryptarg}
        \funcarg{krb5_kdc_rep *}{dec_rep}
\funcendfuncarg
\funcarg{krb5_const_pointer}{decryptarg}
\funcinout
\funcarg{krb5_creds *}{creds}
\funcarg{krb5_ccache}{ccache}
\funcarg{krb5_kdc_rep **}{ret_as_reply}
\end{funcdecl}

This all-purpose initial ticket routine, usually called via
\funcname{krb5_get_in_tkt_with_skey} or
\funcname{krb5_get_in_tkt_with_password} or
\funcname{krb5_get_in_tkt_with_keytab}.


Attempts to get an initial ticket for \funcparam{creds{\ptsto}client}
to use server \funcparam{creds{\ptsto}server}, using the following:
the realm from \funcparam{creds{\ptsto}client}; the options in
\funcparam{options} (listed in Table \ref{KDCOptions});
and \funcparam{ptypes}, the preauthentication
method (valid preauthentication methods are listed in Table
\ref{padata-types}).
\funcname{krb5_get_in_tkt} requests encryption type
\funcparam{etypes} (valid encryption types are ETYPE_DES_CBC_CRC and
ETYPE_RAW_DES_CBC),
using \funcparam{creds{\ptsto}times.starttime},
\funcparam{creds{\ptsto}times.endtime},
\funcparam{creds{\ptsto}times.renew_till} 
as from, till, and rtime.  \funcparam{creds{\ptsto}times.renew_till} is
ignored unless the RENEWABLE option is requested.

\funcparam{key_proc} is called, with \funcparam{context}, \funcparam{keytype},
\funcparam{keyseed} and\funcparam{padata} as arguments, to fill in \funcparam{key} to be used for
decryption.  The valid key types for \funcparam{keytype} are
KEYTYPE_NULL,\footnote{See RFC section 6.3.1} and
KEYTYPE_DES.\footnote{See RFC section 6.3.4}  However, KEYTYPE_DES is
the only key type supported by MIT kerberos.
The content of  \funcparam{keyseed} 
depends on the \funcparam{key_proc} being used. %nlg - need a ref here
The \funcparam{padata} passed
to \funcparam{key_proc} is the preauthentication data returned by the
KDC as part of the reply to the initial ticket request.  It may
contain an element of type KRB5_PADATA_PW_SALT, which
\funcparam{key_proc} should use to determine what salt to use when
generating the key.  \funcparam{key_proc} should fill in
\funcparam{key} with a key for the client, or return an error code.

\funcparam{decrypt_proc} is called to perform the decryption of the
response (the encrypted part is in
\funcparam{dec_rep{\ptsto}enc_part}; the decrypted part should be
allocated and filled into
\funcparam{dec_rep{\ptsto}enc_part2}.
\funcparam{decryptarg} is passed on to \funcparam{decrypt_proc}, and
its content depends on the \funcparam{decrypt_proc} being used.

If \funcparam{addrs} is non-NULL, it is used for the addresses
requested.  If it is null, the system standard addresses are used.

If \funcparam{ret_as_reply} is non-NULL, it is filled in with a
pointer to a structure containing the reply packet from the KDC.  Some
programs may find it useful to have direct access to this information.
For example, it can be used to obtain the pre-authentication data
passed back from the KDC.  The caller is responsible for freeing this
structure by using \funcname{krb5_free_kdc_rep}.

If \funcparam{etypes} is non-NULL, the it is used as for the list of
valid encyrption types. Otherwise, the context default is used (as
returned by \funcname{krb5_get_default_in_tkt_etypes}.

A succesful call will place the ticket in the credentials cache
\funcparam{ccache} and fill in \funcparam{creds} with the ticket
information used/returned.

Returns system errors, preauthentication errors, encryption errors.

% XXX Right now, uses creds->addresses before it's copied into from
% the reply -- it's passed to krb5_obtain_padata.  I think that's
% wrong, and it should be using either addrs or the result of
% krb5_os_localaddr instead.  If I'm wrong, then this spec has to be
% updated to document that creds->addresses is used.  On the other
% hand, if I'm right, then the bug in get_in_tkt needs to be fixed.
% See ov-cambridge PR 1525.

\begin{funcdecl}{krb5_get_in_tkt_with_password}{krb5_error_code}{\funcinout}
\funcarg{krb5_context}{context}
\funcin
\funcarg{const krb5_flags}{options}
\funcarg{krb5_address * const *}{addrs}
\funcarg{const krb5_enctype *}{etypes}
\funcarg{const krb5_preauthtype *}{pre_auth_types}
\funcarg{const char *}{password}
\funcarg{krb5_ccache}{ccache}
\funcinout
\funcarg{krb5_creds *}{creds}
\funcarg{krb5_kdc_rep **}{ret_as_reply}
\end{funcdecl}

Attempts to get an initial ticket using the null-terminated string
\funcparam{password}.  If \funcparam{password} is NULL, the password
is read from the terminal using as a prompt the globalname
\globalname{krb5_default_pwd_prompt1}.  

The password is converted into a key using the appropriate
string-to-key conversion function for the specified
\funcparam{keytype}, and using any salt data returned by the KDC in
response to the authentication request.

See \funcname{krb5_get_in_tkt} for documentation of the
\funcparam{options}, \funcparam{addrs}, \funcparam{pre_auth_type},
\funcparam{etype}, \funcparam{keytype}, \funcparam{ccache},
\funcparam{creds} and \funcparam{ret_as_reply} arguments.

Returns system errors, preauthentication errors, encryption errors.

\begin{funcdecl}{krb5_get_in_tkt_with_keytab}{krb5_error_code}{\funcinout}
\funcarg{krb5_context}{context}
\funcin
\funcarg{const krb5_flags}{options}
\funcarg{krb5_address * const *}{addrs}
\funcarg{const krb5_enctype *}{etypes}
\funcarg{const krb5_preauthtype *}{pre_auth_types}
\funcarg{const krb5_keytab *}{keytab}
\funcarg{krb5_ccache}{ccache}
\funcinout
\funcarg{krb5_creds *}{creds}
\funcarg{krb5_kdc_rep **}{ret_as_reply}
\end{funcdecl}

Attempts to get an initial ticket using \funcparam{keytab}.  If
\funcparam{keytab} is NULL, the default keytab is used 
(e.g., \filename{/etc/v5srvtab}).

See \funcname{krb5_get_in_tkt} for documentation of the
\funcparam{options}, \funcparam{addrs}, \funcparam{pre_auth_type},
\funcparam{etype}, \funcparam{ccache}, \funcparam{creds} and
\funcparam{ret_as_reply} arguments.

Returns system errors, preauthentication errors, encryption errors.

\begin{funcdecl}{krb5_get_in_tkt_with_skey}{krb5_error_code}{\funcinout}
\funcarg{krb5_context}{context}
\funcin
\funcarg{const krb5_flags}{options}
\funcarg{krb5_address * const *}{addrs}
\funcarg{const krb5_enctype *}{etypes}
\funcarg{const krb5_preauthtype *}{pre_auth_types}
\funcarg{const krb5_keyblock *}{key}
\funcarg{krb5_ccache}{ccache}
\funcinout
\funcarg{krb5_creds *}{creds}
\funcarg{krb5_kdc_rep **}{ret_as_reply}
\end{funcdecl}

Attempts to get an initial ticket using \funcparam{key}.  If
\funcparam{key} is NULL, an appropriate key is retrieved from the
system key store (e.g., \filename{/etc/v5srvtab}).

See \funcname{krb5_get_in_tkt} for documentation of the
\funcparam{options}, \funcparam{addrs}, \funcparam{pre_auth_type},
\funcparam{etype}, \funcparam{ccache}, \funcparam{creds} and
\funcparam{ret_as_reply} arguments.

Returns system errors, preauthentication errors, encryption errors.

\begin{funcdecl}{krb5_mk_req}{krb5_error_code}{\funcinout}
\funcarg{krb5_context}{context}
\funcarg{krb5_auth_context *}{auth_context}
\funcin
\funcarg{const krb5_flags}{ap_req_options}
\funcarg{char *}{service}
\funcarg{char *}{hostname}
\funcarg{krb5_data *}{in_data}
\funcinout
\funcarg{krb5_ccache}{ccache}
\funcout
\funcarg{krb5_data *}{outbuf}
\end{funcdecl}

Formats a KRB_AP_REQ message into \funcparam{outbuf}.

The server to receive the message is specified by
\funcparam{hostname}. The principal of the server to receive the message
is specified by \funcparam{hostname} and \funcparam{service}.
If credentials are not present in the credentials cache
\funcparam{ccache} for this server, the TGS request with default
parameters is used in an attempt to obtain such credentials, and they
are stored in \funcparam{ccache}.

\funcparam{ap_req_options} specifies the KRB_AP_REQ options desired.
Valid options are:

\begin{tabular}{ll}
AP_OPTS_USE_SESSION_KEY&\\
AP_OPTS_MUTUAL_REQUIRED&\\
\end{tabular}
\label{ap-req-options}

The checksum method to be used is as specified in \funcparam{auth_context}. 

% XXX Not sure if it's legal in the protocol for no checksum to be
% included, or if so, how the server reacts to a request with no
% checksum.

\funcparam{outbuf} should point to an existing \datatype{krb5_data}
structure.  \funcparam{outbuf{\ptsto}length} and
\funcparam{outbuf{\ptsto}data} will be filled in on success, and the latter
should be freed by the caller when it is no longer needed; if an error
is returned, however, no storage is allocated and {\tt
outbuf{\ptsto}data} does not need to be freed.

Returns system errors, error getting credentials for
\funcparam{server}.

\begin{funcdecl}{krb5_mk_req_extended}{krb5_error_code}{\funcinout}
\funcarg{krb5_context}{context}
\funcarg{krb5_auth_context *}{auth_context}
\funcin
\funcarg{const krb5_flags}{ap_req_options}
\funcarg{krb5_data *}{in_data}
\funcarg{krb5_creds *}{in_creds}
\funcout
\funcarg{krb5_data *}{outbuf}
\end{funcdecl}

Formats a KRB_AP_REQ message into \funcparam{outbuf}, with more complete
options than \funcname{krb5_mk_req}.

\funcparam{outbuf}, \funcparam{ap_req_options}, \funcparam{auth_context},
and \funcparam{ccache} are used in the same fashion as for
\funcname{krb5_mk_req}.

\funcparam{in_creds} is used to supply the credentials (ticket and session
key) needed to form the request.

If \funcparam{in_creds{\ptsto}ticket} has no data (length == 0), then an
error is returned. 

During this call, the structure elements in \funcparam{in_creds} may be
freed and reallocated.  Hence all of the structure elements which are
pointers should point to allocated memory, and there should be no other
pointers aliased to the same memory, since it may be deallocated during
this procedure call.


If \funcparam{ap_req_options} specifies AP_OPTS_USE_SUBKEY, then a
subkey will be generated if need be by \funcname{krb5_generate_subkey}.

A copy of the authenticator will be stored in the
\funcparam{auth_context}, with the principal and checksum fields nulled
out, unless an error is returned.  (This is to prevent pointer sharing
problems; the caller shouldn't need these fields anyway, since the
caller supplied them.)

Returns system errors, errors contacting the KDC, KDC errors getting
a new ticket for the authenticator.

\begin{funcdecl}{krb5_generate_subkey}{krb5_error_code}{\funcinout}
\funcarg{krb5_context}{context}
\funcin
\funcarg{const krb5_keyblock *}{key}
\funcout
\funcarg{krb5_keyblock **}{subkey}
\end{funcdecl}

Generates a pseudo-random sub-session key using the encryption system's
random key functions, based on the input \funcparam{key}.

\funcparam{subkey} is filled in to point to the generated subkey, unless
an error is returned.  The returned key (i.e., \funcparam{*subkey}) is
allocated and should be freed by the caller with
\funcname{krb5_free_keyblock} when it is no longer needed.

\begin{funcdecl}{krb5_rd_req}{krb5_error_code}{\funcinout}
\funcarg{krb5_context}{context}
\funcarg{krb5_auth_context *}{auth_context}
\funcin
\funcarg{const krb5_data *}{inbuf}
\funcarg{krb5_const_principal}{server}
\funcarg{krb5_keytab}{keytab}
\funcinout
\funcarg{krb5_flags *}{ap_req_options}
\funcout
\funcarg{krb5_ticket **}{ticket}
\end{funcdecl}

Parses a KRB_AP_REQ message, returning its contents.  Upon successful
return, if \funcparam{ticket} is non-NULL, \funcparam{*ticket} will be
modified to point to allocated storage containing the ticket
information.  The caller is responsible for deallocating this space by
using \funcname{krb5_free_ticket}.

\funcparam{inbuf} should contain the KRB_AP_REQ message to be parsed.

If \funcparam{auth_context} is NULL, one will be generated and freed
internally by the function.

\funcparam{server} specifies the expected server's name for the ticket.
If \funcparam{server} is NULL, then any server name will be accepted if
the appropriate key can be found, and the caller should verify that the
server principal matches some trust criterion.

If \funcparam{server} is not NULL, and a replay detaction cache has not been
established with the \funcparam{auth_context}, one will be generated. 

\funcparam{keytab} specifies a keytab containing generate a decryption key. If
NULL, \funcparam{krb5_kt_default} will be used to find the default 
keytab and the key taken from there\footnote{i.e., srvtab file in
Kerberos V4 parlance}.


If a keyblock is present in the \funcparam{auth_context}, it will be
used to decrypt the ticket request and the keyblock freed with
\funcname{krb5_free_keyblock}. This is useful for user to user
authentication. If no keyblock is specified, the \funcparam{keytab} is
consulted for an entry matching the requested keytype, server and
version number and used instead.

The authentcator in the request is decrypted and stored in the
\funcparam{auth_context}. The client specified in the decrypted
authenticator is compared to the client specified in the decoded ticket
to ensure that the compare.

If the remote_addr portion of the \funcparam{auth_context} is set, 
then this routine checks if the request came from the right client.

\funcparam{sender_addr} specifies the address(es) expected to be present
in the ticket.

The replay cache is checked to see if the ticket and authenticator have
been seen and if so, returns an error. If not, the ticket and
authenticator are entered into the cache.

Various other checks are made of the decoded data, including,
cross-realm policy, clockskew and ticket validation times.

The keyblock, subkey, and sequence number of the request are all stored
in the \funcparam{auth_context} for future use.

If the request has the AP_OPTS_MUTUAL_REQUIRED bit set, the local
sequence number, which is stored in the auth_context, is XORed with the
remote sequence number in the request.

If \funcparam{ap_req_options} is non-NULL, it will be set to contain the
application request flags.

Returns system errors, encryption errors, replay errors.

\begin{funcdecl}{krb5_rd_req_decoded}{krb5_error_code}{\funcinout}
\funcarg{krb5_context}{context}
\funcarg{krb5_auth_context *}{auth_context}
\funcin
\funcarg{const krb5_ap_req *}{req}
\funcarg{krb5_const_principal}{server}
\funcinout
\funcarg{krb5_keytab}{keytab}
\funcout
\funcarg{krb5_ticket **}{ticket}
\end{funcdecl}

Essentially the same as \funcname{krb5_rd_req}, but uses a decoded AP_REQ
as the input rather than an encoded input.

\begin{funcdecl}{krb5_mk_rep}{krb5_error_code}{\funcinout}
\funcarg{krb5_context}{context}
\funcarg{krb5_auth_context}{auth_context}
\funcout
\funcarg{krb5_data *}{outbuf}
\end{funcdecl}

Formats and encrypts an AP_REP message, including in it the data in the
authentp portion of \funcparam{auth_context}, encrypted using the
keyblock portion of \funcparam{auth_context}. 

When successful, \funcparam{outbuf{\ptsto}length} and
\funcparam{outbuf{\ptsto}data} are filled in with the length of the
AP_REQ message and allocated data holding it.
\funcparam{outbuf{\ptsto}data} should be freed by the
caller when it is no longer needed. 

If the flags in \funcparam{auth_context} indicate that a sequence number
should be used (either {\sc KRB5_AUTH_CONTEXT_DO_SEQUENCE} or
{\sc KRB5_AUTH_CONTEXT_RET_SEQUENCE}) and the local sequqnce number in the
\funcparam{auth_context} is 0, a new number will be generated with
\funcname{krb5_generate_seq_number}.

Returns system errors.

\begin{funcdecl}{krb5_rd_rep}{krb5_error_code}{\funcinout}
\funcarg{krb5_context}{context}
\funcarg{krb5_auth_context}{auth_context}
\funcin
\funcarg{const krb5_data *}{inbuf}
\funcout
\funcarg{krb5_ap_rep_enc_part **}{repl}
\end{funcdecl}

Parses and decrypts an AP_REP message from \funcparam{*inbuf}, filling in
\funcparam{*repl} with a pointer to  allocated storage containing the
values from the message.  The caller is responsible for freeing this
structure with \funcname{krb5_free_ap_rep_enc_part}.

The keyblock stored in \funcparam{auth_context} is used to decrypt the
message after establishing any key pre-processing with
\funcname{krb5_process_key}. 

Returns system errors, encryption errors, replay errors.

\begin{funcdecl}{krb5_mk_error}{krb5_error_code}{\funcinout}
\funcarg{krb5_context}{context}
\funcin
\funcarg{const krb5_error *}{dec_err}
\funcout
\funcarg{krb5_data *}{enc_err}
\end{funcdecl}

Formats the error structure \funcparam{*dec_err} into an error buffer
\funcparam{*enc_err}.

The error buffer storage (\funcparam{enc_err{\ptsto}data}) is
allocated, and should be freed by the caller when finished.

Returns system errors.

\begin{funcdecl}{krb5_rd_error}{krb5_error_code}{\funcinout}
\funcarg{krb5_context}{context}
\funcin
\funcarg{const krb5_data *}{enc_errbuf}
\funcout
\funcarg{krb5_error **}{dec_error}
\end{funcdecl}

Parses an error protocol message from \funcparam{enc_errbuf} and fills in 
\funcparam{*dec_error} with a pointer to allocated storage containing
the error message.  The caller is reponsible for freeing this structure by
using \funcname{krb5_free_error}.

Returns system errors.

\begin{funcdecl}{krb5_generate_seq_number}{krb5_error_code}{\funcinout}
\funcarg{krb5_context}{context}
\funcin
\funcarg{const krb5_keyblock *}{key}
\funcout
\funcarg{krb5_int32 *}{seqno}
\end{funcdecl}

Generates a pseudo-random sequence number suitable for use as an initial
sequence number for the KRB_SAFE and KRB_PRIV message processing
routines.

\funcparam{key} parameterizes the choice of the random sequence number,
which is filled into \funcparam{*seqno} upon return.

\begin{funcdecl}{krb5_sendauth}{krb5_error_code}
\funcinout
\funcarg{krb5_context}{context}
\funcarg{krb5_auth_context *}{auth_context}
\funcin
\funcarg{krb5_pointer}{fd}
\funcarg{char *}{appl_version}
\funcarg{krb5_principal}{client}
\funcarg{krb5_principal}{server}
\funcarg{krb5_flags}{ap_req_options}
\funcarg{krb5_data *}{in_data}
\funcarg{krb5_creds *}{in_creds}
\funcinout
\funcarg{krb5_ccache}{ccache}
\funcout
\funcarg{krb5_error **}{error}
\funcarg{krb5_ap_rep_enc_part **}{rep_result}
\funcarg{krb5_creds **}{out_creds}
\end{funcdecl}

\funcname{krb5_sendauth} provides a convenient means for client and
server programs to send authenticated messages to one another through
network connections.  \funcname{krb5_sendauth} sends an authenticated
ticket from the client program to the server program using the network
connection specified by \funcparam{fd}.  In the MIT Unix implementation,
\funcparam{fd} should be a pointer to a file descriptor describing the
network socket.  This can be changed in other implementations, however,
if the routines \funcname{krb5_read_message},
\funcname{krb5_write_message}, \funcname{krb5_net_read}, and
\funcname{krb5_net_write} are changed.

The paramter \funcparam{appl_version} is a string describing the
application protocol version which the client is expecting to use for
this exchange.  If the server is using a different application protocol,
an error will be returned.

The parameters \funcparam{client} and \funcparam{server} specify the
kerberos principals for the client and the server.  They are
ignored if \funcparam{in_creds} is non-null.  Otherwise,
\funcparam{server} must be non-null, but \funcparam{client} may be
null, in which case the client principal used is the one in the
credential cache's default principal.

The \funcparam{ap_req_options} parameters specifies the options which
should be passed to \funcname{krb5_mk_req}.  Valid options are listed
in Table \ref{ap-req-options}.  If \funcparam{ap_req_options}
specifies MUTUAL_REQUIRED, then \funcname{krb5_sendauth} will perform
a mutual authentication exchange, and if \funcparam{rep_result} is
non-null, it will be filled in with the result of the mutual
authentication exchange; the caller should free
\funcparam{*rep_result} with
\funcname{krb5_free_ap_rep_enc_part} when done with it.

If \funcparam{in_creds} is non-null, then
\funcparam{in_creds{\ptsto}client} and 
\funcparam{in_creds{\ptsto}server} must be filled in, and either
the other structure fields should be filled in with valid credentials,
or \funcparam{in_creds{\ptsto}ticket.length} should be zero.  If
\funcparam{in_creds{\ptsto}ticket.length} is non-zero, then
\funcparam{in_creds} will be used as-is as the credentials to send to
the server, and \funcparam{ccache} is ignored; otherwise,
\funcparam{ccache} is used as described below, and \funcparam{out_creds}
, if not NULL, is filled in with the retrieved credentials.

\funcname{ccache} specifies the credential cache to use when one is
needed (i.e., when \funcname{in_creds} is null or
\funcparam{in_creds{\ptsto}ticket.length} is zero).  When a credential
cache is not needed, \funcname{ccache} is ignored.  When a credential
cache is needed and \funcname{ccache} is null, the default credential
cache is used.  Note that if the credential cache is needed and does
not contain the needed credentials, they will be retrieved from the
KDC and stored in the credential cache.

If mutual authentication is used and \funcparam{rep_result} is non-null,
the sequence number for the server is available to the caller in
\funcparam{*rep_result->seq_number}.  (If mutual authentication is not
used, there is no way to negotiate a sequence number for the server.)

If an error occurs during the authenticated ticket exchange and
\funcparam{error} is non-null, the error packet (if any) that was sent
from the server will be placed in it.  This error should be freed with
\funcname{krb5_free_error}.

\begin{funcdecl}{krb5_recvauth}{krb5_error_code}{\funcinout}
\funcarg{krb5_context}{context}
\funcarg{krb5_auth_context *}{auth_context}
\funcin
\funcarg{krb5_pointer}{fd}
\funcarg{char *}{appl_version}
\funcarg{krb5_principal}{server}
\funcarg{char *}{rc_type}
\funcarg{krb5_int32}{flags}
\funcarg{krb5_keytab}{keytab}
\funcout
\funcarg{krb5_ticket **}{ticket}
\end{funcdecl}

\funcname{krb5_recvauth} provides a convenient means for client and
server programs to send authenticated messages to one another through
network connections.  \funcname{krb5_sendauth} is the matching routine
to \funcname{krb5_recvauth} for the server.  \funcname{krb5_recvauth}
will engage in an authentication dialogue with the
client program running \funcname{krb5_sendauth} to authenticate the
client to the server.  In addition, if requested by the client,
\funcname{krb5_recvauth} will provide mutual authentication to
prove to the client that the server represented by
\funcname{krb5_recvauth} is legitimate. 

\funcparam{fd} is a pointer to the network connection.  As in
\funcname{krb5_sendauth}, in the MIT Unix implementation
\funcparam{fd} is a pointer to a file descriptor.

The parameter \funcparam{appl_version} is a string describing the
application protocol version which the server is expecting to use for
this exchange.  If the client is using a different application protocol,
an error will be returned and the authentication exchange will be
aborted.

If \funcparam{server} is non-null, then \funcname{krb5_recvauth}
verifies that the server principal requested by the client matches
\funcparam{server}.  If not, an error will be returned and the
authentication exchange will be aborted.

The parameters \funcparam{server}, \funcparam{auth_context},
and \funcparam{keytab} are used by \funcname{krb5_rd_req} to obtain the
server's private key.

If \funcparam{server} is non-null, the princicpal component of it is
ysed to determine the replay cache to use. Otherwise,
\funcname{krb5_recvauth} will use a default replay cache.

The \funcparam{flags} argument allows the caller to modify the behavior of
\funcname{krb5_recvauth}.  For non-library callers, \funcparam{flags}
should be 0. 

% XXX Note that if the bug I submitted entitled ``"flags" argument
% should not have been added to krb5_recvauth'' (OpenVision Cambridge
% bug number 1585) causes code changes, this will have to be updated.

\funcparam{ticket} is optional and is only filled in if non-null. It is
filled with the data from the ticket sent by the client, and should be
freed with
\funcname{krb5_free_ticket} when it is no longer needed.

\begin{funcdecl}{krb5_mk_safe}{krb5_error_code}{\funcinout}
\funcarg{krb5_context}{context}
\funcarg{krb5_auth_context}{auth_context}
\funcin
\funcarg{const krb5_data *}{userdata}
\funcout
\funcarg{krb5_data *}{outbuf}
\funcinout
\funcarg{krb5_replay_data *}{outdata}
\end{funcdecl}

Formats a KRB_SAFE message into \funcparam{outbuf}.

\funcparam{userdata} is formatted as the user data in the message.
Portions of \funcparam{auth_context} specify the checksum type; the
keyblockm which might be used to seed the checksum;
full addresses (host and port) for the sender and receiver.
The \funcparam{local_addr} portion of \funcparam{*auth_context} 
is used to form the addresses usedin the KRB_SAFE message. The  \funcparam{remote_addr} is optional; if the
receiver's address is not known, it may be replaced by NULL.
\funcparam{local_addr}, however, is mandatory.

The \funcparam{auth_context} flags select whether sequence numbers or
timestamps should be used to identify the message.  Valid flags are
listed below.

\begin{tabular}{ll}
\multicolumn{1}{c}{Symbol} & Meaning \\
KRB5_AUTH_CONTEXT_DO_TIME               & Use timestamps\\
        &\  and replay cache\\
KRB5_AUTH_CONTEXT_RET_TIME      & Copy timestamp \\
        &\ to \funcparam{*outdata} \\
KRB5_AUTH_CONTEXT_DO_SEQUENCE   & Use sequence numbers \\
KRB5_AUTH_CONTEXT_RET_SEQUENCE  & Copy sequence numbers\\
        &\ to \funcparam{*outdata} \\
\end{tabular}

If timestamps are to be used (i.e., if KRB5_AUTH_CONTEXT_DO_TIME is
set), an entry describing the message will be entered in the replay
cache so that the caller may detect if this message is sent
back to him by an attacker.  If KRB5_AUTH_CONTEXT_DO_TIME is not set,
the \funcparam{auth_context} replay cache is not used.

If sequence numbers are to be used (i.e., if either
KRB5_AUTH_CONTEXT_DO_SEQUENCE or KRB5_AUTH_CONTEXT_RET_SEQUENCE is
set), then \funcparam{auth_context} local sequence number will be
placed in the protected message as its sequence number.  

The \funcparam{outbuf} buffer storage (i.e.,
\funcparam{outbuf{\ptsto}data}) is allocated, and should be freed by
the caller when finished.

Returns system errors, encryption errors.

\begin{funcdecl}{krb5_rd_safe}{krb5_error_code}{\funcinout}
\funcarg{krb5_context}{context}
\funcarg{krb5_auth_context}{auth_context}
\funcin
\funcarg{const krb5_data *}{inbuf}
\funcout
\funcarg{krb5_data *}{outbuf}
\funcinout
\funcarg{krb5_replay_data *}{outdata}
\end{funcdecl}

Parses a KRB_SAFE message from \funcparam{inbuf}, placing the
data in \funcparam{*outbuf} after verifying its integrity.

The keyblock used for verifying the integrity of the message is taken
from the \funcparam{auth_context} recv\_subkey or keyblock. The
keyblock is chosen in the above order by the first one which is not
NULL.
 
The remote_addr and localaddr portions of the \funcparam{*auth_context}
specify the full addresses (host and port) of the sender and receiver,
and must be of type \datatype{ADDRTYPE_ADDRPORT}.


The \funcparam{remote_addr} parameter is mandatory; it
specifies the address of the sender.  If the address of the sender in
the message does not match \funcparam{remote_addr}, the error
KRB5KRB_AP_ERR_BADADDR will be returned.

If \funcparam{local_addr} is non-NULL, then the address of the receiver
in the message much match it.  If it is null, the receiver address in
the message will be checked against the list of local addresses as
returned by \funcname{krb5_os_localaddr}. If the check fails,
KRB5KRB_AP_ERR_BADARRD is returned.

The \funcparam{outbuf} buffer storage (i.e.,
\funcparam{outbuf{\ptsto}data} is allocated storage which the caller
should free when it is no longer needed.

If auth_context_flags portion of \funcparam{auth_context} indicates
that sequence numbers are to be used (i.e., if KRB5_AUTH_CONTEXT_DOSEQUENCE is
set in it), The \funcparam{remote_seq_number} portion of
\funcparam{auth_context} is compared to the sequence number for the
message, and KRB5_KRB_AP_ERR_BADORDER is returned if it does not match.
Otherwise, the sequence number is not used.


If timestamps are to be used (i.e., if KRB5_AUTH_CONTEXT_DO_TIME is set
in the \funcparam{auth_context}), then two additional checks are performed:
\begin{itemize}
\item The timestamp in the message must be within the permitted clock
        skew (which is usually five minutes), or KRB5KRB_AP_ERR_SKEW
        is returned.
\item The message must not be a replayed message, according to
        \funcparam{rcache}.
\end{itemize}

Returns system errors, integrity errors.

\begin{funcdecl}{krb5_mk_priv}{krb5_error_code}{\funcinout}
\funcarg{krb5_context}{context}
\funcarg{krb5_auth_context}{auth_context}
\funcin
\funcarg{const krb5_data *}{userdata}
\funcout
\funcarg{krb5_data *}{outbuf}
\funcarg{krb5_replay_data *}{outdata}
\end{funcdecl}

Formats a KRB_PRIV message into \funcparam{outbuf}.  Behaves similarly
to \funcname{krb5_mk_safe}, but the message is encrypted and
integrity-protected rather than just integrity-protected.

\funcparam{inbuf}, \funcparam{auth_context}, 
\funcparam{outdata} and
\funcparam{outbuf} function as in \funcname{krb5_mk_safe}.

As in \funcname{krb5_mk_safe}, the remote_addr and remote_port part of
the \funcparam{auth_context} is optional; if the receiver's address is
not known, it may be replaced by NULL.  The local_addr, however, is
mandatory.

The encryption type is taken from the \funcparam{auth_context} keyblock
portion. If i_vector portion of the \funcparam{auth_context}
is non-null, it is used as an initialization vector for the encryption
(if the chosen encryption type supports initialization vectors)
and its contents are replaced with the last block of encrypted data
upon return.

The flags from the \funcparam{auth_context} selects whether sequence numbers or timestamps
should be used to identify the message.  Valid flags are listed below.

\begin{tabular}{ll}
\multicolumn{1}{c}{Symbol} & Meaning \\
KRB5_AUTH_CONTEXT_DO_TIME& Use timestamps in replay cache\\
KRB5_AUTH_CONTEXT_RET_TIME& Use timestamps in output data\\
KRB5_AUTH_CONTEXT_DO_SEQUENCE& Use sequence numbers\\
        &\ in replay cache\\
KRB5_AUTH_CONTEXT_RET_SEQUENCE& Use sequence numbers\\
        &\ in replay cache and output data \\
\end{tabular}

Returns system errors, encryption errors.

\begin{funcdecl}{krb5_rd_priv}{krb5_error_code}{\funcinout}
\funcarg{krb5_context}{context}
\funcarg{krb5_auth_context}{auth_context}
\funcin
\funcarg{const krb5_data *}{inbuf}
\funcout
\funcarg{krb5_data *}{outbuf}
\funcarg{krb5_data *}{outdata}
\end{funcdecl}

Parses a KRB_PRIV message from \funcparam{inbuf}, placing the data in
\funcparam{*outbuf} after decrypting it.  Behaves similarly to
\funcname{krb5_rd_safe}, but the message is decrypted rather than
integrity-checked.

\funcparam{inbuf}, \funcparam{auth_context}, 
\funcparam{outdata} and \funcparam{outbuf}
function as in \funcname{krb5_rd_safe}.


The remote_addr part of the \funcparam{auth_context} as set by
\funcname{krb5_auth_con_setaddrs} is mandatory;  it
specifies the address of the sender.  If the address of the sender in
the message does not match the remote_addr, the error
KRB5KRB_AP_ERR_BADADDR will be returned.

If local_addr portion of the auth_context is non-NULL, then the address
of the receiver in the message much match it.  If it is null, the
receiver address in the message will be checked against the list of
local addresses as returned by \funcname{krb5_os_localaddr}.

The \funcparam{keyblock} portion of \funcparam{auth_context} specifies
the key to be used for decryption of the message.  If the
\funcparam{i_vector} element, is non-null, it is used as an
initialization vector for the decryption (if the encryption type of the
message supports initialization vectors) and its contents are replaced
with the last block of encrypted data in the message.

The \funcparam{auth_context} flags specify whether timestamps
(KRB5_AUTH_CONTEXT_DO_TIME) and sequence numbers
(KRB5_AUTH_CONTEXT_DO_SEQUENCE) are to be used.

Returns system errors, integrity errors.

\subsubsection{Miscellaneous main functions}

\begin{funcdecl}{krb5_address_search}{krb5_boolean}{\funcinout}
\funcarg{krb5_context}{context}
\funcin
\funcarg{const krb5_address *}{addr}
\funcarg{krb5_address * const *}{addrlist}
\end{funcdecl}

If \funcparam{addr} is listed in \funcparam{addrlist}, or
\funcparam{addrlist} is null, return TRUE.  If not listed, return FALSE.

\begin{funcdecl}{krb5_address_compare}{krb5_boolean}{\funcinout}
\funcarg{krb5_context}{context}
\funcin
\funcarg{const krb5_address *}{addr1}
\funcarg{const krb5_address *}{addr2}
\end{funcdecl}

If the two addresses are the same, return TRUE, else return FALSE.

\begin{funcdecl}{krb5_fulladdr_order}{int}{\funcinout}
\funcarg{krb5_context}{context}
\funcin
\funcarg{const krb5_fulladdr *}{addr1}
\funcarg{const krb5_fulladdr *}{addr2}
\end{funcdecl}

Return an ordering on the two full addresses:  0 if the same,
$< 0$ if first is less than 2nd, $> 0$ if first is greater than 2nd.

\begin{funcdecl}{krb5_address_order}{int}{\funcinout}
\funcarg{krb5_context}{context}
\funcin
\funcarg{const krb5_address *}{addr1}
\funcarg{const krb5_address *}{addr2}
\end{funcdecl}

Return an ordering on the two addresses:  0 if the same,
$< 0$ if first is less than 2nd, $> 0$ if first is greater than 2nd.

\begin{funcdecl}{krb5_copy_addresses}{krb5_error_code}{\funcinout}
\funcarg{krb5_context}{context}
\funcin
\funcarg{krb5_address * const *}{inaddr}
\funcout
\funcarg{krb5_address ***}{outaddr}
\end{funcdecl}

Copy addresses in \funcparam{inaddr} to \funcparam{*outaddr} which is
allocated memory and should be freed with \funcname{krb5_free_addresses}.

\begin{funcdecl}{krb5_copy_authdata}{krb5_error_code}{\funcinout}
\funcarg{krb5_context}{context}
\funcin
\funcarg{krb5_authdata * const *}{inauthdat}
\funcout
\funcarg{krb5_authdata ***}{outauthdat}
\end{funcdecl}

Copy an authdata structure, filling in \funcparam{*outauthdat} to point to the
newly allocated copy, which should be freed with 
\funcname{krb5_free_authdata}.

\begin{funcdecl}{krb5_copy_authenticator}{krb5_error_code}{\funcinout}
\funcarg{krb5_context}{context}
\funcin
\funcarg{const krb5_authenticator *}{authfrom}
\funcout
\funcarg{krb5_authenticator **}{authto}
\end{funcdecl}

Copy an authenticator structure, filling in \funcparam{*outauthdat} to
point to the newly allocated copy, which should be freed with 
\funcname{krb5_free_authenticator}.

\begin{funcdecl}{krb5_copy_keyblock}{krb5_error_code}{\funcinout}
\funcarg{krb5_context}{context}
\funcin
\funcarg{const krb5_keyblock *}{from}
\funcout
\funcarg{krb5_keyblock **}{to}
\end{funcdecl}

Copy a keyblock, filling in \funcparam{*to} to point to the newly
allocated copy, which should be freed with
\funcname{krb5_free_keyblock}. 

\begin{funcdecl}{krb5_copy_keyblock_contents}{krb5_error_code}{\funcinout}
\funcarg{krb5_context}{context}
\funcin
\funcarg{const krb5_keyblock *}{from}
\funcout
\funcarg{krb5_keyblock *}{to}
\end{funcdecl}

Copy keyblock contents from \funcparam{from} to \funcparam{to}, including
allocated storage.  The allocated storage in \funcparam{to} should be
freed by using {\bf free}(\funcparam{to->contents}).

\begin{funcdecl}{krb5_copy_checksum}{krb5_error_code}{\funcinout}
\funcarg{krb5_context}{context}
\funcin
\funcarg{const krb5_checksum *}{ckfrom}
\funcout
\funcarg{krb5_checksum **}{ckto}
\end{funcdecl}

Copy a checksum structure, filling in \funcparam{*ckto} to point to
the newly allocated copy, which should be freed with
\funcname{krb5_free_checksum}.

\begin{funcdecl}{krb5_copy_creds}{krb5_error_code}{\funcinout}
\funcarg{krb5_context}{context}
\funcin
\funcarg{const krb5_creds *}{incred}
\funcout
\funcarg{krb5_creds **}{outcred}
\end{funcdecl}

Copy a credentials structure, filling in \funcparam{*outcred} to point
to the newly allocated copy, which should be freed with
\funcname{krb5_free_creds}.

\begin{funcdecl}{krb5_copy_data}{krb5_error_code}{\funcinout}
\funcarg{krb5_context}{context}
\funcin
\funcarg{const krb5_data *}{indata}
\funcout
\funcarg{krb5_data **}{outdata}
\end{funcdecl}

Copy a data structure, filling in \funcparam{*outdata} to point to the
newly allocated copy, which should be freed with \funcname{krb5_free_data}.

\begin{funcdecl}{krb5_copy_ticket}{krb5_error_code}{\funcinout}
\funcarg{krb5_context}{context}
\funcin
\funcarg{const krb5_ticket *}{from}
\funcout
\funcarg{krb5_ticket **}{pto}
\end{funcdecl}

Copy a ticket structure, filling in \funcparam{*pto} to point to
the newly allocated copy, which should be freed with
\funcname{krb5_free_ticket}.


\begin{funcdecl}{krb5_get_server_rcache}{krb5_error_code}{\funcinout}
\funcarg{krb5_context}{context}
\funcin
\funcarg{const krb5_data *}{piece}
\funcout
\funcarg{krb5_rcache *}{ret_rcache}
\end{funcdecl}

Generate a replay cache name, allocate space for its handle, and open
it.  \funcparam{piece} is used to distinguish this replay cache from
others currently in use on the system.  Typically, \funcparam{piece}
is the first component of the principal name for the client or server
which is calling \funcname{krb5_get_server_rcache}.

Upon successful return, \funcparam{ret_rcache} is filled in to contain a
handle to an open rcache, which should be closed with
\funcname{krb5_rc_close}.


