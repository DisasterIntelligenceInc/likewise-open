\documentclass[twoside]{article}
\usepackage{fixunder,functions,fancyheadings}
\usepackage{krb5idx}
%\usepackage{hyperref}

%\hypersetup{letterpaper,
%    bookmarks=true,
%pdfpagemode=UseOutlines,
%}

%
%\setlength{\oddsidemargin}{1in}
%\setlength{\evensidemargin}{1.00in}
%\setlength{\textwidth}{6.5in}
\setlength{\oddsidemargin}{0in}
\setlength{\evensidemargin}{1.50in}
\setlength{\textwidth}{5.25in}
\setlength{\marginparsep}{0.0in}
\setlength{\marginparwidth}{1.95 in}
\setlength{\topmargin}{-.5in}
\setlength{\textheight}{9in}
\setlength{\parskip}{.1in}
\setlength{\parindent}{2em}
\setlength{\footrulewidth}{0.4pt}
\setlength{\plainfootrulewidth}{0.4pt}
\setlength{\plainheadrulewidth}{0.4pt}
\makeindex
\newif\ifdraft
\draftfalse
%
% Far, far too inconvenient... it's still very draft-like anyway....
%   [tytso:19900921.0018EDT]
%
%\typein{Draft flag? (type \noexpand\draftfalse<CR> if not draft...)}
\ifdraft
\pagestyle{fancyplain}
\addtolength{\headwidth}{\marginparsep}
\addtolength{\headwidth}{\marginparwidth}
\makeatletter
\renewcommand{\sectionmark}[1]{\markboth {\uppercase{\ifnum \c@secnumdepth >\z@
    \thesection\hskip 1em\relax \fi #1}}{}}%
\renewcommand{\subsectionmark}[1]{\markright {\ifnum \c@secnumdepth >\@ne
          \thesubsection\hskip 1em\relax \fi #1}}
\makeatother
\lhead[\thepage]{\fancyplain{}{\sl\rightmark}}
\rhead[\fancyplain{}{\sl\rightmark}]{\thepage}
\lfoot[]{{\bf DRAFT---DO NOT REDISTRIBUTE}}
\rfoot[{\bf DRAFT---DO NOT REDISTRIBUTE}]{}
\cfoot{\thepage}
\else\pagestyle{headings}\fi

\def\internalfunc{NOTE: This is an internal function, which is not
necessarily intended for use by application programs.  Its interface may
change at any time.\par}

%nlg- time to make this a real document

\title{\Huge Kerberos V5 application programming library}
\date{\ifdraft \\ {\Large DRAFT---}\fi\today}
\author{MIT Information Systems}

\begin{document}
\maketitle
\tableofcontents

%\thispagestyle{empty}
%\begin{center}
%{\Huge Kerberos V5 application programming library}
%\ifdraft \\ {\Large DRAFT---\today}\fi
%\end{center}

\section{Introduction}
	This document describes the routines that make up the Kerberos
V5 application programming interface.  It is geared towards
programmers who already have a basic familiarity with Kerberos and are
in the process of including Kerberos authentication as part of 
applications being developed.

	The function descriptions included are up to date, even if the
description of the functions may be hard to understand for the novice
Kerberos programmer.

\subsection{Acknowledgments}


The Kerberos model is based in part on Needham and Schroeder's trusted
third-party authentication protocol and on modifications suggested by
Denning and Sacco.  The original design and implementation of Kerberos
Versions 1 through 4 was the work of Steve Miller of Digital Equipment
Corporation and Clifford Neuman (now at the Information Sciences
Institute of the University of Southern California), along with Jerome
Saltzer, Technical Director of Project Athena, and Jeffrey Schiller,
MIT Campus Network Manager.  Many other members of Project Athena have
also contributed to the work on Kerberos.  Version 4 is publicly
available, and has seen wide use across the Internet.

Version 5 (described in this document) has evolved from Version 4 based
on new requirements and desires for features not available in Version 4.

%nlg- a bunch more probably needs to be added here to credit all
%those that have contributed to V5 -nlg

\subsection{Kerberos Basics}

Kerberos performs authentication as a trusted third-party
authentication service by using conventional (shared secret
key\footnote{ {\em Secret} and {\em private} are often used
interchangeably in the literature.  In our usage, it takes two (or
more) to share a secret, thus a shared DES key is a {\em secret} key.
Something is only private when no one but its owner knows it.  Thus,
in public key cryptosystems, one has a public and a {\em private} key.
}) cryptography.  Kerberos provides a means of verifying the
identities of principals, without relying on authentication by the
host operating system, without basing trust on host addresses, without
requiring physical security of all the hosts on the network, and under
the assumption that packets traveling along the network can be read,
modified, and inserted at will.

When integrating Kerberos into an application it is important to
review how and when Kerberos functions are used to ensure that the
application's design does not compromise the authentication.  For
instance, an application which uses Kerberos' functions only upon the
{\em initiation} of a stream-based network connection, and assumes the
absence of any active attackers who might be able to ``hijack'' the
stream connection.

%{\Huge nlg- It would be nice to include more examples here of common
%mistakes one can make in designing kerberized systems -nlg}

The Kerberos protocol code libraries, whose API is described in this
document, can be used to provide encryption to any application.  In
order to add authentication to its transactions, a typical network
application adds one or two calls to the Kerberos library, which
results in the transmission of the necessary messages to achieve
authentication.

The two methods for obtaining credentials, the initial ticket exchange
and the ticket granting ticket exchange, use slightly different
protocols and require different API routines.  The basic difference an
API programmer will see is that the initial request does not require a
ticket granting ticket (TGT) but does require the client's secret key
because the reply is sent back encrypted in the client's secret key.
Usually this request is for a TGT and TGT based exchanges are used
from then on.  In a TGT exchange the TGT is sent as part of the
request for tickets and the reply is encrypted in the session key from
the TGT.  For example, once a user's password is used to obtain a TGT,
it is not required for subsequent TGT exchanges.

The reply consists of a ticket and a session key, encrypted either in
the user's secret key (i.e., password), or the TGT session key.  The
combination of a ticket and a session key is known as a set of {\em
credentials}.\footnote{In Kerberos V4, the ``ticket file'' was a bit of
a misnomer, since it contained both tickets and their associated session
keys.  In Kerberos V5, the ``ticket file'' has been renamed to be the
{\em credentials cache}.} An application client can use these
credentials to authenticate to the application server by sending the
ticket and an {\em authenticator} to the server.  The authenticator is
encrypted in the session key of the ticket, and contains the name of the
client, the name of the server, the time the authenticator was created.

In order to verify the authentication, the application server decrypts
the ticket using its service key, which is only known by the application
server and the Kerberos server.  Inside the ticket, the Kerberos server
had placed the name of the client, the name of the server, a DES key
associated with this ticket, and some additional information.  The
application server then uses the ticket session key to decrypt the
authenticator, and verifies that the information in the authenticator
matches the information in the ticket, and that the timestamp in the
authenticator is recent (to prevent reply attacks).  Since the session
key was generated randomly by the Kerberos server, and delivered only
encrypted in the service key, and in a key known only by the user, the
application server can be confident that user is really who he or she
claims to be, by virtue of the fact that the user was able to encrypt
the authenticator in the correct key.

To provide detection of both replay
attacks and message stream modification attacks, the integrity of all
the messages exchanged between principals can also be 
guar\-an\-teed\footnote{Using
\funcname{krb5_mk_safe} and \funcname{krb5_rd_safe} to create and
verify KRB5_SAFE messages} by generating and transmitting a
collision-proof checksum\footnote{aka cryptographic checksum,
elsewhere this is called a hash or digest function} of the client's
message, keyed with the session key.  Privacy and integrity of the
messages exchanged between principals can be secured\footnote{Using
\funcname{krb5_mk_priv} and \funcname{krb5_rd_priv} to create and
verify KRB5_PRIV messages} by encrypting the data to be passed using
the session key.

\subsubsection{The purpose of Realms}

The Kerberos protocol is designed to operate across organizational
boundaries.   Each organization wishing to run a Kerberos
server establishes its own {\em realm}.  The name of the realm in which a
client is registered is part of the client's name, and can be used by the
end-service to decide whether to honor a request.

By establishing {\em inter-realm} keys, the administrators of two
realms can allow a client authenticated in the local realm to use its
credentials remotely.  The exchange of inter-realm keys (a separate
key may be used for each direction) registers the ticket-granting
service of each realm as a principal in the other realm.  A client is
then able to obtain a ticket-granting ticket for the remote realm's
ticket-granting service from its local realm.  When that
ticket-granting ticket is used, the remote ticket-granting service
uses the inter-realm key (which usually differs from its own normal
TGS key) to decrypt the ticket-granting ticket, and is thus certain
that it was issued by the client's own TGS. Tickets issued by the
remote ticket-granting service will indicate to the end-service that
the client was authenticated from another realm.   


This method can be repeated to authenticate throughout an organization
across multiple realms.  To build a valid authentication
path\footnote{An {\em authentication path} is the sequence of
intermediate realms that are transited in communicating from one realm
to another.} to a distant realm, the local realm must share an
inter-realm key with an intermediate realm which
communicates\footnote{A realm is said to {\em communicate} with
another realm if the two realms share an inter-realm key} with either
the distant remote realm or yet another intermediate realm.

Realms are typically organized hierarchically.  Each realm shares a
key with its parent and a different key with each child.  If an
inter-realm key is not directly shared by two realms, the hierarchical
organization allows an authentication path to be easily constructed.
If a hierarchical organization is not used, it may be necessary to
consult some database in order to construct an authentication path
between realms.

Although realms are typically hierarchical, intermediate realms may be
bypassed to achieve cross-realm authentication through alternate
authentication paths\footnote{These might be established to make communication
between two realms more efficient}.  It is important for the
end-service to know which realms were transited when deciding how much
faith to place in the authentication process.  To facilitate this
decision, a field in each ticket contains the names of the realms that
were involved in authenticating the client.

\subsubsection{Fundamental assumptions about the environment}

Kerberos has certain limitations that should be kept in mind when
designing security measures:

\begin{itemize}
\item
Kerberos does not address ``Denial of service'' attacks.  There are
places in these protocols where an intruder can prevent an application
from participating in the proper authentication steps.  Detection and
solution of such attacks (some of which can appear to be not-uncommon
``normal'' failure modes for the system) is usually best left to
the human administrators and users.

\item
Principals must keep their secret keys secret.  If an intruder somehow
steals a principal's key, it will be able to masquerade as that
principal or impersonate any server to the legitimate principal.

\item
``Password guessing'' attacks are not solved by Kerberos.  If a user
chooses a poor password, it is possible for an attacker to
successfully mount an offline dictionary attack by repeatedly
attempting to decrypt, with successive entries from a dictionary,
messages obtained which are encrypted under a key derived from the
user's password.

\end{itemize}

\subsection{Glossary of terms}

Below is a list of terms used throughout this document.

\begin{description}
\item [Authentication] 
Verifying the claimed identity of a principal.

\item [Authentication header]
A record containing a Ticket and an Authenticator to be presented to a
server as part of the authentication process.

\item [Authentication path]
A sequence of intermediate realms transited in the authentication
process when communicating from one realm to another.

\item [Authenticator]
A record containing information that can be shown to
have been recently generated using the session key known only by the 
client and server.

\item [Authorization]
The process of determining whether a client may use a
service,  which objects the client is allowed to access, and the 
type of access allowed for each.

\item [Ciphertext]
The output of an encryption function.  Encryption transforms plaintext
into ciphertext.

\item [Client]
A process that makes use of a network service on behalf of a
user.  Note that in some cases a {\em Server} may itself be a client of
some other server (e.g. a print server may be a client of a file server).

\item [Credentials]
A ticket plus the secret session key necessary to
successfully use that ticket in an authentication exchange.

\item [KDC]
Key Distribution Center, a network service that supplies
tickets and temporary session keys; or an
instance of that service or the host on which it runs.
The KDC services both initial ticket and ticket-granting ticket
requests.
The initial ticket portion is sometimes referred to as the
Authentication Server (or service).
The ticket-granting ticket portion is sometimes referred to as the
ticket-granting server (or service).

\item [Kerberos]
Aside from the 3-headed dog guarding Hades, the name given
to Project Athena's authentication service, the protocol used by that
service, or the code used to implement the authentication service.

\item [Plaintext]
The input to an encryption function or the output of a decryption
function.  Decryption transforms ciphertext into plaintext.

\item [Principal]
A uniquely named client or server instance that participates in
a network communication.

\item [Principal identifier]
The name used to uniquely identify each different
principal.

\item [Seal]
To encipher a record containing several fields in such a way
that the fields cannot be individually replaced without either
knowledge of the encryption key or leaving evidence of tampering.

\item [Secret key]
An encryption key shared by a principal and the KDC,
distributed outside the bounds of the system, with a long lifetime.
In the case of a human user's principal, the secret key is derived from a
password.

\item [Server]
A particular Principal which provides a resource to network clients.

\item [Service]
A resource provided to network clients; often provided by more than one
server (for example, remote file service).

\item [Session key]
A temporary encryption key used between two principals,
with a lifetime limited to the duration of a single login
{\em session}.

\item [Sub-session key] 
A temporary encryption key used between two
principals, selected and exchanged by the principals using the session
key, and with a lifetime limited to the duration of a single
association.

\item [Ticket]
A record that helps a client authenticate itself to a server; it contains
the client's identity, a session key, a timestamp, and other
information, all sealed using the server's secret key.  It only serves to
authenticate a client when presented along with a fresh Authenticator.

\end{description}


\section{Useful KDC parameters to know about}
The following is a list of options which can be passed to the Kerberos
server (also known as the Key Distribution Center or KDC).  These
options affect what sort of tickets the KDC will return to the
application program.  The KDC options can be passed to
\funcname{krb5_get_in_tkt}, \funcname{krb5_get_in_tkt_with_password},
\funcname{krb5_get_in_tkt_with_skey}, and \funcname{krb5_send_tgs}. 


\begin{center}
\begin{tabular}{llc}
\multicolumn{1}{c}{Symbol}&\multicolumn{1}{c}{RFC}& Valid for \\
&\multicolumn{1}{c}{section}&get_in_tkt? \\ \hline
KDC_OPT_FORWARDABLE	& 2.6	& yes		\\
KDC_OPT_FORWARDED	& 2.6	&		\\
KDC_OPT_PROXIABLE	& 2.5	& yes		\\
KDC_OPT_PROXY		& 2.5	&		\\
KDC_OPT_ALLOW_POSTDATE	& 2.4	& yes		\\
KDC_OPT_POSTDATED	& 2.4	& yes		\\
KDC_OPT_RENEWABLE	& 2.3	& yes		\\
KDC_OPT_RENEWABLE_OK	& 2.7	& yes		\\
KDC_OPT_ENC_TKT_IN_SKEY	& 2.7	&		\\
KDC_OPT_RENEW		& 2.3	&		\\
KDC_OPT_VALIDATE	& 2.2	&		\\
\end{tabular}
\end{center}
\label{KDCOptions}

The following is a list of preauthentication methods which are supported
by Kerberos.  Most preauthentication methods are used by
\funcname{krb5_get_in_tkt}, \funcname{krb5_get_in_tkt_with_password}, and
\funcname{krb5_get_in_tkt_with_skey}; at some sites, the Kerberos server can be
configured so that during the initial ticket transation, it will only
return encrypted tickets after the user has proven his or her identity
using a supported preauthentication mechanism.  This is done to make
certain password guessing attacks more difficult to carry out.



\begin{center}
\begin{tabular}{lcc}
\multicolumn{1}{c}{Symbol}&In & Valid for \\
&RFC?&get_in_tkt? \\ \hline
KRB5_PADATA_NONE		& yes	& yes	\\
KRB5_PADATA_AP_REQ		& yes	&	\\
KRB5_PADATA_TGS_REQ		& yes	&	\\
KRB5_PADATA_PW_SALT		& yes	&	\\
KRB5_PADATA_ENC_TIMESTAMP	& yes	& yes	\\
KRB5_PADATA_ENC_SECURID		&	& yes	\\
\end{tabular}
\end{center}
\label{padata-types}

KRB5_PADATA_TGS_REQ is rarely used by a programmer; it is used to pass
the ticket granting ticket to the Ticket Granting Service (TGS) during a
TGS transaction (as opposed to an initial ticket transaction).

KRB5_PW_SALT is not really a preauthentication method at all.  It is
passed back from the Kerberos server to application program, and it
contains a hint to the proper password salting algorithm which should be
used during the initial ticket exchange.

%The encription type can also be specified in
%\funcname{krb5_get_in_tkt}, however normally only one keytype is used
%in any one database.
%
%\begin{center}
%\begin{tabular}{llc}
%\multicolumn{1}{c}{Symbol}&\multicolumn{1}{c}{RFC}& Supported? \\
%& \multicolumn{1}{c}{section} &  \\ \hline
%ETYPE_NULL		& 6.3.1	& 	\\
%ETYPE_DES_CBC_CRC	& 6.3.2	& yes	\\
%ETYPE_DES_CBC_MD4	& 6.3.3	&	\\
%ETYPE_DES_CBC_MD5	& 6.3.4	&	\\
%ETYPE_RAW_DES_CBC	&	& yes	\\
%\end{tabular}
%\end{center}
%\label{etypes}




\section{Error tables}
\subsection{error_table krb5}

% $Source$
% $Author: epeisach $

The Kerberos v5 library error code table follows.
Protocol error codes are ERROR_TABLE_BASE_krb5 + the protocol error
code number.  Other error codes start at ERROR_TABLE_BASE_krb5 + 128.

\begin{small}
\begin{tabular}{ll}
{\sc krb5kdc_err_none }&	 No error \\
{\sc krb5kdc_err_name_exp }& Client's entry in database has expired \\
{\sc krb5kdc_err_service_exp }& Server's entry in database has expired \\
{\sc krb5kdc_err_bad_pvno }& Requested protocol version not supported \\	
{\sc krb5kdc_err_c_old_mast_kvno }& \parbox[t]{2in}{Client's key is encrypted in an old master key} \\
{\sc krb5kdc_err_s_old_mast_kvno }& \parbox[t]{2in}{Server's key is encrypted in an old master key} \\
{\sc krb5kdc_err_c_principal_unknown }&  Client not found in Kerberos database \\
{\sc krb5kdc_err_s_principal_unknown }&  Server not found in Kerberos database \\
{\sc krb5kdc_err_principal_not_unique }&\parbox[t]{2in}{\raggedright{Principal has multiple entries in Kerberos database}} \\
{\sc krb5kdc_err_null_key }& Client or server has a null key \\
{\sc krb5kdc_err_cannot_postdate }& Ticket is ineligible for postdating \\
{\sc krb5kdc_err_never_valid }& \parbox[t]{2in}{Requested effective lifetime is negative or too short} \\
{\sc krb5kdc_err_policy }&	 KDC policy rejects request \\
{\sc krb5kdc_err_badoption }& KDC can't fulfill requested option \\
{\sc krb5kdc_err_etype_nosupp }& KDC has no support for encryption type \\
{\sc krb5kdc_err_sumtype_nosupp }& KDC has no support for checksum type \\
{\sc krb5kdc_err_padata_type_nosupp }&  KDC has no support for padata type \\
{\sc krb5kdc_err_trtype_nosupp }& KDC has no support for transited type \\
{\sc krb5kdc_err_client_revoked }& Clients credentials have been revoked \\
{\sc krb5kdc_err_service_revoked }& Credentials for server have been revoked \\
{\sc krb5kdc_err_tgt_revoked }& TGT has been revoked \\
{\sc krb5kdc_err_client_notyet }& Client not yet valid - try again later \\
{\sc krb5kdc_err_service_notyet }& Server not yet valid - try again later \\
{\sc krb5kdc_err_key_exp }&  	 Password has expired \\
{\sc krb5kdc_preauth_failed }&  	 Preauthentication failed \\
{\sc krb5kdc_err_preauth_require }&	Additional pre-authentication required \\
{\sc krb5kdc_err_server_nomatch }&	Requested server and ticket don't match \\
\multicolumn{2}{c}{error codes 27-30 are currently placeholders}\\

\end{tabular}

\begin{tabular}{ll}
{\sc krb5krb_ap_err_bad_integrity }&  Decrypt integrity check failed \\
{\sc krb5krb_ap_err_tkt_expired }& Ticket expired \\
{\sc krb5krb_ap_err_tkt_nyv }& Ticket not yet valid \\
{\sc krb5krb_ap_err_repeat }& Request is a replay \\
{\sc krb5krb_ap_err_not_us }& The ticket isn't for us \\
{\sc krb5krb_ap_err_badmatch }& Ticket/authenticator don't match \\
{\sc krb5krb_ap_err_skew }& Clock skew too great \\
{\sc krb5krb_ap_err_badaddr }& Incorrect net address \\
{\sc krb5krb_ap_err_badversion }& Protocol version mismatch \\
{\sc krb5krb_ap_err_msg_type }& Invalid message type \\
{\sc krb5krb_ap_err_modified }& Message stream modified \\
{\sc krb5krb_ap_err_badorder }& Message out of order \\
{\sc krb5placehold_43 }&	 KRB5 error code 43 \\
{\sc krb5krb_ap_err_badkeyver }& Key version is not available \\
{\sc krb5krb_ap_err_nokey }& Service key not available \\
{\sc krb5krb_ap_err_mut_fail }& Mutual authentication failed \\
{\sc krb5krb_ap_err_baddirection }& Incorrect message direction \\
{\sc krb5krb_ap_err_method }& Alternative authentication method required \\
{\sc krb5krb_ap_err_badseq }& Incorrect sequence number in message \\
{\sc krb5krb_ap_err_inapp_cksum }& Inappropriate type of checksum in message \\ 
\multicolumn{2}{c}{error codes 51-59 are currently placeholders} \\

{\sc krb5krb_err_generic }& Generic error (see e-text) \\
{\sc krb5krb_err_field_toolong }& Field is too long for this implementation \\
\multicolumn{2}{c}{error codes 62-127 are currently placeholders} \\
\end{tabular}

\begin{tabular}{ll}
{\sc krb5_libos_badlockflag }& Invalid flag for file lock mode \\
{\sc krb5_libos_cantreadpwd }& Cannot read password \\
{\sc krb5_libos_badpwdmatch }& Password mismatch \\
{\sc krb5_libos_pwdintr }&	 Password read interrupted \\
{\sc krb5_parse_illchar }&	 Illegal character in component name \\
{\sc krb5_parse_malformed }& Malformed representation of principal \\
{\sc krb5_config_cantopen }& Can't open/find configuration file \\
{\sc krb5_config_badformat }& Improper format of configuration file \\
{\sc krb5_config_notenufspace }& Insufficient space to return complete information \\
{\sc krb5_badmsgtype }&	 Invalid message type specified for encoding \\
{\sc krb5_cc_badname }&	 Credential cache name malformed \\
{\sc krb5_cc_unknown_type }& Unknown credential cache type  \\
{\sc krb5_cc_notfound }&	 Matching credential not found \\
{\sc krb5_cc_end }&		 End of credential cache reached \\
{\sc krb5_no_tkt_supplied }& Request did not supply a ticket \\
{\sc krb5krb_ap_wrong_princ }&	 Wrong principal in request \\
{\sc krb5krb_ap_err_tkt_invalid }& Ticket has invalid flag set \\
{\sc krb5_princ_nomatch }&	 Requested principal and ticket don't match \\
{\sc krb5_kdcrep_modified }& KDC reply did not match expectations \\
{\sc krb5_kdcrep_skew }&	Clock skew too great in KDC reply \\
{\sc krb5_in_tkt_realm_mismatch }&\parbox[t]{2.5 in}{Client/server realm
mismatch in initial ticket requst}\\

{\sc krb5_prog_etype_nosupp }& Program lacks support for encryption type \\
{\sc krb5_prog_keytype_nosupp }& Program lacks support for key type \\
{\sc krb5_wrong_etype }&	 Requested encryption type not used in message \\
{\sc krb5_prog_sumtype_nosupp }& Program lacks support for checksum type \\
{\sc krb5_realm_unknown }&	 Cannot find KDC for requested realm \\
{\sc krb5_service_unknown }&	Kerberos service unknown \\
{\sc krb5_kdc_unreach }&	 Cannot contact any KDC for requested realm \\
{\sc krb5_no_localname }&	 No local name found for principal name \\

%\multicolumn{1}{c}{some of these should be combined/supplanted by system codes} \\
\end{tabular}

\begin{tabular}{ll}
{\sc krb5_rc_type_exists }&	 Replay cache type is already registered \\
{\sc krb5_rc_malloc }&	 No more memory to allocate (in replay cache code) \\
{\sc krb5_rc_type_notfound }& Replay cache type is unknown \\
{\sc krb5_rc_unknown }&	 Generic unknown RC error \\
{\sc krb5_rc_replay }&	 Message is a replay \\
{\sc krb5_rc_io }&		 Replay I/O operation failed XXX \\
{\sc krb5_rc_noio }&	 \parbox[t]{3in}{Replay cache type does not support non-volatile storage} \\
{\sc krb5_rc_parse }& Replay cache name parse/format error \\
{\sc krb5_rc_io_eof }&	 End-of-file on replay cache I/O \\
{\sc krb5_rc_io_malloc }& \parbox[t]{3in}{No more memory to allocate (in replay cache I/O code)}\\
{\sc krb5_rc_io_perm }&	 Permission denied in replay cache code \\
{\sc krb5_rc_io_io }&	 I/O error in replay cache i/o code \\
{\sc krb5_rc_io_unknown }&	 Generic unknown RC/IO error \\
{\sc krb5_rc_io_space }& Insufficient system space to store replay information \\
{\sc krb5_trans_cantopen }&	 Can't open/find realm translation file \\
{\sc krb5_trans_badformat }& Improper format of realm translation file \\
{\sc krb5_lname_cantopen }&	 Can't open/find lname translation database \\
{\sc krb5_lname_notrans }&	 No translation available for requested principal \\
{\sc krb5_lname_badformat }& Improper format of translation database entry \\
{\sc krb5_crypto_internal }& Cryptosystem internal error \\
{\sc krb5_kt_badname }&	 Key table name malformed \\
{\sc krb5_kt_unknown_type }& Unknown Key table type  \\
{\sc krb5_kt_notfound }&	 Key table entry not found \\
{\sc krb5_kt_end }&		 End of key table reached \\
{\sc krb5_kt_nowrite }&	 Cannot write to specified key table \\
{\sc krb5_kt_ioerr }&	 Error writing to key table \\
{\sc krb5_no_tkt_in_rlm }&	 Cannot find ticket for requested realm \\
{\sc krb5des_bad_keypar }&	 DES key has bad parity \\
{\sc krb5des_weak_key }&	 DES key is a weak key \\
{\sc krb5_bad_keytype }&	 Keytype is incompatible with encryption type \\
{\sc krb5_bad_keysize }&	 Key size is incompatible with encryption type \\
{\sc krb5_bad_msize }&	 Message size is incompatible with encryption type \\
{\sc krb5_cc_type_exists }&	 Credentials cache type is already registered. \\
{\sc krb5_kt_type_exists }&	 Key table type is already registered. \\
{\sc krb5_cc_io }&		 Credentials cache I/O operation failed XXX \\
{\sc krb5_fcc_perm }&	 Credentials cache file permissions incorrect \\
{\sc krb5_fcc_nofile }&	 No credentials cache file found \\
{\sc krb5_fcc_internal }&	 Internal file credentials cache error \\
{\sc krb5_cc_nomem }& \parbox[t]{3in}{No more memory to allocate (in credentials cache code)}\\ 
\end{tabular}

\begin{tabular}{ll}
\multicolumn{2}{c}{errors for dual TGT library calls} \\

{\sc krb5_invalid_flags }& Invalid KDC option combination (library internal error) \\
{\sc krb5_no_2nd_tkt }&	 Request missing second ticket \\
{\sc krb5_nocreds_supplied }& No credentials supplied to library routine \\

\end{tabular}

\begin{tabular}{ll}
\multicolumn{2}{c}{errors for sendauth and recvauth} \\

{\sc krb5_sendauth_badauthvers }& Bad sendauth version was sent \\
{\sc krb5_sendauth_badapplvers }& Bad application version was sent (via sendauth) \\
{\sc krb5_sendauth_badresponse }& Bad response (during sendauth exchange) \\
{\sc krb5_sendauth_rejected }& Server rejected authentication\\
& \ (during sendauth exchange) \\
{\sc krb5_sendauth_mutual_failed }& Mutual authentication failed\\&\ (during sendauth exchange) \\

\end{tabular}

\begin{tabular}{ll}
\multicolumn{2}{c}{errors for preauthentication} \\

{\sc krb5_preauth_bad_type }& Unsupported preauthentication type \\
{\sc krb5_preauth_no_key }&	 Required preauthentication key not supplied \\
{\sc krb5_preauth_failed }&	 Generic preauthentication failure \\

\end{tabular}

\begin{tabular}{ll}
\multicolumn{2}{c}{version number errors} \\

{\sc krb5_rcache_badvno }& Unsupported replay cache format version number \\
{\sc krb5_ccache_badvno }& Unsupported credentials cache format version number \\
{\sc krb5_keytab_badvno }& Unsupported key table format version number \\

\end{tabular}

\begin{tabular}{ll}
\multicolumn{2}{c}{other errors} \\ 

{\sc krb5_prog_atype_nosupp }& Program lacks support for address type \\
{\sc krb5_rc_required }& Message replay detection requires\\&\  rcache parameter \\
{\sc krb5_err_bad_hostname }& Hostname cannot be canonicalized \\
{\sc krb5_err_host_realm_unknown }& Cannot determine realm for host \\
{\sc krb5_sname_unsupp_nametype }& Conversion to service principal undefined\\&\ for name type \\
{\sc krb5krb_ap_err_v4_reply }& Initial Ticket Response appears to be\\
&\ Version 4 error \\
{\sc krb5_realm_cant_resolve }& Cannot resolve KDC for requested realm \\
{\sc krb5_tkt_not_forwardable }& Requesting ticket can't get forwardable tickets \\
\end{tabular}
\end{small}

\subsection{error_table kdb5}

% $Source$
% $Author: epeisach $

The Kerberos v5 database library error code table

\begin{small}
\begin{tabular}{ll}
\multicolumn{2}{c}{From the server side routines} \\
{\sc krb5_kdb_inuse }&	Entry already exists in database\\
{\sc krb5_kdb_uk_serror }&	Database store error\\
{\sc krb5_kdb_uk_rerror }&	Database read error\\
{\sc krb5_kdb_unauth }&	Insufficient access to perform requested operation\\
{\sc krb5_kdb_noentry }&	No such entry in the database\\
{\sc krb5_kdb_ill_wildcard }& Illegal use of wildcard\\
{\sc krb5_kdb_db_inuse }&	Database is locked or in use--try again later\\
{\sc krb5_kdb_db_changed }&	Database was modified during read\\
{\sc krb5_kdb_truncated_record }&	Database record is incomplete or corrupted\\
{\sc krb5_kdb_recursivelock }&	Attempt to lock database twice\\
{\sc krb5_kdb_notlocked }&		Attempt to unlock database when not locked\\
{\sc krb5_kdb_badlockmode }&	Invalid kdb lock mode\\
{\sc krb5_kdb_dbnotinited }&	Database has not been initialized\\
{\sc krb5_kdb_dbinited }&		Database has already been initialized\\
{\sc krb5_kdb_illdirection }&	Bad direction for converting keys\\
{\sc krb5_kdb_nomasterkey }&	Cannot find master key record in database\\
{\sc krb5_kdb_badmasterkey }&	Master key does not match database\\
{\sc krb5_kdb_invalidkeysize }&	Key size in database is invalid\\
{\sc krb5_kdb_cantread_stored }&	Cannot find/read stored master key\\
{\sc krb5_kdb_badstored_mkey }&	Stored master key is corrupted\\
{\sc krb5_kdb_cantlock_db }&	Insufficient access to lock database \\
{\sc krb5_kdb_db_corrupt }&		Database format error\\
{\sc krb5_kdb_bad_version }&	Unsupported version in database entry \\
\end{tabular}
\end{small}

% $Source$
% $Author: epeisach $

\subsection{error_table kv5m}

The Kerberos v5 magic numbers errorcode table follows. These are used
for the magic numbers found in data structures.

\begin{small}
\begin{tabular}{ll} 
{\sc kv5m_none }&		Kerberos V5 magic number table \\
{\sc kv5m_principal }&	Bad magic number for krb5_principal structure \\
{\sc kv5m_data }&		Bad magic number for krb5_data structure \\
{\sc kv5m_keyblock }&	Bad magic number for krb5_keyblock structure \\
{\sc kv5m_checksum }&	Bad magic number for krb5_checksum structure \\
{\sc kv5m_encrypt_block }&	Bad magic number for krb5_encrypt_block structure \\
{\sc kv5m_enc_data }&	Bad magic number for krb5_enc_data structure \\
{\sc kv5m_cryptosystem_entry }&	Bad magic number for krb5_cryptosystem_entry\\&\ structure \\
{\sc kv5m_cs_table_entry }&	Bad magic number for krb5_cs_table_entry structure \\
{\sc kv5m_checksum_entry }&	Bad magic number for krb5_checksum_entry structure \\

{\sc kv5m_authdata }&	Bad magic number for krb5_authdata structure \\
{\sc kv5m_transited }&	Bad magic number for krb5_transited structure \\
{\sc kv5m_enc_tkt_parT }&	Bad magic number for krb5_enc_tkt_part structure \\
{\sc kv5m_ticket }&		Bad magic number for krb5_ticket structure \\
{\sc kv5m_authenticator }&	Bad magic number for krb5_authenticator structure \\
{\sc kv5m_tkt_authent }&	Bad magic number for krb5_tkt_authent structure \\
{\sc kv5m_creds }&		Bad magic number for krb5_creds structure \\
{\sc kv5m_last_req_entry }&	Bad magic number for krb5_last_req_entry structure \\
{\sc kv5m_pa_data }&		Bad magic number for krb5_pa_data structure \\
{\sc kv5m_kdc_req }&		Bad magic number for krb5_kdc_req structure \\
{\sc kv5m_enc_kdc_rep_part }& Bad magic number for krb5_enc_kdc_rep_part structure \\
{\sc kv5m_kdc_rep }&		Bad magic number for krb5_kdc_rep structure \\
{\sc kv5m_error }&		Bad magic number for krb5_error structure \\
{\sc kv5m_ap_req }&		Bad magic number for krb5_ap_req structure \\
{\sc kv5m_ap_rep }&		Bad magic number for krb5_ap_rep structure \\
{\sc kv5m_ap_rep_enc_part }&	Bad magic number for krb5_ap_rep_enc_part structure \\
{\sc kv5m_response }&	Bad magic number for krb5_response structure \\
{\sc kv5m_safe }&		Bad magic number for krb5_safe structure \\
{\sc kv5m_priv }&		Bad magic number for krb5_priv structure \\
{\sc kv5m_priv_enc_part }&	Bad magic number for krb5_priv_enc_part structure \\
{\sc kv5m_cred }&		Bad magic number for krb5_cred structure \\
{\sc kv5m_cred_info }&	Bad magic number for krb5_cred_info structure \\
{\sc kv5m_cred_enc_part }&	Bad magic number for krb5_cred_enc_part structure \\
{\sc kv5m_pwd_data }&	Bad magic number for krb5_pwd_data structure \\
{\sc kv5m_address }&	Bad magic number for krb5_address structure \\
{\sc kv5m_keytab_entry }&	Bad magic number for krb5_keytab_entry structure \\
{\sc kv5m_context }&	Bad magic number for krb5_context structure \\
{\sc kv5m_os_context }&	Bad magic number for krb5_os_context structure \\

\end{tabular}
\end{small}

\subsection{error_table asn1}

The Kerberos v5/ASN.1 error table mappings

\begin{small}
\begin{tabular}{ll}
{\sc asn1_bad_timeformat }&	ASN.1 failed call to system time library \\
{\sc asn1_missing_field }&	ASN.1 structure is missing a required field \\
{\sc asn1_misplaced_field }&	ASN.1 unexpected field number \\
{\sc asn1_type_mismatch }&	ASN.1 type numbers are inconsistent \\
{\sc asn1_overflow }&	ASN.1 value too large \\
{\sc asn1_overrun }&	ASN.1 encoding ended unexpectedly \\
{\sc asn1_bad_id }&	ASN.1 identifier doesn't match expected value \\
{\sc asn1_bad_length }&	ASN.1 length doesn't match expected value \\
{\sc asn1_bad_format }&	ASN.1 badly-formatted encoding \\
{\sc asn1_parse_error }&	ASN.1 parse error \\
\end{tabular}
\end{small}



%\addtolength{\oddsidemargin}{-1in}
%\addtolength{\evensidemargin}{1.00in}
%\addtolength{\textwidth}{-1.75in}
\newpage

\section{libkrb5.a functions}
This section describes the functions provided in the \libname{libkrb5.a}
library.  The library is built from several pieces, mostly for convenience in
programming, maintenance, and porting.

\ifdraft\sloppy\fi

\subsection{Main functions}
The main functions deal with the nitty-gritty details: verifying
tickets, creating authenticators, and the like.

\subsubsection{The krb5_context}
The \datatype{krb5_context} is designed to represent the per process
state. When the library is made thread-safe, the context will represent
the per-thread state. Global parameters which are ``context'' specific
are stored in this structure, including default realm, default
encryption type, default configuration files and the like. Functions
exist to provide full access to the data structures stored in the
context and should not be accessed directly by developers.

\begin{funcdecl}{krb5_init_context}{krb5_error_code}{\funcout}
\funcarg{krb5_context *}{context}
\end{funcdecl}

Initializes the context \funcparam{*context} for the
application. Currently the context contains the encryption types, a
pointer to operating specific data and the default realm. In the future,
the context may be also contain thread specific data.  The data in the
context should be freed with \funcname{krb5_free_context}.

Returns system errors.

\begin{funcdecl}{krb5_free_context}{void}{\funcinout}
\funcarg{krb5_context}{context}
\end{funcdecl}

Frees the context returned by \funcname{krb5_init_context}. Internally
calls \funcname{krb5_os_free_context}.

\begin{funcdecl}{krb5_set_default_in_tkt_etypes}{krb5_error_code}{\funcinout}
\funcarg{krb5_context}{context}
\funcin
\funcarg{const krb5_enctype *}{etypes}
\end{funcdecl}

Sets the desired default encryption type \funcparam{etypes} for the context
if valid.

Returns {\sc enomem}, {\sc krb5_prog_etype_nosupp}.

\begin{funcdecl}{krb5_get_default_in_tkt_etypes}{krb5_error_code}{\funcinout}
\funcarg{krb5_context}{context}
\funcout
\funcarg{krb5_enctype **}{etypes}
\end{funcdecl}

Retrieves the default encryption types from the context and stores them
in \funcarg{etypes} which should be freed by the caller.

Returns {\sc enomem}.

\subsubsection{The krb5_auth_context}

While the \datatype{krb5_context} represents a per-process or per-thread
context, the \datatype{krb5_auth_context} represents a per-connection
context are are used by the various functions involved directly in
client/server authentication.  Some of the data stored in this context
include keyblocks, addresses, sequence numbers, authenticators, checksum
type, and replay cache pointer. 

\begin{funcdecl}{krb5_auth_con_init}{krb5_error_code}{\funcinout}
\funcarg{krb5_context}{context}
\funcout
\funcarg{krb5_auth_context *}{auth_context}
\end{funcdecl}

The auth_context may be described as a per connection context. This
context contains all data pertinent to the the various authentication
routines. This function initializes the auth_context. 

The default flags for the context are set to enable the use of the replay cache
(KRB5_AUTH_CONTEXT_DO_TIME) but no sequence numbers. The function
\funcname{krb5_auth_con_setflags} allows the flags to be changed.

The default checksum type is set to CKSUMTYPE_RSA_MD4_DES. This may be
changed with \funcname{krb5_auth_con_setcksumtype}.

The auth_context structure should be freed with
\funcname{krb5_auth_con_free}.

\begin{funcdecl}{krb5_auth_con_free}{krb5_error_code}{\funcinout}
\funcarg{krb5_context}{context}
\funcarg{krb5_auth_context}{auth_context}
\end{funcdecl}

Frees the auth_context \funcparam{auth_context} returned by
\funcname{krb5_auth_con_init}.

% perhaps some comment about which substructures are freed and which are not?

\begin{funcdecl}{krb5_auth_con_setflags}{krb5_error_code}{\funcinout}
\funcarg{krb5_context}{context}
\funcarg{krb5_auth_context}{auth_context}
\funcin
\funcarg{krb5_int32}{flags}
\end{funcdecl}

Sets the flags of \funcparam{auth_context} to funcparam{flags}. Valid
flags are:

\begin{tabular}{ll}
\multicolumn{1}{c}{Symbol} & Meaning \\
KRB5_AUTH_CONTEXT_DO_TIME & Use timestamps \\
KRB5_AUTH_CONTEXT_RET_TIME & Save timestamps\\ &\  to output structure\\
KRB5_AUTH_CONTEXT_DO_SEQUENCE   & Use sequence numbers \\
KRB5_AUTH_CONTEXT_RET_SEQUENCE  & Copy sequence numbers \\ &\ to output structure\\
\end{tabular}


\begin{funcdecl}{krb5_auth_con_getflags}{krb5_error_code}{\funcinout}
\funcarg{krb5_context}{context}
\funcin
\funcarg{krb5_auth_context}{auth_context}
\funcout
\funcarg{krb5_int32 *}{flags}
\end{funcdecl}

Retrievs the flags of \funcparam{auth_context}.

\begin{funcdecl}{krb5_auth_con_setaddrs}{krb5_error_code}{\funcinout}
\funcarg{krb5_context}{context}
\funcarg{krb5_auth_context}{auth_context}
\funcin
\funcarg{krb5_address *}{local_addr}
\funcarg{krb5_address *}{remote_addr}
\end{funcdecl}

Copies the \funcparam{local_addr} and \funcparam{remote_addr} into the
\funcparam{auth_context}. If either address is NULL, the previous
address remains in place.

\begin{funcdecl}{krb5_auth_con_getaddrs}{krb5_error_code}{\funcinout}
\funcarg{krb5_context}{context}
\funcarg{krb5_auth_context}{auth_context}
\funcout
\funcarg{krb5_address **}{local_addr}
\funcarg{krb5_address **}{remote_addr}
\end{funcdecl}

Retrieves  \funcparam{local_addr} and \funcparam{remote_addr} from the
\funcparam{auth_context}. If \funcparam{local_addr} or
\funcparam{remote_addr} is not NULL, the memory is first freed with
\funcname{krb5_free_address} and then newly allocated. It is the callers
responsibility to free the returned addresses in this way.


\begin{funcdecl}{krb5_auth_con_setports}{krb5_error_code}{\funcinout}
\funcarg{krb5_context}{context}
\funcarg{krb5_auth_context}{auth_context}
\funcin
\funcarg{krb5_address *}{local_port}
\funcarg{krb5_address *}{remote_port}
\end{funcdecl}

Copies the \funcparam{local_port} and \funcparam{remote_port} addresses
into the \funcparam{auth_context}. If either address is NULL, the previous
address remains in place. These addresses are set by
\funcname{krb5_auth_con_genaddrs}. 

\begin{funcdecl}{krb5_auth_con_setuserkey}{krb5_error_code}{\funcinout}
\funcarg{krb5_context}{context}
\funcarg{krb5_auth_context}{auth_context}
\funcin
\funcarg{krb5_keyblock *}{keyblock}
\end{funcdecl}

This function overloads the keyblock field. It is only useful prior to a
\funcname{krb5_rd_req_decode} call for user to user authentication where
the server has the key and needs to use it to decrypt the incoming
request.  Once decrypted this key is no longer necessary and is then
overwritten with the session key sent by the client. 

\begin{funcdecl}{krb5_auth_con_getkey}{krb5_error_code}{\funcinout}
\funcarg{krb5_context}{context}
\funcarg{krb5_auth_context}{auth_context}
\funcout
\funcarg{krb5_keyblock **}{keyblock}
\end{funcdecl}

Retrieves the keyblock stored in \funcparam{auth_context}. The memory
allocated in this function should be freed with a call to
\funcname{krb5_free_keyblock}. 

\begin{funcdecl}{krb5_auth_con_getrecvsubkey}{krb5_error_code}{\funcinout}
\funcarg{krb5_context}{context}
\funcarg{krb5_auth_context}{auth_context}
\funcout
\funcarg{krb5_keyblock **}{keyblock}
\end{funcdecl}

Retrieves the recv\_subkey keyblock stored in
\funcparam{auth_context}. The memory allocated in this function should
be freed with a call to \funcname{krb5_free_keyblock}.

\begin{funcdecl}{krb5_auth_con_getsendsubkey}{krb5_error_code}{\funcinout}
\funcarg{krb5_context}{context}
\funcarg{krb5_auth_context}{auth_context}
\funcout
\funcarg{krb5_keyblock **}{keyblock}
\end{funcdecl}

Retrieves the send\_subkey keyblock stored in
\funcparam{auth_context}. The memory allocated in this function should
be freed with a call to \funcname{krb5_free_keyblock}.

\begin{funcdecl}{krb5_auth_con_setrecvsubkey}{krb5_error_code}{\funcinout}
\funcarg{krb5_context}{context}
\funcarg{krb5_auth_context}{auth_context}
\funcout
\funcarg{krb5_keyblock *}{keyblock}
\end{funcdecl}

Sets the recv\_subkey keyblock stored in \funcparam{auth_context}.

\begin{funcdecl}{krb5_auth_con_setsendsubkey}{krb5_error_code}{\funcinout}
\funcarg{krb5_context}{context}
\funcarg{krb5_auth_context}{auth_context}
\funcout
\funcarg{krb5_keyblock *}{keyblock}
\end{funcdecl}

Sets the send\_subkey keyblock stored in \funcparam{auth_context}.

\begin{funcdecl}{krb5_auth_setcksumtype}{krb5_error_code}{\funcinout}
\funcarg{krb5_context}{context}
\funcarg{krb5_auth_context}{auth_context}
\funcin
\funcarg{krb5_cksumtype}{cksumtype}
\end{funcdecl}

Sets the checksum type used by the other functions in the library. 

\begin{funcdecl}{krb5_auth_getlocalseqnumber}{krb5_error_code}{\funcinout}
\funcarg{krb5_context}{context}
\funcarg{krb5_auth_context}{auth_context}
\funcin
\funcarg{krb5_int32 *}{seqnumber}
\end{funcdecl}

Retrieves the local sequence number that was used during authentication
and stores it in \funcparam{seqnumber}.

\begin{funcdecl}{krb5_auth_getremoteseqnumber}{krb5_error_code}{\funcinout}
\funcarg{krb5_context}{context}
\funcarg{krb5_auth_context}{auth_context}
\funcin
\funcarg{krb5_int32 *}{seqnumber}
\end{funcdecl}

Retrieves the remote sequence number that was used during authentication
and stores it in \funcparam{seqnumber}.

\begin{funcdecl}{krb5_auth_getauthenticator}{krb5_error_code}{\funcinout}
\funcarg{krb5_context}{context}
\funcarg{krb5_auth_context}{auth_context}
\funcout
\funcarg{krb5_authenticator **}{authenticator}
\end{funcdecl}

Retrieves the authenticator that was used during mutual
authentication. It is the callers responsibility to free the memory
allocated to \funcparam{authenticator} by calling
\funcname{krb5_free_authenticator}. 

\begin{funcdecl}{krb5_auth_con_initivector}{krb5_error_code}{\funcinout}
\funcarg{krb5_context}{context}
\funcarg{krb5_auth_context}{auth_context}
\end{funcdecl}

Allocates memory for and zeros the initial vector in the
\funcparam{auth_context} keyblock.

\begin{funcdecl}{krb5_auth_con_setivector}{krb5_error_code}{\funcinout}
\funcarg{krb5_context}{context}
\funcarg{krb5_auth_context *}{auth_context}
\funcin
\funcarg{krb5_pointer}{ivector}
\end{funcdecl}

Sets the i_vector portion of \funcparam{auth_context} to
\funcparam{ivector}. 

\begin{funcdecl}{krb5_auth_con_setrcache}{krb5_error_code}{\funcinout}
\funcarg{krb5_context}{context}
\funcarg{krb5_auth_context}{auth_context}
\funcin
\funcarg{krb5_rcache}{rcache}
\end{funcdecl}

Sets the replay cache that is used by the authentication routines to \funcparam{rcache}.

%%
%% The following prototypes exist, but there is no code at this time
%% krb5_auth_con_getcksumtype, krb5_auth_con_getivector,
%% krb5_auth_con_getrcache. -- Ezra

\subsubsection{Principal access functions}

Principals define a uniquely named client or server instance that
participates in a network communication. The following functions allow
one to create, modify and access portions of the
\datatype{krb5_principal}. 

Other functions found in orther portions of the manual include
\funcname{krb5_sname_to_principal}, \funcname{krb5_free_principal}, 

While it is possible to directly access the data structure in the
structure, it is recommended that the functions be used. 

\begin{funcdecl}{krb5_parse_name}{krb5_error_code}{\funcinout}
\funcarg{krb5_context}{context}
\funcin
\funcarg{const char *}{name}
\funcout
\funcarg{krb5_principal *}{principal}
\end{funcdecl}

Converts a single-string representation \funcparam{name} of the
principal name to the multi-part principal format used in the protocols.

A single-string representation of a Kerberos name consists of one or
more principal name components, separated by slashes, optionally
followed by the ``@'' character and a realm name.  If the realm name
is not specified, the local realm is used.

The slash and ``@'' characters may be quoted (i.e., included as part
of a component rather than as a component separator or realm prefix)
by preceding them with a backslash (``$\backslash$'') character.
Similarly, newline, tab, backspace, and null characters may be
included in a component by using $\backslash{}n$, $\backslash{}t$,
$\backslash{}b$ or $\backslash{}0$, respectively.

The realm in a Kerberos name may not contain the slash, colon or null
characters.

\funcparam{*principal} will point to allocated storage which should be freed by
the caller (using \funcname{krb5_free_principal}) after use.

\funcname{krb5_parse_name} returns KRB5_PARSE_MALFORMED if the string is
 badly formatted, or ENOMEM if space for the return value can't be
allocated.

\begin{funcdecl}{krb5_unparse_name}{krb5_error_code}{\funcinout}
\funcarg{krb5_context}{context}
\funcin
\funcarg{krb5_const_principal}{principal}
\funcout
\funcarg{char **}{name}
\end{funcdecl}

Converts the multi-part principal name \funcparam{principal} from the
format used in the protocols to a single-string representation of the
name.  The resulting single-string representation will use the format
and quoting conventions described for \funcname{krb5_parse_name}
above.

\funcparam{*name} points to allocated storage and should be freed by the caller
when finished.

\funcname{krb5_unparse_name} returns {\sc KRB_PARSE_MALFORMED} if the principal
does not contain at least 2 components, and system errors ({\sc ENOMEM} if
unable to allocate memory).

\begin{funcdecl}{krb5_unparse_name_ext}{krb5_error_code}{\funcinout}
\funcarg{krb5_context}{context}
\funcin
\funcarg{krb5_const_principal}{principal}
\funcinout
\funcarg{char **}{name}
\funcarg{unsigned int *}{size}
\end{funcdecl}

\funcname{krb5_unparse_name_ext} is designed for applications which
must unparse a large number of principals, and are concerned about the
speed impact of needing to do a lot of memory allocations and
deallocations.  It functions similarly to \funcname{krb5_unparse_name}
except if \funcparam{*name} is non-null, in which case, it is assumed
to contain an allocated buffer of size \funcparam{*size} and this
buffer will be resized with \funcname{realloc} to hold the unparsed
name.  Note that in this case,
\funcparam{size} must not be null.  

If \funcparam{size} is non-null (whether or not \funcparam{*name} is
null when the function is called), it will be filled in with the size
of the unparsed name upon successful return.

\begin{funcdecl}{krb5_princ_realm}{krb5_data *}{\funcinout}
\funcparam{krb5_context}{context}
\funcparam{krb5_principal}{principal}
\end{funcdecl}

A macro which returns the realm of \funcparam{principal}.

\begin{funcdecl}{krb5_princ_set_realm}{void}{\funcinout}
\funcparam{krb5_context}{context}
\funcparam{krb5_principal}{principal}
\funcparam{krb5_data *}{realm}
\end{funcdecl}

A macro which returns sets the realm of \funcparam{principal} to
\funcparam{realm}. 

\begin{funcdecl}{krb5_princ_set_realm_data}{void}{\funcinout}
\funcparam{krb5_context}{context}
\funcparam{krb5_principal}{principal}
\funcparam{char *}{data}
\end{funcdecl}

\internalfunc

A macro which returns sets the data portion of the realm of
\funcparam{principal} to \funcparam{data}. 

\begin{funcdecl}{krb5_princ_set_realm_length}{void}{\funcinout}
\funcparam{krb5_context}{context}
\funcparam{krb5_principal}{principal}
\funcparam{int}{length}
\end{funcdecl}

\internalfunc

A macro which returns sets the length \funcparam{principal} to
\funcparam{length}. 

\begin{funcdecl}{krb5_princ_size}{void}{\funcinout}
\funcparam{krb5_context}{context}
\funcparam{krb5_principal}{principal}
\end{funcdecl}

\internalfunc

A macro which gives the number of elements in the principal.
May also be used on the left size of an assignment.

\begin{funcdecl}{krb5_princ_type}{void}{\funcinout}
\funcparam{krb5_context}{context}
\funcparam{krb5_principal}{principal}
\end{funcdecl}

\internalfunc

A macro which gives the type of the principal.
May also be used on the left size of an assignment.

\begin{funcdecl}{krb5_princ_data}{}
\funcparam{krb5_context}{context}
\funcparam{krb5_principal}{principal}
\end{funcdecl}

\internalfunc

A macro which gives the pointer to data portion of the principal.
May also be used on the left size of an assignment.


\begin{funcdecl}{krb5_princ_component}{}
\funcparam{krb5_context}{context}
\funcparam{krb5_principal}{principal}
\funcparam{int}{i}
\end{funcdecl}

\internalfunc

A macro which gives the pointer to ith element of the principal.
May also be used on the left size of an assignment.

\begin{funcdecl}{krb5_build_principal}{krb5_error_code}{\funcinout}
\funcarg{krb5_context}{context}
\funcout
\funcarg{krb5_principal *}{princ}
\funcin
\funcarg{unsigned int}{rlen}
\funcarg{const char *}{realm}
\funcarg{char}{*s1, *s2, ..., 0}
\end{funcdecl}
\begin{funcdecl}{krb5_build_principal_va}{krb5_error_code}{\funcinout}
\funcarg{krb5_context}{context}
\funcout
\funcarg{krb5_principal *}{princ}
\funcin
\funcarg{unsigned int}{rlen}
\funcarg{const char *}{realm}
\funcarg{va_list}{ap}
\end{funcdecl}

\begin{sloppypar}
\funcname{krb5_build_principal} and
\funcname{krb5_build_principal_va} 
perform the same function; the former takes variadic arguments, while
the latter takes a pre-computed varargs pointer.
\end{sloppypar}

Both functions take a realm name \funcparam{realm}, realm name length
\funcparam{rlen}, and a list of null-terminated strings, and fill in a
pointer to a principal structure \funcparam{princ}, making it point to a
structure representing the named principal.
The last string must be followed in the argument list by a null pointer.


\begin{funcdecl}{krb5_build_principal_ext}{krb5_error_code}{\funcinout}
\funcarg{krb5_context}{context}
\funcout
\funcarg{krb5_principal *}{princ}
\funcin
\funcarg{unsigned int}{rlen}
\funcarg{const char *}{realm}
\funcarg{}{int len1, char *s1, int len2, char *s2, ..., 0}
\end{funcdecl}

\funcname{krb5_build_principal_ext} is similar to
\funcname{krb5_build_principal} but it takes its components as a list of
(length, contents) pairs rather than a list of null-terminated strings.
A length of zero indicates the end of the list.

\begin{funcdecl}{krb5_copy_principal}{krb5_error_code}{\funcinout}
\funcarg{krb5_context}{context}
\funcin
\funcarg{krb5_const_principal}{inprinc}
\funcout
\funcarg{krb5_principal *}{outprinc}
\end{funcdecl}

Copy a principal structure, filling in \funcparam{*outprinc} to point to
the newly allocated copy, which should be freed with
\funcname{krb5_free_principal}.

\begin{funcdecl}{krb5_principal_compare}{krb5_boolean}{\funcinout}
\funcarg{krb5_context}{context}
\funcin
\funcarg{krb5_const_principal}{princ1}
\funcarg{krb5_const_principal}{princ2}
\end{funcdecl}

If the two principals are the same, return TRUE, else return FALSE.

\begin{funcdecl}{krb5_realm_compare}{krb5_boolean}{\funcinout}
\funcarg{krb5_context}{context}
\funcin
\funcarg{krb5_const_principal}{princ1}
\funcarg{krb5_const_principal}{princ2}
\end{funcdecl}

If the realms of the two principals are the same, return TRUE, else
return FALSE. 


\begin{funcdecl}{krb5_425_conv_principal}{krb5_error_code}{\funcinout}
\funcarg{krb5_context}{context}
\funcin
\funcarg{const char *}{name}
\funcarg{const char *}{instance}
\funcarg{const char *}{realm}
\funcout
\funcarg{krb5_principal *}{princ}
\end{funcdecl}

Build a principal \funcparam{princ} from a V4 specification made up of 
\funcparam{name}.\funcparam{instance}@\funcparam{realm}. The routine is
site-customized to convert the V4 naming scheme to a V5 one. For
instance, the V4 ``rcmd'' is changed to ``host''. 

The returned principal should be freed with
\funcname{krb5_free_principal}. 

\subsubsection{The application functions}

\begin{funcdecl}{krb5_encode_kdc_rep}{krb5_error_code}{\funcin}
\funcarg{const krb5_msgtype}{type}
\funcarg{const krb5_enc_kdc_rep_part *}{encpart}
\funcarg{krb5_encrypt_block *}{eblock}
\funcarg{const krb5_keyblock *}{client_key}
\funcinout
\funcarg{krb5_kdc_rep *}{dec_rep}
\funcout
\funcarg{krb5_data **}{enc_rep}
\end{funcdecl}

\internalfunc

Takes KDC rep parts in \funcparam{*rep} and \funcparam{*encpart}, and
formats it into \funcparam{*enc_rep}, using message type
\funcparam{type} and encryption key \funcparam{client_key} and
encryption block \funcparam{eblock}.

\funcparam{enc_rep{\ptsto}data} will point to  allocated storage upon
non-error return; the caller should free it when finished.

Returns system errors.

\begin{funcdecl}{krb5_decode_kdc_rep}{krb5_error_code}{\funcinout}
\funcarg{krb5_context}{context}
\funcin
\funcarg{krb5_data *}{enc_rep}
\funcarg{const krb5_keyblock *}{key}
\funcarg{const krb5_enctype}{etype}
\funcout
\funcarg{krb5_kdc_rep **}{dec_rep}
\end{funcdecl}

\internalfunc

Takes a KDC_REP message and decrypts encrypted part using
\funcparam{etype} and \funcparam{*key}, putting result in \funcparam{*dec_rep}.
The pointers in \funcparam{dec_rep}
are all set to allocated storage which should be freed by the caller
when finished with the response (by using \funcname{krb5_free_kdc_rep}).


If the response isn't a KDC_REP (tgs or as), it returns an error from
the decoding routines.

Returns errors from encryption routines, system errors.

\begin{funcdecl}{krb5_kdc_rep_decrypt_proc}{krb5_error_code}{\funcinout}
\funcarg{krb5_context}{context}
\funcin
\funcarg{const krb5_keyblock *}{key}
\funcarg{krb5_const_pointer}{decryptarg}
\funcinout
\funcarg{krb5_kdc_rep *}{dec_rep}
\end{funcdecl}

Decrypt the encrypted portion of \funcparam{dec_rep}, using the
encryption key \funcparam{key}.  The parameter \funcparam{decryptarg} is
ignored.

The result is in allocated storage pointed to by
\funcparam{dec_rep{\ptsto}enc_part2}, unless some error occurs.

This function is suitable for use as the \funcparam{decrypt_proc}
argument to \funcname{krb5_get_in_tkt}.

\begin{funcdecl}{krb5_encrypt_tkt_part}{krb5_error_code}{\funcinout}
\funcarg{krb5_context}{context}
\funcin
\funcarg{const krb5_keyblock *}{srv_key}
\funcinout
\funcarg{krb5_ticket *}{dec_ticket}
\end{funcdecl}

\internalfunc

\begin{sloppypar}
Encrypts the unecrypted part of the ticket found in 
\funcparam{dec_ticket{\ptsto}enc_part2} using
\funcparam{srv_key}, and places result in 
\funcparam{dec_ticket{\ptsto}enc_part}.
The \funcparam{dec_ticket{\ptsto}enc_part} will be allocated by this
function.
\end{sloppypar}

Returns errors from encryption routines, system errors

\funcparam{enc_part{\ptsto}data} is allocated and filled in with
encrypted stuff.

\begin{funcdecl}{krb5_decrypt_tkt_part}{krb5_error_code}{\funcinout}
\funcarg{krb5_context}{context}
\funcin
\funcarg{const krb5_keyblock *}{srv_key}
\funcinout
\funcarg{krb5_ticket *}{dec_ticket}
\end{funcdecl}

\internalfunc

Takes encrypted \funcparam{dec_ticket{\ptsto}enc_part}, decrypts with
\funcparam{dec_ticket{\ptsto}etype}
using \funcparam{srv_key}, and places result in
\funcparam{dec_ticket{\ptsto}enc_part2}.  The storage of
\funcparam{dec_ticket{\ptsto}enc_part2} will be allocated before return.

Returns errors from encryption routines, system errors

\begin{funcdecl}{krb5_send_tgs}{krb5_error_code}{\funcinout}
\funcarg{krb5_context}{context}
\funcin
\funcarg{const krb5_flags}{kdcoptions}
\funcarg{const krb5_ticket_times *}{timestruct}
\funcarg{const krb5_enctype *}{etypes}
\funcarg{const krb5_cksumtype}{sumtype}
\funcarg{krb5_const_principal}{sname}
\funcarg{krb5_address * const *}{addrs}
\funcarg{krb5_authdata * const *}{authorization_data}
\funcarg{krb5_pa_data * const *}{padata}
\funcarg{const krb5_data *}{second_ticket}
\funcinout
\funcarg{krb5_creds *}{in_cred}
\funcout
\funcarg{krb5_response *}{rep}
\end{funcdecl}

\internalfunc

Sends a request to the TGS and waits for a response.
\funcparam{kdcoptions} is used for the options in the KRB_TGS_REQ.
\funcparam{timestruct} values are used for from, till, and rtime in the
KRB_TGS_REQ.
\funcparam{etypes} is a list of etypes used in the KRB_TGS_REQ.
\funcparam{sumtype} is used for the checksum in the AP_REQ in the KRB_TGS_REQ.
\funcparam{sname} is used for sname in the KRB_TGS_REQ.
\funcparam{addrs}, if non-NULL, is used for addresses in the KRB_TGS_REQ.
\funcparam{authorization_data}, if non-NULL, is used for
\funcparam{authorization_data} in the KRB_TGS_REQ.  
\funcparam{padata}, if non-NULL, is combined with any other supplied
pre-authentication data for the KRB_TGS_REQ.
\funcparam{second_ticket}, if required by options, is used for the 2nd
ticket in the KRB_TGS_REQ.
\funcparam{in_cred} is used for the ticket and session key in the KRB_AP_REQ header in the KRB_TGS_REQ.

The KDC realm is extracted from \funcparam{in_cred{\ptsto}server}'s realm.

The response is placed into \funcparam{*rep}.
\funcparam{rep{\ptsto}response.data} is set to point at allocated storage
which should be freed by the caller when finished.

Returns system errors.

\begin{funcdecl}{krb5_get_cred_from_kdc}{krb5_error_code}{\funcinout}
\funcarg{krb5_context}{context}
\funcin
\funcarg{krb5_ccache}{ccache}
\funcarg{krb5_creds *}{in_cred}
\funcout                        
\funcarg{krb5_cred **}{out_cred}
\funcarg{krb5_creds ***}{tgts}
\end{funcdecl}


Retrieve credentials for principal \funcparam{in_cred{\ptsto}client},
server \funcparam{creds{\ptsto}server}, possibly
\funcparam{creds{\ptsto}second_ticket} if needed by the ticket flags.

\funcparam{ccache} is used to fetch initial TGT's to start the authentication
path to the server.

Credentials are requested from the KDC for the server's realm.  Any
TGT credentials obtained in the process of contacting the KDC are
returned in an array of credentials; \funcparam{tgts} is filled in to
point to an array of pointers to credential structures (if no TGT's were
used, the pointer is zeroed).  TGT's may be returned even if no useful
end ticket was obtained.

The returned credentials are NOT cached.

If credentials are obtained, \funcparam{creds} is filled in with the results;
\funcparam{creds{\ptsto}ticket} and
\funcparam{creds{\ptsto}keyblock{\ptsto}key} are set to allocated storage,
which should be freed by the caller when finished.

Returns errors, system errors.

\begin{funcdecl}{krb5_get_cred_via_tkt}{krb5_error_code}{\funcinout}
\funcarg{krb5_context}{context}
\funcin
\funcarg{krb5_creds *}{tkt}
\funcarg{const krb5_flags}{kdcoptions}
\funcarg{krb5_address *const *}{address}
\funcarg{krb5_creds *}{in_cred}
\funcout
\funcarg{krb5_creds **}{out_cred}
\end{funcdecl}

Takes a ticket \funcparam{tkt} and a target credential
\funcparam{in_cred}, attempts to fetch a TGS from the KDC. Upon
success the resulting is stored in \funcparam{out_cred}. The memory
allocated in \funcparam{out_cred} should be freed by the called when
finished by using \funcname{krb5_free_creds}. 

\funcparam{kdcoptions} refers to the options as listed in Table
\ref{KDCOptions}. The optional \funcparam{address} is used for addressed
in the KRB_TGS_REQ (see \funcname{krb5_send_tgs}).

Returns errors, system errors.


\begin{funcdecl}{krb5_get_credentials}{krb5_error_code}{\funcinout}
\funcarg{krb5_context}{context}
\funcin
\funcarg{const krb5_flags}{options}
\funcarg{krb5_ccache}{ccache}
\funcarg{krb5_creds *}{in_creds}
\funcout
\funcarg{krb5_creds *}{out_creds}
\end{funcdecl}

This routine attempts to use the credentials cache \funcparam{ccache} or a TGS
exchange to get an additional ticket for the client identified by
\funcparam{in_creds{\ptsto}client}, with following information: 
\begin{itemize}
\item {\bf The server} identified by \funcparam{in_creds{\ptsto}server} 
\item {\bf The options} in \funcparam{options}.
Valid choices are KRB5_GC_USER_USER and KRB5_GC_GC_CACHED
\item {\bf The expiration date} specified in
\funcparam{in_creds{\ptsto}times.endtime}  
\item {\bf The session key type} specified in
\funcparam{in_creds{\ptsto}keyblock.keytype} if it is non-zero.
\end{itemize}

If \funcparam{options} specifies KRB5_GC_CACHED,
then \funcname{krb5_get_credentials} will only search the credentials cache
for a ticket.  

If \funcparam{options} specifies KRB5_GC_USER_USER, then
\funcname{krb5_get_credentials} will get credentials for a user to user
authentication.  In a user to user authentication, the secret key for
the server 
is the session key from the server's ticket-granting-ticket
(TGT).  The TGT is passed from the server to the client over the
network --- this is safe since the TGT is encrypted in a key
known only by the Kerberos server --- and the client must pass
this TGT to \funcname{krb5_get_credentials} in
\funcparam{in_creds{\ptsto}second_ticket}.  The Kerberos server will use
this TGT to construct a user to user ticket which can be verified by
the server by using the session key from its TGT.

The effective {\bf expiration date} is the minimum of the following:
\begin{itemize}
\item The expiration date as specified in
\funcparam{in_creds{\ptsto}times.endtime} 
\item The requested start time plus the maximum lifetime of the
server as specified by the server's entry in the
Kerberos database.
\item The requested start time plus the maximum lifetime of tickets
allowed in the local site, as specified by the KDC.
This is currently a compile-time option,
KRB5_KDB_MAX_LIFE in config.h, and is by default 1 day.
\end{itemize}

If any special authorization data needs to be included in the ticket,
--- for example, restrictions on how the ticket can be used --- 
they should be specified in \funcparam{in_creds{\ptsto}authdata}.   If there
is no special authorization data to be passed,
\funcparam{in_creds{\ptsto}authdata} should be NULL.

Any returned ticket and intermediate ticket-granting tickets are
stored in \funcparam{ccache}.

Returns errors from encryption routines, system errors.

\begin{funcdecl}{krb5_get_in_tkt}{krb5_error_code}{\funcinout}
\funcarg{krb5_context}{context}
\funcin
\funcarg{const krb5_flags}{options}
\funcarg{krb5_address * const *}{addrs}
\funcarg{const krb5_enctype *}{etypes}
\funcarg{const krb5_preauthtype *}{ptypes}
\funcfuncarg{krb5_error_code}{(*key_proc)}
        \funcarg{krb5_context}{context}
        \funcarg{const krb5_keytype}{type}
        \funcarg{krb5_data *}{salt}
        \funcarg{krb5_const_pointer}{keyseed}
        \funcarg{krb5_keyblock **}{key}
\funcendfuncarg
\funcarg{krb5_const_pointer}{keyseed}
\funcfuncarg{krb5_error_code}{(*decrypt_proc)}
        \funcarg{krb5_context}{context}
        \funcarg{const krb5_keyblock *}{key}
        \funcarg{krb5_const_pointer}{decryptarg}
        \funcarg{krb5_kdc_rep *}{dec_rep}
\funcendfuncarg
\funcarg{krb5_const_pointer}{decryptarg}
\funcinout
\funcarg{krb5_creds *}{creds}
\funcarg{krb5_ccache}{ccache}
\funcarg{krb5_kdc_rep **}{ret_as_reply}
\end{funcdecl}

This all-purpose initial ticket routine, usually called via
\funcname{krb5_get_in_tkt_with_skey} or
\funcname{krb5_get_in_tkt_with_password} or
\funcname{krb5_get_in_tkt_with_keytab}.


Attempts to get an initial ticket for \funcparam{creds{\ptsto}client}
to use server \funcparam{creds{\ptsto}server}, using the following:
the realm from \funcparam{creds{\ptsto}client}; the options in
\funcparam{options} (listed in Table \ref{KDCOptions});
and \funcparam{ptypes}, the preauthentication
method (valid preauthentication methods are listed in Table
\ref{padata-types}).
\funcname{krb5_get_in_tkt} requests encryption type
\funcparam{etypes} (valid encryption types are ETYPE_DES_CBC_CRC and
ETYPE_RAW_DES_CBC),
using \funcparam{creds{\ptsto}times.starttime},
\funcparam{creds{\ptsto}times.endtime},
\funcparam{creds{\ptsto}times.renew_till} 
as from, till, and rtime.  \funcparam{creds{\ptsto}times.renew_till} is
ignored unless the RENEWABLE option is requested.

\funcparam{key_proc} is called, with \funcparam{context}, \funcparam{keytype},
\funcparam{keyseed} and\funcparam{padata} as arguments, to fill in \funcparam{key} to be used for
decryption.  The valid key types for \funcparam{keytype} are
KEYTYPE_NULL,\footnote{See RFC section 6.3.1} and
KEYTYPE_DES.\footnote{See RFC section 6.3.4}  However, KEYTYPE_DES is
the only key type supported by MIT kerberos.
The content of  \funcparam{keyseed} 
depends on the \funcparam{key_proc} being used. %nlg - need a ref here
The \funcparam{padata} passed
to \funcparam{key_proc} is the preauthentication data returned by the
KDC as part of the reply to the initial ticket request.  It may
contain an element of type KRB5_PADATA_PW_SALT, which
\funcparam{key_proc} should use to determine what salt to use when
generating the key.  \funcparam{key_proc} should fill in
\funcparam{key} with a key for the client, or return an error code.

\funcparam{decrypt_proc} is called to perform the decryption of the
response (the encrypted part is in
\funcparam{dec_rep{\ptsto}enc_part}; the decrypted part should be
allocated and filled into
\funcparam{dec_rep{\ptsto}enc_part2}.
\funcparam{decryptarg} is passed on to \funcparam{decrypt_proc}, and
its content depends on the \funcparam{decrypt_proc} being used.

If \funcparam{addrs} is non-NULL, it is used for the addresses
requested.  If it is null, the system standard addresses are used.

If \funcparam{ret_as_reply} is non-NULL, it is filled in with a
pointer to a structure containing the reply packet from the KDC.  Some
programs may find it useful to have direct access to this information.
For example, it can be used to obtain the pre-authentication data
passed back from the KDC.  The caller is responsible for freeing this
structure by using \funcname{krb5_free_kdc_rep}.

If \funcparam{etypes} is non-NULL, the it is used as for the list of
valid encyrption types. Otherwise, the context default is used (as
returned by \funcname{krb5_get_default_in_tkt_etypes}.

A succesful call will place the ticket in the credentials cache
\funcparam{ccache} and fill in \funcparam{creds} with the ticket
information used/returned.

Returns system errors, preauthentication errors, encryption errors.

% XXX Right now, uses creds->addresses before it's copied into from
% the reply -- it's passed to krb5_obtain_padata.  I think that's
% wrong, and it should be using either addrs or the result of
% krb5_os_localaddr instead.  If I'm wrong, then this spec has to be
% updated to document that creds->addresses is used.  On the other
% hand, if I'm right, then the bug in get_in_tkt needs to be fixed.
% See ov-cambridge PR 1525.

\begin{funcdecl}{krb5_get_in_tkt_with_password}{krb5_error_code}{\funcinout}
\funcarg{krb5_context}{context}
\funcin
\funcarg{const krb5_flags}{options}
\funcarg{krb5_address * const *}{addrs}
\funcarg{const krb5_enctype *}{etypes}
\funcarg{const krb5_preauthtype *}{pre_auth_types}
\funcarg{const char *}{password}
\funcarg{krb5_ccache}{ccache}
\funcinout
\funcarg{krb5_creds *}{creds}
\funcarg{krb5_kdc_rep **}{ret_as_reply}
\end{funcdecl}

Attempts to get an initial ticket using the null-terminated string
\funcparam{password}.  If \funcparam{password} is NULL, the password
is read from the terminal using as a prompt the globalname
\globalname{krb5_default_pwd_prompt1}.  

The password is converted into a key using the appropriate
string-to-key conversion function for the specified
\funcparam{keytype}, and using any salt data returned by the KDC in
response to the authentication request.

See \funcname{krb5_get_in_tkt} for documentation of the
\funcparam{options}, \funcparam{addrs}, \funcparam{pre_auth_type},
\funcparam{etype}, \funcparam{keytype}, \funcparam{ccache},
\funcparam{creds} and \funcparam{ret_as_reply} arguments.

Returns system errors, preauthentication errors, encryption errors.

\begin{funcdecl}{krb5_get_in_tkt_with_keytab}{krb5_error_code}{\funcinout}
\funcarg{krb5_context}{context}
\funcin
\funcarg{const krb5_flags}{options}
\funcarg{krb5_address * const *}{addrs}
\funcarg{const krb5_enctype *}{etypes}
\funcarg{const krb5_preauthtype *}{pre_auth_types}
\funcarg{const krb5_keytab *}{keytab}
\funcarg{krb5_ccache}{ccache}
\funcinout
\funcarg{krb5_creds *}{creds}
\funcarg{krb5_kdc_rep **}{ret_as_reply}
\end{funcdecl}

Attempts to get an initial ticket using \funcparam{keytab}.  If
\funcparam{keytab} is NULL, the default keytab is used 
(e.g., \filename{/etc/v5srvtab}).

See \funcname{krb5_get_in_tkt} for documentation of the
\funcparam{options}, \funcparam{addrs}, \funcparam{pre_auth_type},
\funcparam{etype}, \funcparam{ccache}, \funcparam{creds} and
\funcparam{ret_as_reply} arguments.

Returns system errors, preauthentication errors, encryption errors.

\begin{funcdecl}{krb5_get_in_tkt_with_skey}{krb5_error_code}{\funcinout}
\funcarg{krb5_context}{context}
\funcin
\funcarg{const krb5_flags}{options}
\funcarg{krb5_address * const *}{addrs}
\funcarg{const krb5_enctype *}{etypes}
\funcarg{const krb5_preauthtype *}{pre_auth_types}
\funcarg{const krb5_keyblock *}{key}
\funcarg{krb5_ccache}{ccache}
\funcinout
\funcarg{krb5_creds *}{creds}
\funcarg{krb5_kdc_rep **}{ret_as_reply}
\end{funcdecl}

Attempts to get an initial ticket using \funcparam{key}.  If
\funcparam{key} is NULL, an appropriate key is retrieved from the
system key store (e.g., \filename{/etc/v5srvtab}).

See \funcname{krb5_get_in_tkt} for documentation of the
\funcparam{options}, \funcparam{addrs}, \funcparam{pre_auth_type},
\funcparam{etype}, \funcparam{ccache}, \funcparam{creds} and
\funcparam{ret_as_reply} arguments.

Returns system errors, preauthentication errors, encryption errors.

\begin{funcdecl}{krb5_mk_req}{krb5_error_code}{\funcinout}
\funcarg{krb5_context}{context}
\funcarg{krb5_auth_context *}{auth_context}
\funcin
\funcarg{const krb5_flags}{ap_req_options}
\funcarg{char *}{service}
\funcarg{char *}{hostname}
\funcarg{krb5_data *}{in_data}
\funcinout
\funcarg{krb5_ccache}{ccache}
\funcout
\funcarg{krb5_data *}{outbuf}
\end{funcdecl}

Formats a KRB_AP_REQ message into \funcparam{outbuf}.

The server to receive the message is specified by
\funcparam{hostname}. The principal of the server to receive the message
is specified by \funcparam{hostname} and \funcparam{service}.
If credentials are not present in the credentials cache
\funcparam{ccache} for this server, the TGS request with default
parameters is used in an attempt to obtain such credentials, and they
are stored in \funcparam{ccache}.

\funcparam{ap_req_options} specifies the KRB_AP_REQ options desired.
Valid options are:

\begin{tabular}{ll}
AP_OPTS_USE_SESSION_KEY&\\
AP_OPTS_MUTUAL_REQUIRED&\\
\end{tabular}
\label{ap-req-options}

The checksum method to be used is as specified in \funcparam{auth_context}. 

% XXX Not sure if it's legal in the protocol for no checksum to be
% included, or if so, how the server reacts to a request with no
% checksum.

\funcparam{outbuf} should point to an existing \datatype{krb5_data}
structure.  \funcparam{outbuf{\ptsto}length} and
\funcparam{outbuf{\ptsto}data} will be filled in on success, and the latter
should be freed by the caller when it is no longer needed; if an error
is returned, however, no storage is allocated and {\tt
outbuf{\ptsto}data} does not need to be freed.

Returns system errors, error getting credentials for
\funcparam{server}.

\begin{funcdecl}{krb5_mk_req_extended}{krb5_error_code}{\funcinout}
\funcarg{krb5_context}{context}
\funcarg{krb5_auth_context *}{auth_context}
\funcin
\funcarg{const krb5_flags}{ap_req_options}
\funcarg{krb5_data *}{in_data}
\funcarg{krb5_creds *}{in_creds}
\funcout
\funcarg{krb5_data *}{outbuf}
\end{funcdecl}

Formats a KRB_AP_REQ message into \funcparam{outbuf}, with more complete
options than \funcname{krb5_mk_req}.

\funcparam{outbuf}, \funcparam{ap_req_options}, \funcparam{auth_context},
and \funcparam{ccache} are used in the same fashion as for
\funcname{krb5_mk_req}.

\funcparam{in_creds} is used to supply the credentials (ticket and session
key) needed to form the request.

If \funcparam{in_creds{\ptsto}ticket} has no data (length == 0), then an
error is returned. 

During this call, the structure elements in \funcparam{in_creds} may be
freed and reallocated.  Hence all of the structure elements which are
pointers should point to allocated memory, and there should be no other
pointers aliased to the same memory, since it may be deallocated during
this procedure call.


If \funcparam{ap_req_options} specifies AP_OPTS_USE_SUBKEY, then a
subkey will be generated if need be by \funcname{krb5_generate_subkey}.

A copy of the authenticator will be stored in the
\funcparam{auth_context}, with the principal and checksum fields nulled
out, unless an error is returned.  (This is to prevent pointer sharing
problems; the caller shouldn't need these fields anyway, since the
caller supplied them.)

Returns system errors, errors contacting the KDC, KDC errors getting
a new ticket for the authenticator.

\begin{funcdecl}{krb5_generate_subkey}{krb5_error_code}{\funcinout}
\funcarg{krb5_context}{context}
\funcin
\funcarg{const krb5_keyblock *}{key}
\funcout
\funcarg{krb5_keyblock **}{subkey}
\end{funcdecl}

Generates a pseudo-random sub-session key using the encryption system's
random key functions, based on the input \funcparam{key}.

\funcparam{subkey} is filled in to point to the generated subkey, unless
an error is returned.  The returned key (i.e., \funcparam{*subkey}) is
allocated and should be freed by the caller with
\funcname{krb5_free_keyblock} when it is no longer needed.

\begin{funcdecl}{krb5_rd_req}{krb5_error_code}{\funcinout}
\funcarg{krb5_context}{context}
\funcarg{krb5_auth_context *}{auth_context}
\funcin
\funcarg{const krb5_data *}{inbuf}
\funcarg{krb5_const_principal}{server}
\funcarg{krb5_keytab}{keytab}
\funcinout
\funcarg{krb5_flags *}{ap_req_options}
\funcout
\funcarg{krb5_ticket **}{ticket}
\end{funcdecl}

Parses a KRB_AP_REQ message, returning its contents.  Upon successful
return, if \funcparam{ticket} is non-NULL, \funcparam{*ticket} will be
modified to point to allocated storage containing the ticket
information.  The caller is responsible for deallocating this space by
using \funcname{krb5_free_ticket}.

\funcparam{inbuf} should contain the KRB_AP_REQ message to be parsed.

If \funcparam{auth_context} is NULL, one will be generated and freed
internally by the function.

\funcparam{server} specifies the expected server's name for the ticket.
If \funcparam{server} is NULL, then any server name will be accepted if
the appropriate key can be found, and the caller should verify that the
server principal matches some trust criterion.

If \funcparam{server} is not NULL, and a replay detaction cache has not been
established with the \funcparam{auth_context}, one will be generated. 

\funcparam{keytab} specifies a keytab containing generate a decryption key. If
NULL, \funcparam{krb5_kt_default} will be used to find the default 
keytab and the key taken from there\footnote{i.e., srvtab file in
Kerberos V4 parlance}.


If a keyblock is present in the \funcparam{auth_context}, it will be
used to decrypt the ticket request and the keyblock freed with
\funcname{krb5_free_keyblock}. This is useful for user to user
authentication. If no keyblock is specified, the \funcparam{keytab} is
consulted for an entry matching the requested keytype, server and
version number and used instead.

The authentcator in the request is decrypted and stored in the
\funcparam{auth_context}. The client specified in the decrypted
authenticator is compared to the client specified in the decoded ticket
to ensure that the compare.

If the remote_addr portion of the \funcparam{auth_context} is set, 
then this routine checks if the request came from the right client.

\funcparam{sender_addr} specifies the address(es) expected to be present
in the ticket.

The replay cache is checked to see if the ticket and authenticator have
been seen and if so, returns an error. If not, the ticket and
authenticator are entered into the cache.

Various other checks are made of the decoded data, including,
cross-realm policy, clockskew and ticket validation times.

The keyblock, subkey, and sequence number of the request are all stored
in the \funcparam{auth_context} for future use.

If the request has the AP_OPTS_MUTUAL_REQUIRED bit set, the local
sequence number, which is stored in the auth_context, is XORed with the
remote sequence number in the request.

If \funcparam{ap_req_options} is non-NULL, it will be set to contain the
application request flags.

Returns system errors, encryption errors, replay errors.

\begin{funcdecl}{krb5_rd_req_decoded}{krb5_error_code}{\funcinout}
\funcarg{krb5_context}{context}
\funcarg{krb5_auth_context *}{auth_context}
\funcin
\funcarg{const krb5_ap_req *}{req}
\funcarg{krb5_const_principal}{server}
\funcinout
\funcarg{krb5_keytab}{keytab}
\funcout
\funcarg{krb5_ticket **}{ticket}
\end{funcdecl}

Essentially the same as \funcname{krb5_rd_req}, but uses a decoded AP_REQ
as the input rather than an encoded input.

\begin{funcdecl}{krb5_mk_rep}{krb5_error_code}{\funcinout}
\funcarg{krb5_context}{context}
\funcarg{krb5_auth_context}{auth_context}
\funcout
\funcarg{krb5_data *}{outbuf}
\end{funcdecl}

Formats and encrypts an AP_REP message, including in it the data in the
authentp portion of \funcparam{auth_context}, encrypted using the
keyblock portion of \funcparam{auth_context}. 

When successful, \funcparam{outbuf{\ptsto}length} and
\funcparam{outbuf{\ptsto}data} are filled in with the length of the
AP_REQ message and allocated data holding it.
\funcparam{outbuf{\ptsto}data} should be freed by the
caller when it is no longer needed. 

If the flags in \funcparam{auth_context} indicate that a sequence number
should be used (either {\sc KRB5_AUTH_CONTEXT_DO_SEQUENCE} or
{\sc KRB5_AUTH_CONTEXT_RET_SEQUENCE}) and the local sequqnce number in the
\funcparam{auth_context} is 0, a new number will be generated with
\funcname{krb5_generate_seq_number}.

Returns system errors.

\begin{funcdecl}{krb5_rd_rep}{krb5_error_code}{\funcinout}
\funcarg{krb5_context}{context}
\funcarg{krb5_auth_context}{auth_context}
\funcin
\funcarg{const krb5_data *}{inbuf}
\funcout
\funcarg{krb5_ap_rep_enc_part **}{repl}
\end{funcdecl}

Parses and decrypts an AP_REP message from \funcparam{*inbuf}, filling in
\funcparam{*repl} with a pointer to  allocated storage containing the
values from the message.  The caller is responsible for freeing this
structure with \funcname{krb5_free_ap_rep_enc_part}.

The keyblock stored in \funcparam{auth_context} is used to decrypt the
message after establishing any key pre-processing with
\funcname{krb5_process_key}. 

Returns system errors, encryption errors, replay errors.

\begin{funcdecl}{krb5_mk_error}{krb5_error_code}{\funcinout}
\funcarg{krb5_context}{context}
\funcin
\funcarg{const krb5_error *}{dec_err}
\funcout
\funcarg{krb5_data *}{enc_err}
\end{funcdecl}

Formats the error structure \funcparam{*dec_err} into an error buffer
\funcparam{*enc_err}.

The error buffer storage (\funcparam{enc_err{\ptsto}data}) is
allocated, and should be freed by the caller when finished.

Returns system errors.

\begin{funcdecl}{krb5_rd_error}{krb5_error_code}{\funcinout}
\funcarg{krb5_context}{context}
\funcin
\funcarg{const krb5_data *}{enc_errbuf}
\funcout
\funcarg{krb5_error **}{dec_error}
\end{funcdecl}

Parses an error protocol message from \funcparam{enc_errbuf} and fills in 
\funcparam{*dec_error} with a pointer to allocated storage containing
the error message.  The caller is reponsible for freeing this structure by
using \funcname{krb5_free_error}.

Returns system errors.

\begin{funcdecl}{krb5_generate_seq_number}{krb5_error_code}{\funcinout}
\funcarg{krb5_context}{context}
\funcin
\funcarg{const krb5_keyblock *}{key}
\funcout
\funcarg{krb5_int32 *}{seqno}
\end{funcdecl}

Generates a pseudo-random sequence number suitable for use as an initial
sequence number for the KRB_SAFE and KRB_PRIV message processing
routines.

\funcparam{key} parameterizes the choice of the random sequence number,
which is filled into \funcparam{*seqno} upon return.

\begin{funcdecl}{krb5_sendauth}{krb5_error_code}
\funcinout
\funcarg{krb5_context}{context}
\funcarg{krb5_auth_context *}{auth_context}
\funcin
\funcarg{krb5_pointer}{fd}
\funcarg{char *}{appl_version}
\funcarg{krb5_principal}{client}
\funcarg{krb5_principal}{server}
\funcarg{krb5_flags}{ap_req_options}
\funcarg{krb5_data *}{in_data}
\funcarg{krb5_creds *}{in_creds}
\funcinout
\funcarg{krb5_ccache}{ccache}
\funcout
\funcarg{krb5_error **}{error}
\funcarg{krb5_ap_rep_enc_part **}{rep_result}
\funcarg{krb5_creds **}{out_creds}
\end{funcdecl}

\funcname{krb5_sendauth} provides a convenient means for client and
server programs to send authenticated messages to one another through
network connections.  \funcname{krb5_sendauth} sends an authenticated
ticket from the client program to the server program using the network
connection specified by \funcparam{fd}.  In the MIT Unix implementation,
\funcparam{fd} should be a pointer to a file descriptor describing the
network socket.  This can be changed in other implementations, however,
if the routines \funcname{krb5_read_message},
\funcname{krb5_write_message}, \funcname{krb5_net_read}, and
\funcname{krb5_net_write} are changed.

The paramter \funcparam{appl_version} is a string describing the
application protocol version which the client is expecting to use for
this exchange.  If the server is using a different application protocol,
an error will be returned.

The parameters \funcparam{client} and \funcparam{server} specify the
kerberos principals for the client and the server.  They are
ignored if \funcparam{in_creds} is non-null.  Otherwise,
\funcparam{server} must be non-null, but \funcparam{client} may be
null, in which case the client principal used is the one in the
credential cache's default principal.

The \funcparam{ap_req_options} parameters specifies the options which
should be passed to \funcname{krb5_mk_req}.  Valid options are listed
in Table \ref{ap-req-options}.  If \funcparam{ap_req_options}
specifies MUTUAL_REQUIRED, then \funcname{krb5_sendauth} will perform
a mutual authentication exchange, and if \funcparam{rep_result} is
non-null, it will be filled in with the result of the mutual
authentication exchange; the caller should free
\funcparam{*rep_result} with
\funcname{krb5_free_ap_rep_enc_part} when done with it.

If \funcparam{in_creds} is non-null, then
\funcparam{in_creds{\ptsto}client} and 
\funcparam{in_creds{\ptsto}server} must be filled in, and either
the other structure fields should be filled in with valid credentials,
or \funcparam{in_creds{\ptsto}ticket.length} should be zero.  If
\funcparam{in_creds{\ptsto}ticket.length} is non-zero, then
\funcparam{in_creds} will be used as-is as the credentials to send to
the server, and \funcparam{ccache} is ignored; otherwise,
\funcparam{ccache} is used as described below, and \funcparam{out_creds}
, if not NULL, is filled in with the retrieved credentials.

\funcname{ccache} specifies the credential cache to use when one is
needed (i.e., when \funcname{in_creds} is null or
\funcparam{in_creds{\ptsto}ticket.length} is zero).  When a credential
cache is not needed, \funcname{ccache} is ignored.  When a credential
cache is needed and \funcname{ccache} is null, the default credential
cache is used.  Note that if the credential cache is needed and does
not contain the needed credentials, they will be retrieved from the
KDC and stored in the credential cache.

If mutual authentication is used and \funcparam{rep_result} is non-null,
the sequence number for the server is available to the caller in
\funcparam{*rep_result->seq_number}.  (If mutual authentication is not
used, there is no way to negotiate a sequence number for the server.)

If an error occurs during the authenticated ticket exchange and
\funcparam{error} is non-null, the error packet (if any) that was sent
from the server will be placed in it.  This error should be freed with
\funcname{krb5_free_error}.

\begin{funcdecl}{krb5_recvauth}{krb5_error_code}{\funcinout}
\funcarg{krb5_context}{context}
\funcarg{krb5_auth_context *}{auth_context}
\funcin
\funcarg{krb5_pointer}{fd}
\funcarg{char *}{appl_version}
\funcarg{krb5_principal}{server}
\funcarg{char *}{rc_type}
\funcarg{krb5_int32}{flags}
\funcarg{krb5_keytab}{keytab}
\funcout
\funcarg{krb5_ticket **}{ticket}
\end{funcdecl}

\funcname{krb5_recvauth} provides a convenient means for client and
server programs to send authenticated messages to one another through
network connections.  \funcname{krb5_sendauth} is the matching routine
to \funcname{krb5_recvauth} for the server.  \funcname{krb5_recvauth}
will engage in an authentication dialogue with the
client program running \funcname{krb5_sendauth} to authenticate the
client to the server.  In addition, if requested by the client,
\funcname{krb5_recvauth} will provide mutual authentication to
prove to the client that the server represented by
\funcname{krb5_recvauth} is legitimate. 

\funcparam{fd} is a pointer to the network connection.  As in
\funcname{krb5_sendauth}, in the MIT Unix implementation
\funcparam{fd} is a pointer to a file descriptor.

The parameter \funcparam{appl_version} is a string describing the
application protocol version which the server is expecting to use for
this exchange.  If the client is using a different application protocol,
an error will be returned and the authentication exchange will be
aborted.

If \funcparam{server} is non-null, then \funcname{krb5_recvauth}
verifies that the server principal requested by the client matches
\funcparam{server}.  If not, an error will be returned and the
authentication exchange will be aborted.

The parameters \funcparam{server}, \funcparam{auth_context},
and \funcparam{keytab} are used by \funcname{krb5_rd_req} to obtain the
server's private key.

If \funcparam{server} is non-null, the princicpal component of it is
ysed to determine the replay cache to use. Otherwise,
\funcname{krb5_recvauth} will use a default replay cache.

The \funcparam{flags} argument allows the caller to modify the behavior of
\funcname{krb5_recvauth}.  For non-library callers, \funcparam{flags}
should be 0. 

% XXX Note that if the bug I submitted entitled ``"flags" argument
% should not have been added to krb5_recvauth'' (OpenVision Cambridge
% bug number 1585) causes code changes, this will have to be updated.

\funcparam{ticket} is optional and is only filled in if non-null. It is
filled with the data from the ticket sent by the client, and should be
freed with
\funcname{krb5_free_ticket} when it is no longer needed.

\begin{funcdecl}{krb5_mk_safe}{krb5_error_code}{\funcinout}
\funcarg{krb5_context}{context}
\funcarg{krb5_auth_context}{auth_context}
\funcin
\funcarg{const krb5_data *}{userdata}
\funcout
\funcarg{krb5_data *}{outbuf}
\funcinout
\funcarg{krb5_replay_data *}{outdata}
\end{funcdecl}

Formats a KRB_SAFE message into \funcparam{outbuf}.

\funcparam{userdata} is formatted as the user data in the message.
Portions of \funcparam{auth_context} specify the checksum type; the
keyblockm which might be used to seed the checksum;
full addresses (host and port) for the sender and receiver.
The \funcparam{local_addr} portion of \funcparam{*auth_context} 
is used to form the addresses usedin the KRB_SAFE message. The  \funcparam{remote_addr} is optional; if the
receiver's address is not known, it may be replaced by NULL.
\funcparam{local_addr}, however, is mandatory.

The \funcparam{auth_context} flags select whether sequence numbers or
timestamps should be used to identify the message.  Valid flags are
listed below.

\begin{tabular}{ll}
\multicolumn{1}{c}{Symbol} & Meaning \\
KRB5_AUTH_CONTEXT_DO_TIME               & Use timestamps\\
        &\  and replay cache\\
KRB5_AUTH_CONTEXT_RET_TIME      & Copy timestamp \\
        &\ to \funcparam{*outdata} \\
KRB5_AUTH_CONTEXT_DO_SEQUENCE   & Use sequence numbers \\
KRB5_AUTH_CONTEXT_RET_SEQUENCE  & Copy sequence numbers\\
        &\ to \funcparam{*outdata} \\
\end{tabular}

If timestamps are to be used (i.e., if KRB5_AUTH_CONTEXT_DO_TIME is
set), an entry describing the message will be entered in the replay
cache so that the caller may detect if this message is sent
back to him by an attacker.  If KRB5_AUTH_CONTEXT_DO_TIME is not set,
the \funcparam{auth_context} replay cache is not used.

If sequence numbers are to be used (i.e., if either
KRB5_AUTH_CONTEXT_DO_SEQUENCE or KRB5_AUTH_CONTEXT_RET_SEQUENEC is
set), then \funcparam{auth_context} local sequence number will be
placed in the protected message as its sequence number.  

The \funcparam{outbuf} buffer storage (i.e.,
\funcparam{outbuf{\ptsto}data}) is allocated, and should be freed by
the caller when finished.

Returns system errors, encryption errors.

\begin{funcdecl}{krb5_rd_safe}{krb5_error_code}{\funcinout}
\funcarg{krb5_context}{context}
\funcarg{krb5_auth_context}{auth_context}
\funcin
\funcarg{const krb5_data *}{inbuf}
\funcout
\funcarg{krb5_data *}{outbuf}
\funcinout
\funcarg{krb5_replay_data *}{outdata}
\end{funcdecl}

Parses a KRB_SAFE message from \funcparam{inbuf}, placing the
data in \funcparam{*outbuf} after verifying its integrity.

The keyblock used for verifying the integrity of the message is taken
from the \funcparam{auth_context} recv\_subkey or keyblock. The
keyblock is chosen in the above order by the first one which is not
NULL.
 
The remote_addr and localaddr portions of the \funcparam{*auth_context}
specify the full addresses (host and port) of the sender and receiver,
and must be of type \datatype{ADDRTYPE_ADDRPORT}.


The \funcparam{remote_addr} parameter is mandatory; it
specifies the address of the sender.  If the address of the sender in
the message does not match \funcparam{remote_addr}, the error
KRB5KRB_AP_ERR_BADADDR will be returned.

If \funcparam{local_addr} is non-NULL, then the address of the receiver
in the message much match it.  If it is null, the receiver address in
the message will be checked against the list of local addresses as
returned by \funcname{krb5_os_localaddr}. If the check fails,
KRB5KRB_AP_ERR_BADARRD is returned.

The \funcparam{outbuf} buffer storage (i.e.,
\funcparam{outbuf{\ptsto}data} is allocated storage which the caller
should free when it is no longer needed.

If auth_context_flags portion of \funcparam{auth_context} indicates
that sequence numbers are to be used (i.e., if KRB5_AUTH_CONTEXT_DOSEQUENCE is
set in it), The \funcparam{remote_seq_number} portion of
\funcparam{auth_context} is compared to the sequence number for the
message, and KRB5_KRB_AP_ERR_BADORDER is returned if it does not match.
Otherwise, the sequence number is not used.


If timestamps are to be used (i.e., if KRB5_AUTH_CONTEXT_DO_TIME is set
in the \funcparam{auth_context}), then two additional checks are performed:
\begin{itemize}
\item The timestamp in the message must be within the permitted clock
        skew (which is usually five minutes), or KRB5KRB_AP_ERR_SKEW
        is returned.
\item The message must not be a replayed message, according to
        \funcparam{rcache}.
\end{itemize}

Returns system errors, integrity errors.

\begin{funcdecl}{krb5_mk_priv}{krb5_error_code}{\funcinout}
\funcarg{krb5_context}{context}
\funcarg{krb5_auth_context}{auth_context}
\funcin
\funcarg{const krb5_data *}{userdata}
\funcout
\funcarg{krb5_data *}{outbuf}
\funcarg{krb5_replay_data *}{outdata}
\end{funcdecl}

Formats a KRB_PRIV message into \funcparam{outbuf}.  Behaves similarly
to \funcname{krb5_mk_safe}, but the message is encrypted and
integrity-protected rather than just integrity-protected.

\funcparam{inbuf}, \funcparam{auth_context}, 
\funcparam{outdata} and
\funcparam{outbuf} function as in \funcname{krb5_mk_safe}.

As in \funcname{krb5_mk_safe}, the remote_addr and remote_port part of
the \funcparam{auth_context} is optional; if the receiver's address is
not known, it may be replaced by NULL.  The local_addr, however, is
mandatory.

The encryption type is taken from the \funcparam{auth_context} keyblock
portion. If i_vector portion of the \funcparam{auth_context}
is non-null, it is used as an initialization vector for the encryption
(if the chosen encryption type supports initialization vectors)
and its contents are replaced with the last block of encrypted data
upon return.

The flags from the \funcparam{auth_context} selects whether sequence numbers or timestamps
should be used to identify the message.  Valid flags are listed below.

\begin{tabular}{ll}
\multicolumn{1}{c}{Symbol} & Meaning \\
KRB5_AUTH_CONTEXT_DO_TIME& Use timestamps in replay cache\\
KRB5_AUTH_CONTEXT_RET_TIME& Use timestamps in output data\\
KRB5_AUTH_CONTEXT_DO_SEQUENCE& Use sequence numbers\\
        &\ in replay cache\\
KRB5_AUTH_CONTEXT_RET_SEQUENCE& Use sequence numbers\\
        &\ in replay cache and output data \\
\end{tabular}

Returns system errors, encryption errors.

\begin{funcdecl}{krb5_rd_priv}{krb5_error_code}{\funcinout}
\funcarg{krb5_context}{context}
\funcarg{krb5_auth_context}{auth_context}
\funcin
\funcarg{const krb5_data *}{inbuf}
\funcout
\funcarg{krb5_data *}{outbuf}
\funcarg{krb5_data *}{outdata}
\end{funcdecl}

Parses a KRB_PRIV message from \funcparam{inbuf}, placing the data in
\funcparam{*outbuf} after decrypting it.  Behaves similarly to
\funcname{krb5_rd_safe}, but the message is decrypted rather than
integrity-checked.

\funcparam{inbuf}, \funcparam{auth_context}, 
\funcparam{outdata} and \funcparam{outbuf}
function as in \funcname{krb5_rd_safe}.


The remote_addr part of the \funcparam{auth_context} as set by
\funcname{krb5_auth_con_setaddrs} is mandatory;  it
specifies the address of the sender.  If the address of the sender in
the message does not match the remote_addr, the error
KRB5KRB_AP_ERR_BADADDR will be returned.

If local_addr portion of the auth_context is non-NULL, then the address
of the receiver in the message much match it.  If it is null, the
receiver address in the message will be checked against the list of
local addresses as returned by \funcname{krb5_os_localaddr}.

The \funcparam{keyblock} portion of \funcparam{auth_context} specifies
the key to be used for decryption of the message.  If the
\funcparam{i_vector} element, is non-null, it is used as an
initialization vector for the decryption (if the encryption type of the
message supports initialization vectors) and its contents are replaced
with the last block of encrypted data in the message.

The \funcparam{auth_context} flags specify whether timestamps
(KRB5_AUTH_CONTEXT_DO_TIME) and sequence numbers
(KRB5_AUTH_CONTEXT_DO_SEQUENCE) are to be used.

Returns system errors, integrity errors.

\subsubsection{Miscellaneous main functions}

\begin{funcdecl}{krb5_address_search}{krb5_boolean}{\funcinout}
\funcarg{krb5_context}{context}
\funcin
\funcarg{const krb5_address *}{addr}
\funcarg{krb5_address * const *}{addrlist}
\end{funcdecl}

If \funcparam{addr} is listed in \funcparam{addrlist}, or
\funcparam{addrlist} is null, return TRUE.  If not listed, return FALSE.

\begin{funcdecl}{krb5_address_compare}{krb5_boolean}{\funcinout}
\funcarg{krb5_context}{context}
\funcin
\funcarg{const krb5_address *}{addr1}
\funcarg{const krb5_address *}{addr2}
\end{funcdecl}

If the two addresses are the same, return TRUE, else return FALSE.

\begin{funcdecl}{krb5_fulladdr_order}{int}{\funcinout}
\funcarg{krb5_context}{context}
\funcin
\funcarg{const krb5_fulladdr *}{addr1}
\funcarg{const krb5_fulladdr *}{addr2}
\end{funcdecl}

Return an ordering on the two full addresses:  0 if the same,
$< 0$ if first is less than 2nd, $> 0$ if first is greater than 2nd.

\begin{funcdecl}{krb5_address_order}{int}{\funcinout}
\funcarg{krb5_context}{context}
\funcin
\funcarg{const krb5_address *}{addr1}
\funcarg{const krb5_address *}{addr2}
\end{funcdecl}

Return an ordering on the two addresses:  0 if the same,
$< 0$ if first is less than 2nd, $> 0$ if first is greater than 2nd.

\begin{funcdecl}{krb5_copy_addresses}{krb5_error_code}{\funcinout}
\funcarg{krb5_context}{context}
\funcin
\funcarg{krb5_address * const *}{inaddr}
\funcout
\funcarg{krb5_address ***}{outaddr}
\end{funcdecl}

Copy addresses in \funcparam{inaddr} to \funcparam{*outaddr} which is
allocated memory and should be freed with \funcname{krb5_free_addresses}.

\begin{funcdecl}{krb5_copy_authdata}{krb5_error_code}{\funcinout}
\funcarg{krb5_context}{context}
\funcin
\funcarg{krb5_authdata * const *}{inauthdat}
\funcout
\funcarg{krb5_authdata ***}{outauthdat}
\end{funcdecl}

Copy an authdata structure, filling in \funcparam{*outauthdat} to point to the
newly allocated copy, which should be freed with 
\funcname{krb5_free_authdata}.

\begin{funcdecl}{krb5_copy_authenticator}{krb5_error_code}{\funcinout}
\funcarg{krb5_context}{context}
\funcin
\funcarg{const krb5_authenticator *}{authfrom}
\funcout
\funcarg{krb5_authenticator **}{authto}
\end{funcdecl}

Copy an authenticator structure, filling in \funcparam{*outauthdat} to
point to the newly allocated copy, which should be freed with 
\funcname{krb5_free_authenticator}.

\begin{funcdecl}{krb5_copy_keyblock}{krb5_error_code}{\funcinout}
\funcarg{krb5_context}{context}
\funcin
\funcarg{const krb5_keyblock *}{from}
\funcout
\funcarg{krb5_keyblock **}{to}
\end{funcdecl}

Copy a keyblock, filling in \funcparam{*to} to point to the newly
allocated copy, which should be freed with
\funcname{krb5_free_keyblock}. 

\begin{funcdecl}{krb5_copy_keyblock_contents}{krb5_error_code}{\funcinout}
\funcarg{krb5_context}{context}
\funcin
\funcarg{const krb5_keyblock *}{from}
\funcout
\funcarg{krb5_keyblock *}{to}
\end{funcdecl}

Copy keyblock contents from \funcparam{from} to \funcparam{to}, including
allocated storage.  The allocated storage in \funcparam{to} should be
freed by using {\bf free}(\funcparam{to->contents}).

\begin{funcdecl}{krb5_copy_checksum}{krb5_error_code}{\funcinout}
\funcarg{krb5_context}{context}
\funcin
\funcarg{const krb5_checksum *}{ckfrom}
\funcout
\funcarg{krb5_checksum **}{ckto}
\end{funcdecl}

Copy a checksum structure, filling in \funcparam{*ckto} to point to
the newly allocated copy, which should be freed with
\funcname{krb5_free_checksum}.

\begin{funcdecl}{krb5_copy_creds}{krb5_error_code}{\funcinout}
\funcarg{krb5_context}{context}
\funcin
\funcarg{const krb5_creds *}{incred}
\funcout
\funcarg{krb5_creds **}{outcred}
\end{funcdecl}

Copy a credentials structure, filling in \funcparam{*outcred} to point
to the newly allocated copy, which should be freed with
\funcname{krb5_free_creds}.

\begin{funcdecl}{krb5_copy_data}{krb5_error_code}{\funcinout}
\funcarg{krb5_context}{context}
\funcin
\funcarg{const krb5_data *}{indata}
\funcout
\funcarg{krb5_data **}{outdata}
\end{funcdecl}

Copy a data structure, filling in \funcparam{*outdata} to point to the
newly allocated copy, which should be freed with \funcname{krb5_free_data}.

\begin{funcdecl}{krb5_copy_ticket}{krb5_error_code}{\funcinout}
\funcarg{krb5_context}{context}
\funcin
\funcarg{const krb5_ticket *}{from}
\funcout
\funcarg{krb5_ticket **}{pto}
\end{funcdecl}

Copy a ticket structure, filling in \funcparam{*pto} to point to
the newly allocated copy, which should be freed with
\funcname{krb5_free_ticket}.


\begin{funcdecl}{krb5_get_server_rcache}{krb5_error_code}{\funcinout}
\funcarg{krb5_context}{context}
\funcin
\funcarg{const krb5_data *}{piece}
\funcout
\funcarg{krb5_rcache *}{ret_rcache}
\end{funcdecl}

Generate a replay cache name, allocate space for its handle, and open
it.  \funcparam{piece} is used to distinguish this replay cache from
others currently in use on the system.  Typically, \funcparam{piece}
is the first component of the principal name for the client or server
which is calling \funcname{krb5_get_server_rcache}.

Upon successful return, \funcparam{ret_rcache} is filled in to contain a
handle to an open rcache, which should be closed with
\funcname{krb5_rc_close}.




\subsection{Credentials cache functions}
The credentials cache functions (some of which are macros which call to
specific types of credentials caches) deal with storing credentials
(tickets, session keys, and other identifying information) in a
semi-permanent store for later use by different programs.

\begin{funcdecl}{krb5_cc_resolve}{krb5_error_code}{\funcinout}
\funcarg{krb5_context}{context}
\funcin
\funcarg{char *}{string_name}
\funcout
\funcarg{krb5_ccache *}{id}
\end{funcdecl}

Fills in \funcparam{id} with a ccache identifier which corresponds to
the name in \funcparam{string_name}.  

Requires that \funcparam{string_name} be of the form ``type:residual'' and
``type'' is a type known to the library.

\begin{funcdecl}{krb5_cc_gen_new}{krb5_error_code}{\funcinout}
\funcarg{krb5_context}{context}
\funcin
\funcarg{krb5_cc_ops *}{ops}
\funcout
\funcarg{krb5_ccache *}{id}
\end{funcdecl}


Fills in \funcparam{id} with a unique ccache identifier of a type defined by
\funcparam{ops}.  The cache is left unopened.

\begin{funcdecl}{krb5_cc_register}{krb5_error_code}{\funcinout}
\funcarg{krb5_context}{context}
\funcin
\funcarg{krb5_cc_ops *}{ops}
\funcarg{krb5_boolean}{override}
\end{funcdecl}

Adds a new cache type identified and implemented by \funcparam{ops} to
the set recognized by \funcname{krb5_cc_resolve}.
If \funcparam{override} is FALSE, a ticket cache type named
\funcparam{ops{\ptsto}prefix} must not be known.

\begin{funcdecl}{krb5_cc_get_name}{char *}{\funcinout}
\funcarg{krb5_context}{context}
\funcin
\funcarg{krb5_ccache}{id}
\end{funcdecl}

Returns the name of the ccache denoted by \funcparam{id}.

\begin{funcdecl}{krb5_cc_default_name}{char *}{\funcinout}
\funcarg{krb5_context}{context}
\end{funcdecl}

Returns the name of the default credentials cache; this may be equivalent to
\funcnamenoparens{getenv}({\tt "KRB5CCACHE"}) with an appropriate fallback.

\begin{funcdecl}{krb5_cc_default}{krb5_error_code}{\funcinout}
\funcarg{krb5_context}{context}
\funcout
\funcarg{krb5_ccache *}{ccache}
\end{funcdecl}

Equivalent to
\funcnamenoparens{krb5_cc_resolve}(\funcparam{context},
\funcname{krb5_cc_default_name},
\funcparam{ccache}).

\begin{funcdecl}{krb5_cc_initialize}{krb5_error_code}{\funcinout}
\funcarg{krb5_context}{context}
\funcarg{krb5_ccache}{id}
\funcin
\funcarg{krb5_principal}{primary_principal}
\end{funcdecl}

Creates/refreshes a credentials cache identified by \funcparam{id} with
primary principal set to \funcparam{primary_principal}.
If the credentials cache already exists, its contents are destroyed.

Errors:  permission errors, system errors.

Modifies: cache identified by \funcparam{id}.

\begin{funcdecl}{krb5_cc_destroy}{krb5_error_code}{\funcinout}
\funcarg{krb5_context}{context}
\funcarg{krb5_ccache}{id}
\end{funcdecl}

Destroys the credentials cache identified by \funcparam{id}, invalidates
\funcparam{id}, and releases any other resources acquired during use of
the credentials cache.  Requires that \funcparam{id} identifies a valid
credentials cache.  After return, \funcparam{id} must not be used unless
it is first reinitialized using \funcname{krb5_cc_resolve} or
\funcname{krb5_cc_gen_new}.

Errors:  permission errors.

\begin{funcdecl}{krb5_cc_close}{krb5_error_code}{\funcinout}
\funcarg{krb5_context}{context}
\funcarg{krb5_ccache}{id}
\end{funcdecl}

Closes the credentials cache \funcparam{id}, invalidates
\funcparam{id}, and releases \funcparam{id} and any other resources
acquired during use of the credentials cache.  Requires that
\funcparam{id} identifies a valid credentials cache.  After return,
\funcparam{id} must not be used unless it is first reinitialized using
\funcname{krb5_cc_resolve} or \funcname{krb5_cc_gen_new}.


\begin{funcdecl}{krb5_cc_store_cred}{krb5_error_code}{\funcinout}
\funcarg{krb5_context}{context}
\funcin
\funcarg{krb5_ccache}{id}
\funcarg{krb5_creds *}{creds}
\end{funcdecl}

Stores \funcparam{creds} in the cache \funcparam{id}, tagged with
\funcparam{creds{\ptsto}client}.
Requires that \funcparam{id} identifies a valid credentials cache.

Errors: permission errors, storage failure errors.

\begin{funcdecl}{krb5_cc_retrieve_cred}{krb5_error_code}{\funcinout}
\funcarg{krb5_context}{context}
\funcin
\funcarg{krb5_ccache}{id}
\funcarg{krb5_flags}{whichfields}
\funcarg{krb5_creds *}{mcreds}
\funcout
\funcarg{krb5_creds *}{creds}
\end{funcdecl}

Searches the cache \funcparam{id} for credentials matching
\funcparam{mcreds}.  The fields which are to be matched are specified by
set bits in \funcparam{whichfields}, and always include the principal
name \funcparam{mcreds{\ptsto}server}.
Requires that \funcparam{id} identifies a valid credentials cache.

If at least one match is found, one of the matching credentials is
returned in \funcparam{*creds}. The credentials should be freed using
\funcname{krb5_free_credentials}.

Errors: error code if no matches found.

\begin{funcdecl}{krb5_cc_get_principal}{krb5_error_code}{\funcinout}
\funcarg{krb5_context}{context}
\funcin
\funcarg{krb5_ccache}{id}
\funcarg{krb5_principal *}{principal}
\end{funcdecl}

Retrieves the primary principal of the credentials cache (as
set by the \funcname{krb5_cc_initialize} request)
The primary principal is filled into \funcparam{*principal}; the caller
should release this memory by calling \funcname{krb5_free_principal} on
\funcparam{*principal} when finished.

Requires that \funcparam{id} identifies a valid credentials cache.

\begin{funcdecl}{krb5_cc_start_seq_get}{krb5_error_code}{\funcinout}
\funcarg{krb5_context}{context}
\funcarg{krb5_ccache}{id}
\funcout
\funcarg{krb5_cc_cursor *}{cursor}
\end{funcdecl}

Prepares to sequentially read every set of cached credentials.
\funcparam{cursor} is filled in with a cursor to be used in calls to
\funcname{krb5_cc_next_cred}.

\begin{funcdecl}{krb5_cc_next_cred}{krb5_error_code}{\funcinout}
\funcarg{krb5_context}{context}
\funcarg{krb5_ccache}{id}
\funcout
\funcarg{krb5_creds *}{creds}
\funcinout
\funcarg{krb5_cc_cursor *}{cursor}
\end{funcdecl}

Fetches the next entry from \funcparam{id}, returning its values in
\funcparam{*creds}, and updates \funcparam{*cursor} for the next request.
Requires that \funcparam{id} identifies a valid credentials cache and
\funcparam{*cursor} be a cursor returned by
\funcname{krb5_cc_start_seq_get} or a subsequent call to
\funcname{krb5_cc_next_cred}.

Errors: error code if no more cache entries.

\begin{funcdecl}{krb5_cc_end_seq_get}{krb5_error_code}{\funcinout}
\funcarg{krb5_context}{context}
\funcarg{krb5_ccache}{id}
\funcarg{krb5_cc_cursor *}{cursor}
\end{funcdecl}

Finishes sequential processing mode and invalidates \funcparam{*cursor}.
\funcparam{*cursor} must never be re-used after this call.

Requires that \funcparam{id} identifies a valid credentials cache and
\funcparam{*cursor} be a cursor returned by
\funcname{krb5_cc_start_seq_get} or a subsequent call to
\funcname{krb5_cc_next_cred}.

Errors: may return error code if \funcparam{*cursor} is invalid.


\begin{funcdecl}{krb5_cc_remove_cred}{krb5_error_code}{\funcinout}
\funcarg{krb5_context}{context}
\funcin
\funcarg{krb5_ccache}{id}
\funcarg{krb5_flags}{which}
\funcarg{krb5_creds *}{cred}
\end{funcdecl}

Removes any credentials from \funcparam{id} which match the principal
name {cred{\ptsto}server} and the fields in \funcparam{cred} masked by
\funcparam{which}.
Requires that \funcparam{id} identifies a valid credentials cache.

Errors: returns error code if nothing matches; returns error code if
couldn't delete.

\begin{funcdecl}{krb5_cc_set_flags}{krb5_error_code}{\funcinout}
\funcarg{krb5_context}{context}
\funcarg{krb5_ccache}{id}
\funcin
\funcarg{krb5_flags}{flags}
\end{funcdecl}

Sets the flags on the cache \funcparam{id} to \funcparam{flags}.  Useful
flags are defined in {\tt <krb5.h>}.

\begin{funcdecl}{krb5_get_notification_message}{unsigned int}{\funcvoid}
\end{funcdecl}

Intended for use by Windows. Will register a unique message type using
\funcname{RegisterWindowMessage} which will be notified whenever the
cache changes. This will allow all processes to recheck their caches.


\subsection{Replay cache functions}
The replay cache functions deal with verifying that AP_REQ's do not
contain duplicate authenticators; the storage must be non-volatile for
the site-determined validity period of authenticators.

Each replay cache has a string ``name'' associated with it.  The use of
this name is dependent on the underlying caching strategy (for
file-based things, it would be a cache file name).  The
caching strategy uses non-volatile storage so that replay
integrity can be maintained across system failures.

\begin{funcdecl}{krb5_auth_to_rep}{krb5_error_code}{\funcinout}
\funcarg{krb5_context}{context}
\funcin
\funcarg{krb5_tkt_authent *}{auth}
\funcout
\funcarg{krb5_donot_replay *}{rep}
\end{funcdecl}
Extract the relevant parts of \funcparam{auth} and fill them into the
structure pointed to by \funcparam{rep}.  \funcparam{rep{\ptsto}client}
and \funcparam{rep{\ptsto}server} are set to allocated storage and
should be freed when \funcparam{*rep} is no longer needed.

\begin{funcdecl}{krb5_rc_resolve_full}{krb5_error_code}{\funcinout}
\funcarg{krb5_context}{context}
\funcarg{krb5_rcache *}{id}
\funcin
\funcarg{char *}{string_name}
\end{funcdecl}

\begin{sloppypar}
\funcparam{id} is filled in to identify a replay cache which
corresponds to the name in \funcparam{string_name}.  The cache is not opened.
Requires that \funcparam{string_name} be of the form ``type:residual''
and that ``type'' is a type known to the library.
\end{sloppypar}

Before the cache can be used \funcname{krb5_rc_initialize} or
\funcname{krb5_rc_recover} must be called.

Errors: error if cannot resolve name.


\begin{funcdecl}{krb5_rc_resolve_type}{krb5_error_code}{\funcinout}
\funcarg{krb5_context}{context}
\funcarg{krb5_rcache *}{id}
\funcin
\funcarg{char *}{type}
\end{funcdecl}

\internalfunc

Looks up \funcparam{type} in the list of knows cache types and if found
attaches the operations to \funcparam{*id} which must be previously
allocated. 

If \funcparam{type} is not found, {\sc krb5_rc_type_notfound} is returned.

\begin{funcdecl}{krb5_rc_register_type}{krb5_error_code}{\funcin}
\funcarg{krb5_context}{context}
\funcarg{krb5_rc_ops *}{ops}
\end{funcdecl}
Adds a new replay cache type implemented and identified by
\funcparam{ops} to the set recognized by
\funcname{krb5_rc_resolve}.  This function requires that a ticket
cache of the type named in 
\funcparam{ops{\ptsto}prefix} has not been previously registered.


\begin{funcdecl}{krb5_rc_default_name}{char *}{\funcin}
\funcarg{krb5_context}{context}
\end{funcdecl}

\begin{sloppypar}
Returns  the name of the default replay cache; this may be equivalent to
\funcnamenoparens{getenv}({\tt "KRB5RCACHE"}) with an appropriate fallback.
\end{sloppypar}

\begin{funcdecl}{krb5_rc_default_type}{char *}{\funcin}
\funcarg{krb5_context}{context}
\end{funcdecl}

Returns the type of the default replay cache.

\begin{funcdecl}{krb5_rc_default}{krb5_error_code}{\funcinout}
\funcarg{krb5_context}{context}
\funcarg{krb5_rcache *}{id}
\end{funcdecl}

This function returns an unopened replay cache of the default type and
default name (as would be returned by \funcname{krb5_rc_default_type}
and \funcname{krb5_rc_default_name}).  Before the cache can be used
\funcname{krb5_rc_initialize} or \funcname{krb5_rc_recover} must be
called.


\begin{funcdecl}{krb5_rc_initialize}{krb5_error_code}{\funcin}
\funcarg{krb5_context}{context}
\funcarg{krb5_rcache}{id}
\funcarg{krb5_deltat}{auth_lifespan}
\end{funcdecl}

Creates/refreshes the replay cache identified by \funcparam{id} and sets its
authenticator lifespan to \funcparam{auth_lifespan}.  If the 
replay cache already exists, its contents are destroyed.

Errors: permission errors, system errors

\begin{funcdecl}{krb5_rc_recover}{krb5_error_code}{\funcin}
\funcarg{krb5_context}{context}
\funcarg{krb5_rcache}{id}
\end{funcdecl}
Attempts to recover the replay cache \funcparam{id}, (presumably after a
system crash or server restart).

Errors: error indicating that no cache was found to recover

\begin{funcdecl}{krb5_rc_destroy}{krb5_error_code}{\funcin}
\funcarg{krb5_context}{context}
\funcarg{krb5_rcache}{id}
\end{funcdecl}

Destroys the replay cache \funcparam{id}.
Requires that \funcparam{id} identifies a valid replay cache.

Errors: permission errors.

\begin{funcdecl}{krb5_rc_close}{krb5_error_code}{\funcin}
\funcarg{krb5_context}{context}
\funcarg{krb5_rcache}{id}
\end{funcdecl}

Closes the replay cache \funcparam{id}, invalidates \funcparam{id},
and releases any other resources acquired during use of the replay cache.
Requires that \funcparam{id} identifies a valid replay cache.

Errors: permission errors

\begin{funcdecl}{krb5_rc_store}{krb5_error_code}{\funcin}
\funcarg{krb5_context}{context}
\funcarg{krb5_rcache}{id}
\funcarg{krb5_donot_replay *}{rep}
\end{funcdecl}
Stores \funcparam{rep} in the replay cache \funcparam{id}.
Requires that \funcparam{id} identifies a valid replay cache.

Returns KRB5KRB_AP_ERR_REPEAT if \funcparam{rep} is already in the
cache.  May also return permission errors, storage failure errors.

\begin{funcdecl}{krb5_rc_expunge}{krb5_error_code}{\funcin}
\funcarg{krb5_context}{context}
\funcarg{krb5_rcache}{id}
\end{funcdecl}
Removes all expired replay information (i.e. those entries which are
older than then authenticator lifespan of the cache) from the cache
\funcparam{id}.  Requires that \funcparam{id} identifies a valid replay
cache.

Errors: permission errors.

\begin{funcdecl}{krb5_rc_get_lifespan}{krb5_error_code}{\funcin}
\funcarg{krb5_context}{context}
\funcarg{krb5_rcache}{id}
\funcout
\funcarg{krb5_deltat *}{auth_lifespan}
\end{funcdecl}
Fills in \funcparam{auth_lifespan} with the lifespan of
the cache \funcparam{id}.
Requires that \funcparam{id} identifies a valid replay cache.

\begin{funcdecl}{krb5_rc_resolve}{krb5_error_code}{\funcinout}
\funcarg{krb5_context}{context}
\funcarg{krb5_rcache}{id}
\funcin
\funcarg{char *}{name}
\end{funcdecl}

Initializes private data attached to \funcparam{id}.  This function MUST
be called before the other per-replay cache functions.

Requires that \funcparam{id} points to allocated space, with an
initialized \funcparam{id{\ptsto}ops} field.

Since \funcname{krb5_rc_resolve} allocates memory,
\funcname{krb5_rc_close} must be called to free the allocated memory,
even if neither \funcname{krb5_rc_initialize} or
\funcname{krb5_rc_recover} were successfully called by the application.

Returns:  allocation errors.


\begin{funcdecl}{krb5_rc_get_name}{char *}{\funcin}
\funcarg{krb5_context}{context}
\funcarg{krb5_rcache}{id}
\end{funcdecl}

Returns the name (excluding the type) of the rcache \funcparam{id}.
Requires that \funcparam{id} identifies a valid replay cache.

\begin{funcdecl}{krb5_rc_get_type}{char *}{\funcin}
\funcarg{krb5_context}{context}
\funcarg{krb5_rcache}{id}
\end{funcdecl}

Returns the type (excluding the name) of the rcache \funcparam{id}.
Requires that \funcparam{id} identifies a valid replay cache.




\subsection{Key table functions}
The key table functions deal with storing and retrieving service keys
for use by unattended services which participate in authentication exchanges.

Keytab routines are all be atomic.  Every routine that acquires
a non-sharable resource releases it before it returns. 

All keytab types support multiple concurrent sequential scans.

The order of values returned from \funcname{krb5_kt_next_entry} is
unspecified.

Although the ``right thing'' should happen if the program aborts
abnormally, a close routine, \funcname{krb5_kt_free_entry},  is provided
for freeing resources, etc.  People should use the close routine when
they are finished.

\begin{funcdecl}{krb5_kt_register}{krb5_error_code}{\funcinout}
\funcarg{krb5_context}{context}
\funcin
\funcarg{krb5_kt_ops *}{ops}
\end{funcdecl}


Adds a new ticket cache type to the set recognized by
\funcname{krb5_kt_resolve}.
Requires that a keytab type named \funcparam{ops{\ptsto}prefix} is not
yet known.

An error is returned if \funcparam{ops{\ptsto}prefix} is already known.

\begin{funcdecl}{krb5_kt_resolve}{krb5_error_code}{\funcinout}
\funcarg{krb5_context}{context}
\funcin
\funcarg{const char *}{string_name}
\funcout
\funcarg{krb5_keytab *}{id}
\end{funcdecl}

Fills in \funcparam{*id} with a handle identifying the keytab with name
``string_name''.  The keytab is not opened.
Requires that \funcparam{string_name} be of the form ``type:residual'' and
``type'' is a type known to the library.

Errors: badly formatted name.
                
\begin{funcdecl}{krb5_kt_default_name}{krb5_error_code}{\funcinout}
\funcarg{krb5_context}{context}
\funcin
\funcarg{char *}{name}
\funcarg{int}{namesize}
\end{funcdecl}

\funcparam{name} is filled in with the first \funcparam{namesize} bytes of
the name of the default keytab.
If the name is shorter than \funcparam{namesize}, then the remainder of
\funcparam{name} will be zeroed.


\begin{funcdecl}{krb5_kt_default}{krb5_error_code}{\funcinout}
\funcarg{krb5_context}{context}
\funcin
\funcarg{krb5_keytab *}{id}
\end{funcdecl}

Fills in \funcparam{id} with a handle identifying the default keytab.

\begin{funcdecl}{krb5_kt_read_service_key}{krb5_error_code}{\funcinout}
\funcarg{krb5_context}{context}
\funcin
\funcarg{krb5_pointer}{keyprocarg}
\funcarg{krb5_principal}{principal}
\funcarg{krb5_kvno}{vno}
\funcarg{krb5_keytype}{keytype}
\funcout
\funcarg{krb5_keyblock **}{key}
\end{funcdecl}

If \funcname{keyprocarg} is not NULL, it is taken to be a
\datatype{char *} denoting the name of a keytab.  Otherwise, the default
keytab will be used.
The keytab is opened and searched for the entry identified by
\funcparam{principal}, \funcparam{keytype}, and \funcparam{vno}, 
returning the resulting key
in \funcparam{*key} or returning an error code if it is not found. 

\funcname{krb5_free_keyblock} should be called on \funcparam{*key} when
the caller is finished with the key.

Returns an error code if the entry is not found.

\begin{funcdecl}{krb5_kt_add_entry}{krb5_error_code}{\funcinout}
\funcarg{krb5_context}{context}
\funcin
\funcarg{krb5_keytab}{id}
\funcarg{krb5_keytab_entry *}{entry}
\end{funcdecl}

Calls the keytab-specific add routine \funcname{krb5_kt_add_internal}
with the same function arguments.  If this routine is not available,
then KRB5_KT_NOWRITE is returned.

\begin{funcdecl}{krb5_kt_remove_entry}{krb5_error_code}{\funcinout}
\funcarg{krb5_context}{context}
\funcin
\funcarg{krb5_keytab}{id}
\funcarg{krb5_keytab_entry *}{entry}
\end{funcdecl}

Calls the keytab-specific remove routine
\funcname{krb5_kt_remove_internal} with the same function arguments.
If this routine is not available, then KRB5_KT_NOWRITE is returned.

\begin{funcdecl}{krb5_kt_get_name}{krb5_error_code}{\funcinout}
\funcarg{krb5_context}{context}
\funcarg{krb5_keytab}{id}
\funcout
\funcarg{char *}{name}
\funcin
\funcarg{unsigned int}{namesize}
\end{funcdecl}

\funcarg{name} is filled in with the first \funcparam{namesize} bytes of
the name of the keytab identified by \funcname{id}.
If the name is shorter than \funcparam{namesize}, then \funcarg{name}
will be null-terminated.

\begin{funcdecl}{krb5_kt_close}{krb5_error_code}{\funcinout}
\funcarg{krb5_context}{context}
\funcarg{krb5_keytab}{id}
\end{funcdecl}

Closes the keytab identified by \funcparam{id} and invalidates
\funcparam{id}, and releases any other resources acquired during use of
the key table.

Requires that \funcparam{id} identifies a keytab.

\begin{funcdecl}{krb5_kt_get_entry}{krb5_error_code}{\funcinout}
\funcarg{krb5_context}{context}
\funcarg{krb5_keytab}{id}
\funcin
\funcarg{krb5_principal}{principal}
\funcarg{krb5_kvno}{vno}
\funcarg{krb5_keytype}{keytype}
\funcout
\funcarg{krb5_keytab_entry *}{entry}
\end{funcdecl}

\begin{sloppypar}
Searches the keytab identified by \funcparam{id} for an entry whose
principal matches \funcparam{principal}, whose keytype matches 
\funcparam{keytype}, and
whose key version number matches \funcparam{vno}.  If \funcparam{vno} is
zero, the first entry whose principal matches is returned.
\end{sloppypar}

Returns an error code if no suitable entry is found.  If an entry is
found, the entry is returned in \funcparam{*entry}; its contents should
be deallocated by calling \funcname{krb5_kt_free_entry} when no longer
needed.

\begin{funcdecl}{krb5_kt_free_entry}{krb5_error_code}{\funcinout}
\funcarg{krb5_context}{context}
\funcarg{krb5_keytab_entry *}{entry}
\end{funcdecl}

Releases all storage allocated for \funcparam{entry}, which must point
to a structure previously filled in by \funcname{krb5_kt_get_entry} or
\funcname{krb5_kt_next_entry}.

\begin{funcdecl}{krb5_kt_start_seq_get}{krb5_error_code}{\funcinout}
\funcarg{krb5_context}{context}
\funcarg{krb5_keytab}{id}
\funcout
\funcarg{krb5_kt_cursor *}{cursor}
\end{funcdecl}

Prepares to read sequentially every key in the keytab identified by
\funcparam{id}.
\funcparam{cursor} is filled in with a cursor to be used in calls to
\funcname{krb5_kt_next_entry}.

\begin{funcdecl}{krb5_kt_next_entry}{krb5_error_code}{\funcinout}
\funcarg{krb5_context}{context}
\funcarg{krb5_keytab}{id}
\funcout
\funcarg{krb5_keytab_entry *}{entry}
\funcinout
\funcarg{krb5_kt_cursor *}{cursor}
\end{funcdecl}

Fetches the ``next'' entry in the keytab, returning it in
\funcparam{*entry}, and updates \funcparam{*cursor} for the next
request.  If the keytab changes during the sequential get, an error is
guaranteed.  \funcparam{*entry} should be freed after use by calling
\funcname{krb5_kt_free_entry}.

Requires that \funcparam{id} identifies a valid keytab.  and
\funcparam{*cursor} be a cursor returned by
\funcname{krb5_kt_start_seq_get} or a subsequent call to
\funcname{krb5_kt_next_entry}.

Errors: error code if no more cache entries or if the keytab changes.

\begin{funcdecl}{krb5_kt_end_seq_get}{krb5_error_code}{\funcinout}
\funcarg{krb5_context}{context}
\funcarg{krb5_keytab}{id}
\funcarg{krb5_kt_cursor *}{cursor}
\end{funcdecl}

Finishes sequential processing mode and invalidates \funcparam{cursor},
which must never be re-used after this call.

Requires that \funcparam{id} identifies a valid keytab  and
\funcparam{*cursor} be a cursor returned by
\funcname{krb5_kt_start_seq_get} or a subsequent call to
\funcname{krb5_kt_next_entry}.

May return error code if \funcparam{cursor} is invalid.




\subsection{Free functions}
The free functions deal with deallocation of memory that has been
allocated by various routines. It is recommended that the developer use
these routines as they will know about the contents of the structures.

\begin{funcdecl}{krb5_xfree}{void}{\funcinout}
\funcarg{void *}{ptr}
\end{funcdecl}

Frees the pointer \funcarg{ptr}. This is a wrapper macro to
\funcname{free} that is designed to keep lint ``happy.''

\begin{funcdecl}{krb5_free_data}{void}{\funcinout}
\funcarg{krb5_context}{context}
\funcarg{krb5_data *}{val}
\end{funcdecl}

Frees the data structure \funcparam{val}, including the pointer
\funcparam{val} which has been allocate by any of numerous routines.


\begin{funcdecl}{krb5_free_princial}{void}{\funcinout}
\funcarg{krb5_context}{context}
\funcarg{krb5_principal}{val}
\end{funcdecl}

Frees the pwd_data \funcparam{val} that has been allocated from
\funcname{krb5_copy_principal}. 

\begin{funcdecl}{krb5_free_authenticator}{void}{\funcinout}
\funcarg{krb5_context}{context}
\funcarg{krb5_authenticator *}{val}
\end{funcdecl}

Frees the authenticator \funcparam{val}, including the pointer
\funcparam{val}. 

\begin{funcdecl}{krb5_free_authenticator_contents}{void}{\funcinout}
\funcarg{krb5_context}{context}
\funcarg{krb5_authenticator *}{val}
\end{funcdecl}

Frees the authenticator contents of \funcparam{val}. The pointer 
\funcparam{val} is not freed.


\begin{funcdecl}{krb5_free_addresses}{void}{\funcinout}
\funcarg{krb5_context}{context}
\funcarg{krb5_address **}{val}
\end{funcdecl}

Frees the series of addresses \funcparam{*val} that have been allocated from
\funcname{krb5_copy_addresses}. 

\begin{funcdecl}{krb5_free_address}{void}{\funcinout}
\funcarg{krb5_context}{context}
\funcarg{krb5_address *}{val}
\end{funcdecl}

Frees the address \funcparam{val}.

\begin{funcdecl}{krb5_free_authdata}{void}{\funcinout}
\funcarg{krb5_context}{context}
\funcarg{krb5_authdata **}{val}
\end{funcdecl}

Frees the authdata structure pointed to by \funcparam{val} that has been
allocated from 
\funcname{krb5_copy_authdata}. 

\begin{funcdecl}{krb5_free_enc_tkt_part}{void}{\funcinout}
\funcarg{krb5_context}{context}
\funcarg{krb5_enc_tkt_part *}{val}
\end{funcdecl}

Frees \funcparam{val} that has been allocated from
\funcname{krb5_enc_tkt_part} and \funcname{krb5_decrypt_tkt_part}.

\begin{funcdecl}{krb5_free_ticket}{void}{\funcinout}
\funcarg{krb5_context}{context}
\funcarg{krb5_ticket *}{val}
\end{funcdecl}

Frees the ticket \funcparam{val} that has been allocated from
\funcname{krb5_copy_ticket} and other routines.

\begin{funcdecl}{krb5_free_tickets}{void}{\funcinout}
\funcarg{krb5_context}{context}
\funcarg{krb5_ticket **}{val}
\end{funcdecl}

Frees the tickets pointed to by \funcparam{val}.

\begin{funcdecl}{krb5_free_kdc_req}{void}{\funcinout}
\funcarg{krb5_context}{context}
\funcarg{krb5_kdc_req *}{val}
\end{funcdecl}

Frees the kdc_req \funcparam{val} and all substructures. The pointer
\funcparam{val} is freed as well.

\begin{funcdecl}{krb5_free_kdc_rep}{void}{\funcinout}
\funcarg{krb5_context}{context}
\funcarg{krb5_kdc_rep *}{val}
\end{funcdecl}

Frees the kdc_rep \funcparam{val} that has been allocated from
\funcname{krb5_get_in_tkt}. 

\begin{funcdecl}{krb5_free_kdc_rep_part}{void}{\funcinout}
\funcarg{krb5_context}{context}
\funcarg{krb5_enc_kdc_rep_part *}{val}
\end{funcdecl}

Frees the kdc_rep_part \funcparam{val}.

\begin{funcdecl}{krb5_free_error}{void}{\funcinout}
\funcarg{krb5_context}{context}
\funcarg{krb5_error *}{val}
\end{funcdecl}

Frees the error \funcparam{val} that has been allocated from
\funcname{krb5_read_error} or \funcname{krb5_sendauth}. 

\begin{funcdecl}{krb5_free_ap_req}{void}{\funcinout}
\funcarg{krb5_context}{context}
\funcarg{krb5_ap_req *}{val}
\end{funcdecl}

Frees the ap_req \funcparam{val}.

\begin{funcdecl}{krb5_free_ap_rep}{void}{\funcinout}
\funcarg{krb5_context}{context}
\funcarg{krb5_ap_rep *}{val}
\end{funcdecl}

Frees the ap_rep \funcparam{val}.

\begin{funcdecl}{krb5_free_safe}{void}{\funcinout}
\funcarg{krb5_context}{context}
\funcarg{krb5_safe *}{val}
\end{funcdecl}

Frees the safe application data \funcparam{val} that is allocated with
\funcparam{decode_krb5_safe}. 


\begin{funcdecl}{krb5_free_priv}{void}{\funcinout}
\funcarg{krb5_context}{context}
\funcarg{krb5_priv *}{val}
\end{funcdecl}

Frees the private data  \funcparam{val} that has been allocated from
\funcname{decode_krb5_priv}. 

\begin{funcdecl}{krb5_free_priv_enc_part}{void}{\funcinout}
\funcarg{krb5_context}{context}
\funcarg{krb5_priv_enc_part *}{val}
\end{funcdecl}

Frees the private encoded part \funcparam{val} that has been allocated from
\funcname{decode_krb5_enc_priv_part}. 

\begin{funcdecl}{krb5_free_cred}{void}{\funcinout}
\funcarg{krb5_context}{context}
\funcarg{krb5_cred *}{val}
\end{funcdecl}

Frees the credential \funcparam{val}.

\begin{funcdecl}{krb5_free_creds}{void}{\funcinout}
\funcarg{krb5_context}{context}
\funcarg{krb5_creds *}{val}
\end{funcdecl}

Calls \funcname{krb5_free_cred_contents} with \funcparam{val} as the
argument. \funcparam{val} is freed as well.

\begin{funcdecl}{krb5_free_cred_contents}{void}{\funcinout}
\funcarg{krb5_context}{context}
\funcarg{krb5_creds *}{val}
\end{funcdecl}

The function zeros out the session key stored in the credential and then
frees the credentials structures. The argument \funcparam{val} is
{\bf not} freed.


\begin{funcdecl}{krb5_free_cred_enc_part}{void}{\funcinout}
\funcarg{krb5_context}{context}
\funcarg{krb5_cred_enc_part *}{val}
\end{funcdecl}

Frees the addresses and ticket_info elements of
\funcparam{val}. \funcparam{val} is {\bf not} freed by this routine.

\begin{funcdecl}{krb5_free_checksum}{void}{\funcinout}
\funcarg{krb5_context}{context}
\funcarg{krb5_checksum *}{val}
\end{funcdecl}

The checksum and the pointer \funcparam{val} are both freed. 

\begin{funcdecl}{krb5_free_keyblock}{void}{\funcinout}
\funcarg{krb5_context}{context}
\funcarg{krb5_keyblock *}{val}
\end{funcdecl}

The keyblock contents of \funcparam{val} are zeroed and the memory
freed. The pointer \funcparam{val} is freed as well.

\begin{funcdecl}{krb5_free_pa_data}{void}{\funcinout}
\funcarg{krb5_context}{context}
\funcarg{krb5_pa_data **}{val}
\end{funcdecl}

Frees the contents of \funcparam{*val}. \funcparam{val} is freed as
well.

\begin{funcdecl}{krb5_free_ap_rep_enc_part}{void}{\funcinout}
\funcarg{krb5_context}{context}
\funcarg{krb5_ap_rep_enc_part *}{val}
\end{funcdecl}

Frees the subkey keyblock (if set) as well as \funcparam{val} that has
been allocated from \funcname{krb5_rd_rep} or \funcname{krb5_send_auth}.

\begin{funcdecl}{krb5_free_tkt_authent}{void}{\funcinout}
\funcarg{krb5_context}{context}
\funcarg{krb5_tkt_authent *}{val}
\end{funcdecl}

Frees the ticket and authenticator portions of \funcparam{val}. The
pointer \funcparam{val} is freed as well.

\begin{funcdecl}{krb5_free_pwd_data}{void}{\funcinout}
\funcarg{krb5_context}{context}
\funcarg{passwd_pwd_data *}{val}
\end{funcdecl}

Frees the pwd_data \funcparam{val} that has been allocated from
\funcname{decode_krb5_pwd_data}. 

\begin{funcdecl}{krb5_free_pwd_sequences}{void}{\funcinout}
\funcarg{krb5_context}{context}
\funcarg{passwd_phrase_element **}{val}
\end{funcdecl}

Frees the passwd_phrase_element \funcparam{val}. This is usually called
from \funcname{krb5_free_pwd_data}.

\begin{funcdecl}{krb5_free_realm_tree}{void}{\funcinout}
\funcarg{krb5_context}{context}
\funcarg{krb5_principal *}{realms}
\end{funcdecl}

Frees the realms tree \funcparam{realms} returned by
\funcname{krb5_walk_realm_tree}.

\begin{funcdecl}{krb5_free_tgt_creds}{void}{\funcinout}
\funcarg{krb5_context}{context}
\funcarg{krb5_creds **}{tgts}
\end{funcdecl}

Frees the TGT credentials \funcparam{tgts} returned by
\funcname{krb5_get_cred_from_kdc}.



\subsection{Operating-system specific functions}
The operating-system specific functions provide an interface between the
other parts of the \libname{libkrb5.a} libraries and the operating system.

Beware! Any of the functions below are allowed to be implemented as
macros.  Prototypes for functions can be found in {\tt
<krb5.h>}; other definitions (including macros, if used) are
in {\tt <krb5/libos.h>}.

The following global symbols are provided in \libname{libos.a}.  If you
wish to substitute for any of them, you must substitute for all of them
(they are all declared and initialized in the same object file):
\begin{description}
% These come from src/lib/osconfig.c
\item[extern char *\globalname{krb5_defkeyname}:] default name of key
table file
\item[extern char *\globalname{krb5_lname_file}:] name of aname/lname
translation database
\item[extern int \globalname{krb5_max_dgram_size}:] maximum allowable
datagram size
\item[extern int \globalname{krb5_max_skdc_timeout}:] maximum
per-message KDC reply timeout
\item[extern int \globalname{krb5_skdc_timeout_shift}:] shift factor
(bits) to exponentially back-off the KDC timeouts
\item[extern int \globalname{krb5_skdc_timeout_1}:] initial KDC timeout
\item[extern char *\globalname{krb5_kdc_udp_portname}:] name of KDC UDP port
\item[extern char *\globalname{krb5_default_pwd_prompt1}:] first prompt
for password reading.
\item[extern char *\globalname{krb5_default_pwd_prompt2}:] second prompt

\end{description}

\subsubsection{Operating specific context}
The \datatype{krb5_context} has space for operating system specific
data. These functions are called from \funcname{krb5_init_context} and
\funcname{krb5_free_context}, but are included here for completeness.

\begin{funcdecl}{krb5_os_init_context}{krb5_error_code}{\funcinout}
\funcarg{krb5_context}{context}
\end{funcdecl}

\internalfunc

Initializes \funcparam{context{\ptsto}os_context} and establishes the
location of the initial configuration files. 

\begin{funcdecl}{krb5_os_free_context}{krb5_error_code}{\funcinout}
\funcarg{krb5_context}{context}
\end{funcdecl}

\internalfunc

Frees the operating system specific portion of \funcparam{context}. 

\subsubsection{Configuration based functions}
These functions allow access to configuration specific information. In
some cases, the configuration may be overriden by program control.


\begin{funcdecl}{krb5_set_config_files}{krb5_error_code}{\funcinout}
\funcarg{krb5_context}{context}
\funcin
\funcarg{const char **}{filenames}
\end{funcdecl}

Sets the list of configuration files to be examined in determining
machine defaults. \funcparam{filenames} is an array of files to check in
order. The array must have a NULL entry as the last element.

Returns system errors.

\begin{funcdecl}{krb5_get_krbhst}{krb5_error_code}{\funcin}
\funcarg{krb5_context}{context}
\funcarg{const krb5_data *}{realm}
\funcout
\funcarg{char ***}{hostlist}
\end{funcdecl}

Figures out the Kerberos server names for the given \funcparam{realm},
filling in \funcparam{hostlist} with a null terminated array of
pointers to hostnames. 
 
If \funcparam{realm} is unknown, the filled-in pointer is set to NULL.

The pointer array and strings pointed to are all in allocated storage,
and should be freed by the caller when finished.

Returns system errors.

\begin{funcdecl}{krb5_free_krbhst}{krb5_error_code}{\funcin}
\funcarg{krb5_context}{context}
\funcarg{char * const *}{hostlist}
\end{funcdecl}

Frees the storage taken by a host list returned by \funcname{krb5_get_krbhst}.

\begin{funcdecl}{krb5_get_default_realm}{krb5_error_code}{\funcin}
\funcarg{krb5_context}{context}
\funcout
\funcarg{char **}{lrealm}
\end{funcdecl}

Retrieves the default realm to be used if no user-specified realm is
available (e.g. to interpret a user-typed principal name with the
realm omitted for convenience), filling in \funcparam{lrealm} with a
pointer to the default realm in allocated storage.

It is the caller's responsibility for freeing the allocated storage
pointed to be \funcparam{lream} when it is finished with it.

Returns system errors.

\begin{funcdecl}{krb5_set_default_realm}{krb5_error_code}{\funcin}
\funcarg{krb5_context}{context}
\funcarg{char *}{realm}
\end{funcdecl}

Sets the default realm to be used if no user-specified realm is
available (e.g. to interpret a user-typed principal name with the
realm omitted for convenience). (c.f. krb5_get_default_realm)

If \funcparam{realm} is NULL, then the operating system default value
will used.

Returns system errors.

\begin{funcdecl}{krb5_get_host_realm}{krb5_error_code}{\funcin}
\funcarg{krb5_context}{context}
\funcarg{const char *}{host}
\funcout
\funcarg{char ***}{realmlist}
\end{funcdecl}

Figures out the Kerberos realm names for \funcparam{host}, filling in
\funcparam{realmlist} with a
pointer to an argv[] style list of names, terminated with a null pointer.
 
If \funcparam{host} is NULL, the local host's realms are determined.

If there are no known realms for the host, the filled-in pointer is set
to NULL.

The pointer array and strings pointed to are all in allocated storage,
and should be freed by the caller when finished.

Returns system errors.

\begin{funcdecl}{krb5_free_host_realm}{krb5_error_code}{\funcin}
\funcarg{krb5_context}{context}
\funcarg{char * const *}{realmlist}
\end{funcdecl}

Frees the storage taken by a \funcparam{realmlist} returned by
\funcname{krb5_get_local_realm}.

\begin{funcdecl}{krb5_get_realm_domain}{krb5_error_code}{\funcinout}
\funcarg{krb5_context}{context}
\funcin
\funcarg{const char *}{realm}
\funcout
\funcarg{char **}{domain}
\end{funcdecl}

Determines the proper name of a realm. This is mainly so that a krb4
principal can be converted properly into a krb5 one. If
\funcparam{realm} is null, the function will assume the default realm of
the host. The returned \funcparam{*domain} is allocated and must be
freed by the caller. 

\subsubsection{Disk based functions}
These functions all relate to disk based I/O.

\begin{funcdecl}{krb5_lock_file}{krb5_error_code}{\funcin}
\funcarg{krb5_context}{context}
\funcarg{in}{fd}
\funcarg{int}{mode}
\end{funcdecl}

Attempts to lock the file in the given \funcparam{mode}; returns 0 for a
successful lock, or an error code otherwise.

The caller should arrange for the file referred by \funcparam{fd} to be
opened in such a way as to allow the required lock.

Modes are given in {\tt <krb5/libos.h>}

\begin{funcdecl}{krb5_unlock_file}{krb5_error_code}{\funcin}
\funcarg{krb5_context}{context}
\funcarg{int}{fd}
\end{funcdecl}

Attempts to (completely) unlock the file.  Returns 0 if successful,
or an error code otherwise.


\begin{funcdecl}{krb5_create_secure_file}{krb5_error_code}{\funcin}
\funcarg{krb5_context}{context}
\funcarg{const char *}{pathname}
\end{funcdecl}

Creates a file named pathname which can only be read by the current
user.

\begin{funcdecl}{krb5_sync_disk_file}{krb5_error_code}{\funcin}
\funcarg{krb5_context}{context}
\funcarg{FILE *}{fp}
\end{funcdecl}

Assures that the changes made to the file pointed to by the file
handle
fp are forced out to disk.

\subsubsection{Network based routines}

These routines send and receive network data the specifics 
of addresses and families on a given operating system.

\begin{funcdecl}{krb5_os_localaddr}{krb5_error_code}{\funcin}
\funcarg{krb5_context}{context}
\funcout
\funcarg{krb5_address ***}{addr}
\end{funcdecl}

Return all the protocol addresses of this host.

Compile-time configuration flags will indicate which protocol family
addresses might be returned.
\funcparam{*addr} is filled in to point to an array of address pointers,
terminated by a null pointer.  All the storage pointed to is allocated
and should be freed by the caller with \funcname{krb5_free_address}
when no longer needed.

\begin{funcdecl}{krb5_gen_portaddr}{krb5_error_code}{\funcin}
\funcarg{krb5_context}{context}
\funcarg{const krb5_address *}{adr}
\funcarg{krb5_const_pointer}{ptr}
\funcout
\funcarg{krb5_address **}{outaddr}
\end{funcdecl}

Given an address \funcparam{adr} and an additional address-type specific
portion pointed to by
\funcparam{port} this routine
combines them into a freshly-allocated
\datatype{krb5_address} with type \datatype{ADDRTYPE_ADDRPORT} and fills in
\funcparam{*outaddr} to point to this address.  For IP addresses,
\funcparam{ptr} should point to a network-byte-order TCP or UDP port
number.  Upon success, \funcparam{*outaddr} will point to an allocated
address which should be freed with \funcname{krb5_free_address}.


\begin{funcdecl}{krb5_sendto_kdc}{krb5_error_code}{\funcin}
\funcarg{krb5_context}{context}
\funcarg{const krb5_data *}{send}
\funcarg{const krb5_data *}{realm}
\funcout
\funcarg{krb5_data *}{receive}
\end{funcdecl}

Send the message \funcparam{send} to a KDC for realm \funcparam{realm} and
return the response (if any) in \funcparam{receive}.

If the message is sent and a response is received, 0 is returned,
otherwise an error code is returned.

The storage for \funcparam{receive} is allocated and should be freed by
the caller when finished.


\begin{funcdecl}{krb5_net_read}{int}{\funcin}
\funcarg{krb5_context}{context}
\funcarg{int}{fd}
\funcout
\funcarg{char *}{buf}
\funcin
\funcarg{int}{len}
\end{funcdecl}

Like read(2), but guarantees that it reads as much as was requested
or returns -1 and sets errno.

(make sure your sender will send all the stuff you are looking for!)
Only useful on stream sockets and pipes.

\begin{funcdecl}{krb5_net_write}{int}{\funcin}
\funcarg{krb5_context}{context}
\funcarg{int}{fd}
\funcarg{const char *}{buf}
\funcarg{int}{len}
\end{funcdecl}

Like write(2), but guarantees that it writes as much as was requested
or returns -1 and sets errno.

Only useful on stream sockets and pipes.

\begin{funcdecl}{krb5_write_message}{krb5_error_code}{\funcin}
\funcarg{krb5_context}{context}
\funcarg{krb5_pointer}{fd}
\funcarg{krb5_data *}{data}
\end{funcdecl}


\funcname{krb5_write_message} writes data to the network as a message,
using the network connection pointed to by \funcparam{fd}.

\begin{funcdecl}{krb5_read_message}{krb5_error_code}{\funcin}
\funcarg{krb5_context}{context}
\funcarg{krb5_pointer}{fd}
\funcout
\funcarg{krb5_data *}{data}
\end{funcdecl}

Reads data from the network as a message, using the network connection
pointed to by fd.

\subsubsection{Operating specific access functions}
These functions are involved with access control decisions and policies.

\begin{funcdecl}{krb5_aname_to_localname}{krb5_error_code}{\funcin}
\funcarg{krb5_context}{context}
\funcarg{krb5_const_principal}{aname}
\funcarg{int}{lnsize}
\funcout
\funcarg{char *}{lname}
\end{funcdecl}

Converts a principal name \funcparam{aname} to a local name suitable for use by
programs wishing a translation to an environment-specific name (e.g.
user account name).

\funcparam{lnsize} specifies the maximum length name that is to be filled into
\funcparam{lname}.
The translation will be null terminated in all non-error returns.

Returns system errors.

\begin{funcdecl}{krb5_kuserok}{krb5_boolean}{\funcin}
\funcarg{krb5_context}{context}
\funcarg{krb5_principal}{principal}
\funcarg{const char *}{luser}
\end{funcdecl}

Given a Kerberos principal \funcparam{principal}, and a local username
\funcparam{luser},
determine whether user is authorized to login to the account \funcparam{luser}.
Returns TRUE if authorized, FALSE if not authorized.

\begin{funcdecl}{krb5_sname_to_principal}{krb5_error_code}{\funcin}
\funcarg{krb5_context}{context}
\funcarg{const char *}{hostname}
\funcarg{const char *}{sname}
\funcarg{krb5_int32}{type}
\funcout
\funcarg{krb5_principal *}{ret_princ}
\end{funcdecl}

Given a hostname \funcparam{hostname} and a generic service name
\funcparam{sname}, this function generates a full principal name to be
used when authenticating with the named service on the host.  The full
prinicpal name is  returned  in \funcparam{ret_princ}.

The realm of the
principal is determined internally by calling \funcname{krb5_get_host_realm}.

The \funcparam{type} argument controls how
\funcname{krb5_sname_to_principal} generates the principal name,
\funcparam{ret_princ}, for the named service, \funcparam{sname}.
Currently, two values   are supported: KRB5_NT_SRV_HOST, and
KRB5_NT_UNKNOWN.  

If \funcparam{type} is set to
KRB5_NT_SRV_HOST, the hostname will be
canonicalized, i.e. a fully qualified lowercase hostname using
the primary name and the domain name, before \funcparam{ret_princ} is
generated in the form
"sname/hostname@LOCAL.REALM." Most applications should use
KRB5_NT_SRV_HOST.  

However, if \funcparam{type} is set to KRB5_NT_UNKNOWN,
while the generated principal name will have    the form
"sname/hostname@LOCAL.REALM" the hostname will not be canonicalized
first.  It will appear exactly as it was passed in \funcparam{hostname}.  

The caller should release \funcparam{ret_princ}'s storage by calling
\funcname{krb5_free_principal} when it is finished with the principal.



\subsubsection{Miscellaneous operating specific functions}
These functions handle the other operating specific functions that do
not fall into any other major class.

\begin{funcdecl}{krb5_timeofday}{krb5_error_code}{\funcin}
\funcarg{krb5_context}{context}
\funcout
\funcarg{krb5_context}{context}
\funcarg{krb5_int32 *}{timeret}
\end{funcdecl}

Retrieves the system time of day, in seconds since the local system's
epoch.
[The ASN.1 encoding routines must convert this to the standard ASN.1
encoding as needed]

\begin{funcdecl}{krb5_us_timeofday}{krb5_error_code}{\funcin}
\funcarg{krb5_context}{context}
\funcout
\funcarg{krb5_int32 *}{seconds}
\funcarg{krb5_int32 *}{microseconds}
\end{funcdecl}

Retrieves the system time of day, in seconds since the local system's
epoch.
[The ASN.1 encoding routines must convert this to the standard ASN.1
encoding as needed]

{\raggedright The seconds portion is returned in \funcparam{*seconds}, the
microseconds portion in \funcparam{*microseconds}.}

\begin{funcdecl}{krb5_read_password}{krb5_error_code}{\funcin}
\funcarg{krb5_context}{context}
\funcarg{const char *}{prompt}
\funcarg{const char *}{prompt2}
\funcout
\funcarg{char *}{return_pwd}
\funcinout
\funcarg{unsigned int *}{size_return}
\end{funcdecl}

Read a password from the keyboard.  The first \funcparam{*size_return}
bytes of the password entered are returned in \funcparam{return_pwd}.
If fewer than \funcparam{*size_return} bytes are typed as a password,
the remainder of \funcparam{return_pwd} is zeroed.  Upon success, the
total number of bytes filled in is stored in \funcparam{*size_return}.

\funcparam{prompt} is used as the prompt for the first reading of a password.
It is printed to the terminal, and then a password is read from the
keyboard.  No newline or spaces are emitted between the prompt and the
cursor, unless the newline/space is included in the prompt.

If \funcparam{prompt2} is a null pointer, then the password is read
once.  If \funcparam{prompt2} is set, then it is used as a prompt to
read another password in the same manner as described for
\funcparam{prompt}.  After the second password is read, the two
passwords are compared, and an error is returned if they are not
identical.

Echoing is turned off when the password is read.

If there is an error in reading or verifying the password, an error code
is returned; else zero is returned.


\begin{funcdecl}{krb5_random_confounder}{krb5_error_code}{\funcin}
\funcarg{krb5_context}{context}
\funcarg{int}{size}
\funcout
\funcarg{krb5_pointer}{fillin}
\end{funcdecl}

Given a length and a pointer, fills in the area pointed to by
\funcparam{fillin} with \funcparam{size} random octets suitable for use
in a confounder.

\begin{funcdecl}{krb5_gen_replay_name}{krb5_error_code}{\funcin}
\funcarg{krb5_context}{context}
\funcarg{const krb5_address *}{inaddr}
\funcarg{const char *}{uniq}
\funcout
\funcarg{char **}{string}
\end{funcdecl}

Given a \datatype{krb5_address} with type \datatype{ADDRTYPE_ADDRPORT}
in \funcparam{inaddr}, this function unpacks its component address and
additional type, and uses them along with \funcparam{uniq} to allocate a
fresh string to represent the address and additional information.  The
string is suitable for use as a replay cache tag.  This string is
allocated and should be freed with \funcname{free} when the caller has
finished using it.  When using IP addresses, the components in
\funcparam{inaddr{\ptsto}contents} must be of type
\datatype{ADDRTYPE_INET} and \datatype{ADDRTYPE_PORT}.

% XXX Note that if the bug I sent in entitled ``krb5_gen_replay_name
% outputs char * when krb5_get_server_rcache expects krb5_data''
% (OpenVision Cambridge bug number 1582) causes the code of this
% function to change, the documentation above will have to be updated.


\appendix
\cleardoublepage
\printindex
\end{document}
